% Generated by Sphinx.
\def\sphinxdocclass{report}
\documentclass[letterpaper,10pt,english]{sphinxmanual}
\usepackage[utf8]{inputenc}
\DeclareUnicodeCharacter{00A0}{\nobreakspace}
\usepackage[T1]{fontenc}
\usepackage{babel}
\usepackage{times}
\usepackage[Bjarne]{fncychap}
\usepackage{longtable}
\usepackage{sphinx}
\usepackage{multirow}


\title{MouseDB Documentation}
\date{December 11, 2011}
\release{1.0.0}
\author{Dave Bridges, Ph.D.}
\newcommand{\sphinxlogo}{}
\renewcommand{\releasename}{Release}
\makeindex

\makeatletter
\def\PYG@reset{\let\PYG@it=\relax \let\PYG@bf=\relax%
    \let\PYG@ul=\relax \let\PYG@tc=\relax%
    \let\PYG@bc=\relax \let\PYG@ff=\relax}
\def\PYG@tok#1{\csname PYG@tok@#1\endcsname}
\def\PYG@toks#1+{\ifx\relax#1\empty\else%
    \PYG@tok{#1}\expandafter\PYG@toks\fi}
\def\PYG@do#1{\PYG@bc{\PYG@tc{\PYG@ul{%
    \PYG@it{\PYG@bf{\PYG@ff{#1}}}}}}}
\def\PYG#1#2{\PYG@reset\PYG@toks#1+\relax+\PYG@do{#2}}

\def\PYG@tok@gd{\def\PYG@tc##1{\textcolor[rgb]{0.63,0.00,0.00}{##1}}}
\def\PYG@tok@gu{\let\PYG@bf=\textbf\def\PYG@tc##1{\textcolor[rgb]{0.50,0.00,0.50}{##1}}}
\def\PYG@tok@gt{\def\PYG@tc##1{\textcolor[rgb]{0.00,0.25,0.82}{##1}}}
\def\PYG@tok@gs{\let\PYG@bf=\textbf}
\def\PYG@tok@gr{\def\PYG@tc##1{\textcolor[rgb]{1.00,0.00,0.00}{##1}}}
\def\PYG@tok@cm{\let\PYG@it=\textit\def\PYG@tc##1{\textcolor[rgb]{0.25,0.50,0.56}{##1}}}
\def\PYG@tok@vg{\def\PYG@tc##1{\textcolor[rgb]{0.73,0.38,0.84}{##1}}}
\def\PYG@tok@m{\def\PYG@tc##1{\textcolor[rgb]{0.13,0.50,0.31}{##1}}}
\def\PYG@tok@mh{\def\PYG@tc##1{\textcolor[rgb]{0.13,0.50,0.31}{##1}}}
\def\PYG@tok@cs{\def\PYG@tc##1{\textcolor[rgb]{0.25,0.50,0.56}{##1}}\def\PYG@bc##1{\colorbox[rgb]{1.00,0.94,0.94}{##1}}}
\def\PYG@tok@ge{\let\PYG@it=\textit}
\def\PYG@tok@vc{\def\PYG@tc##1{\textcolor[rgb]{0.73,0.38,0.84}{##1}}}
\def\PYG@tok@il{\def\PYG@tc##1{\textcolor[rgb]{0.13,0.50,0.31}{##1}}}
\def\PYG@tok@go{\def\PYG@tc##1{\textcolor[rgb]{0.19,0.19,0.19}{##1}}}
\def\PYG@tok@cp{\def\PYG@tc##1{\textcolor[rgb]{0.00,0.44,0.13}{##1}}}
\def\PYG@tok@gi{\def\PYG@tc##1{\textcolor[rgb]{0.00,0.63,0.00}{##1}}}
\def\PYG@tok@gh{\let\PYG@bf=\textbf\def\PYG@tc##1{\textcolor[rgb]{0.00,0.00,0.50}{##1}}}
\def\PYG@tok@ni{\let\PYG@bf=\textbf\def\PYG@tc##1{\textcolor[rgb]{0.84,0.33,0.22}{##1}}}
\def\PYG@tok@nl{\let\PYG@bf=\textbf\def\PYG@tc##1{\textcolor[rgb]{0.00,0.13,0.44}{##1}}}
\def\PYG@tok@nn{\let\PYG@bf=\textbf\def\PYG@tc##1{\textcolor[rgb]{0.05,0.52,0.71}{##1}}}
\def\PYG@tok@no{\def\PYG@tc##1{\textcolor[rgb]{0.38,0.68,0.84}{##1}}}
\def\PYG@tok@na{\def\PYG@tc##1{\textcolor[rgb]{0.25,0.44,0.63}{##1}}}
\def\PYG@tok@nb{\def\PYG@tc##1{\textcolor[rgb]{0.00,0.44,0.13}{##1}}}
\def\PYG@tok@nc{\let\PYG@bf=\textbf\def\PYG@tc##1{\textcolor[rgb]{0.05,0.52,0.71}{##1}}}
\def\PYG@tok@nd{\let\PYG@bf=\textbf\def\PYG@tc##1{\textcolor[rgb]{0.33,0.33,0.33}{##1}}}
\def\PYG@tok@ne{\def\PYG@tc##1{\textcolor[rgb]{0.00,0.44,0.13}{##1}}}
\def\PYG@tok@nf{\def\PYG@tc##1{\textcolor[rgb]{0.02,0.16,0.49}{##1}}}
\def\PYG@tok@si{\let\PYG@it=\textit\def\PYG@tc##1{\textcolor[rgb]{0.44,0.63,0.82}{##1}}}
\def\PYG@tok@s2{\def\PYG@tc##1{\textcolor[rgb]{0.25,0.44,0.63}{##1}}}
\def\PYG@tok@vi{\def\PYG@tc##1{\textcolor[rgb]{0.73,0.38,0.84}{##1}}}
\def\PYG@tok@nt{\let\PYG@bf=\textbf\def\PYG@tc##1{\textcolor[rgb]{0.02,0.16,0.45}{##1}}}
\def\PYG@tok@nv{\def\PYG@tc##1{\textcolor[rgb]{0.73,0.38,0.84}{##1}}}
\def\PYG@tok@s1{\def\PYG@tc##1{\textcolor[rgb]{0.25,0.44,0.63}{##1}}}
\def\PYG@tok@gp{\let\PYG@bf=\textbf\def\PYG@tc##1{\textcolor[rgb]{0.78,0.36,0.04}{##1}}}
\def\PYG@tok@sh{\def\PYG@tc##1{\textcolor[rgb]{0.25,0.44,0.63}{##1}}}
\def\PYG@tok@ow{\let\PYG@bf=\textbf\def\PYG@tc##1{\textcolor[rgb]{0.00,0.44,0.13}{##1}}}
\def\PYG@tok@sx{\def\PYG@tc##1{\textcolor[rgb]{0.78,0.36,0.04}{##1}}}
\def\PYG@tok@bp{\def\PYG@tc##1{\textcolor[rgb]{0.00,0.44,0.13}{##1}}}
\def\PYG@tok@c1{\let\PYG@it=\textit\def\PYG@tc##1{\textcolor[rgb]{0.25,0.50,0.56}{##1}}}
\def\PYG@tok@kc{\let\PYG@bf=\textbf\def\PYG@tc##1{\textcolor[rgb]{0.00,0.44,0.13}{##1}}}
\def\PYG@tok@c{\let\PYG@it=\textit\def\PYG@tc##1{\textcolor[rgb]{0.25,0.50,0.56}{##1}}}
\def\PYG@tok@mf{\def\PYG@tc##1{\textcolor[rgb]{0.13,0.50,0.31}{##1}}}
\def\PYG@tok@err{\def\PYG@bc##1{\fcolorbox[rgb]{1.00,0.00,0.00}{1,1,1}{##1}}}
\def\PYG@tok@kd{\let\PYG@bf=\textbf\def\PYG@tc##1{\textcolor[rgb]{0.00,0.44,0.13}{##1}}}
\def\PYG@tok@ss{\def\PYG@tc##1{\textcolor[rgb]{0.32,0.47,0.09}{##1}}}
\def\PYG@tok@sr{\def\PYG@tc##1{\textcolor[rgb]{0.14,0.33,0.53}{##1}}}
\def\PYG@tok@mo{\def\PYG@tc##1{\textcolor[rgb]{0.13,0.50,0.31}{##1}}}
\def\PYG@tok@mi{\def\PYG@tc##1{\textcolor[rgb]{0.13,0.50,0.31}{##1}}}
\def\PYG@tok@kn{\let\PYG@bf=\textbf\def\PYG@tc##1{\textcolor[rgb]{0.00,0.44,0.13}{##1}}}
\def\PYG@tok@o{\def\PYG@tc##1{\textcolor[rgb]{0.40,0.40,0.40}{##1}}}
\def\PYG@tok@kr{\let\PYG@bf=\textbf\def\PYG@tc##1{\textcolor[rgb]{0.00,0.44,0.13}{##1}}}
\def\PYG@tok@s{\def\PYG@tc##1{\textcolor[rgb]{0.25,0.44,0.63}{##1}}}
\def\PYG@tok@kp{\def\PYG@tc##1{\textcolor[rgb]{0.00,0.44,0.13}{##1}}}
\def\PYG@tok@w{\def\PYG@tc##1{\textcolor[rgb]{0.73,0.73,0.73}{##1}}}
\def\PYG@tok@kt{\def\PYG@tc##1{\textcolor[rgb]{0.56,0.13,0.00}{##1}}}
\def\PYG@tok@sc{\def\PYG@tc##1{\textcolor[rgb]{0.25,0.44,0.63}{##1}}}
\def\PYG@tok@sb{\def\PYG@tc##1{\textcolor[rgb]{0.25,0.44,0.63}{##1}}}
\def\PYG@tok@k{\let\PYG@bf=\textbf\def\PYG@tc##1{\textcolor[rgb]{0.00,0.44,0.13}{##1}}}
\def\PYG@tok@se{\let\PYG@bf=\textbf\def\PYG@tc##1{\textcolor[rgb]{0.25,0.44,0.63}{##1}}}
\def\PYG@tok@sd{\let\PYG@it=\textit\def\PYG@tc##1{\textcolor[rgb]{0.25,0.44,0.63}{##1}}}

\def\PYGZbs{\char`\\}
\def\PYGZus{\char`\_}
\def\PYGZob{\char`\{}
\def\PYGZcb{\char`\}}
\def\PYGZca{\char`\^}
\def\PYGZsh{\char`\#}
\def\PYGZpc{\char`\%}
\def\PYGZdl{\char`\$}
\def\PYGZti{\char`\~}
% for compatibility with earlier versions
\def\PYGZat{@}
\def\PYGZlb{[}
\def\PYGZrb{]}
\makeatother

\begin{document}

\maketitle
\tableofcontents
\phantomsection\label{index::doc}


Contents:


\chapter{MouseDB Concepts}
\label{concepts:welcome-to-mousedb-s-documentation}\label{concepts::doc}\label{concepts:mousedb-concepts}
Data storage for MouseDB is separated into packages which contain information about animals, and information collected about animals.  There is also a separate module for timed matings of animals.  This document will describe the basics of how data is stored in each of these modules.


\section{Animal Module}
\label{concepts:animal-module}
Animals are tracked as individual entities, and given associations to breeding cages to follow ancestry, and strains.  Animals are the main objects in the database, and most other data is linked to and can be accessed from animals.


\subsection{Animal}
\label{concepts:animal}
Most parameters about an animal are set within the {\hyperref[api:mousedb.animal.models.Animal]{\code{Animal}}} object. Here is where the animals strain, breeding, parentage and many other parameters are included.  Animals have foreignkey relationships with both {\hyperref[api:mousedb.animal.models.Strain]{\code{Strain}}} and {\hyperref[api:mousedb.animal.models.Breeding]{\code{Breeding}}}, so an animal may only belong to one of each of those.  As an example, a mouse cannot come from more than one Breeding set, and cannot belong to more than one strain.


\subsubsection{Backcrosses and Generations}
\label{concepts:backcrosses-and-generations}
For this software, optional tracking of backcrosses and generations is available and is stored as an attribute of an animal.  When an inbred cross is made against a pure background, the backcross increases by 1.  When a heterozygote cross is made, the generation increases by one.  As an example, for every time a mouse in a C57/BL6 background is crossed against a wildtype C57/B6 mouse, the backcross (but not the generation) increases by one.  For every time a mutant strain is crosses against itself (either vs a heterozygote or homozygote of that strain), the generation will increase by one.  Backcrosses should typically be performed against a separate colony of purebred mouse, rather than against wild-type alleles of the mutant strain.


\subsection{Breeding Cages}
\label{concepts:breeding-cages}
A breeding cage is defined as a set of one or more male and one or more female mice.  Because of this, it is not always clear who the precise parentage of an animal is.  If the parentage is known, then the Mother and Father fields can be set for a particular animal.  In the case of Active, if an End field is specified, then the Active field is set to False.  In the case of Cage, if a Cage is provided, and animals are specified under Male or Females for a Breeding object, then the Cage field for those animals is set to that of the breeding cage.  The same is true for both Rack and Rack Position.


\subsection{Strains}
\label{concepts:strains}
A strain is a set of mice with a similar genetics.  Importantly strains are separated from Backgrounds.  For example, one might have mice with the genotype ob/ob but these mice may be in either a C57-Black6 or a mixed background.  This difference is set at the individual animal level.
The result of this is that a query for a particular strain may then need to be filtered to a specific background.


\section{Data Module}
\label{concepts:data-module}
Data (or measurements) can be stored for any type of measurement.  Conceptually, several pieces of data belong to an experiment (for example several mice are measured at some time) and several experiments belong to a study.  Measurements can be stored independent of experiments and experiments can be performed outside of the context of a study.  It is however, perfered that measurements are stored within an experiment and experiments are stored within studies as this will greatly facilitate the organization of the data.


\subsection{Studies}
\label{concepts:studies}
In general studies are a collection of experiments.  These can be grouped together on the basis of animals and/or treatment groups.  A study must have at least one treatment group, which defines the animals and their conditions.


\subsection{Experiments}
\label{concepts:experiments}
An experiment is a collection of measurements for a given set of animals.  In general, an experiment is defined as a number of measurements take in a given day.


\subsection{Measurements}
\label{concepts:measurements}
A measurement is an animal, an assay and a measurement value.  It can be associated with an experiment, or can stand alone as an individual value.  Measurements can be viewed in the context of a study, an experiment, a treatment group or an animal by going to the appropriate page.


\section{Timed Matings Module}
\label{concepts:timed-matings-module}
Timed matings are a specific type of breeding set.  Generally, for these experiments a mating cage is set up and pregnancy is defined by a plug event.  Based on this information, the age of an embryo can be estimated.  When a breeding cage is defined, one option is to set this cage as a timed mating cage (ie Timed\_Mating=True).  If this is the case, then a plug event can be registered and recorded for this mating set.  If the mother gives birth then this cage is implicitly set as a normal breeding cage.


\section{Groups Module}
\label{concepts:groups-module}
This app defines generic Group and License information for a particular installation of MouseDB.  Because every page on this site identifies both the Group and data restrictions, at a minimum, group information must be provided upon installation (see installation instructions).


\chapter{MouseDB Installation}
\label{installation:mousedb-installation}\label{installation::doc}

\section{Configuration}
\label{installation:configuration}
MouseDB requires both a database and a webserver to be set up.  Ideally, the database should be hosted separately from the webserver and MouseDB installation, but this is not necessary, as both can be used from the same server.  If you are using a remote server for the database, it is best to set up a user for this database that can only be accessed from the webserver.  If you want to set up several installations (ie for different users or different laboratories), you need separate databases and MouseDB installations for each.  You will also need to set up the webserver with different addresses for each installation.


\section{Software Dependencies}
\label{installation:software-dependencies}\begin{enumerate}
\item {} 
\textbf{Python}.  Requires Version 2.6, is not yet compatible with Python 3.0.  Download from \href{http://www.python.org/download/}{http://www.python.org/download/}.

\item {} 
\textbf{MouseDB source code}.  Download from one of the following:

\end{enumerate}
\begin{enumerate}
\item {} 
Using \textbf{pip or easy\_install}.  If setuptools (available at \href{http://pypi.python.org/pypi/setuptools}{http://pypi.python.org/pypi/setuptools}) is installed type \textbf{pip install mousedb} at a command prompt.

\item {} 
\href{http://github.com/davebridges/mousedb/downloads}{http://github.com/davebridges/mousedb/downloads} for the current release.  If you will not be contributing to the code, download from here.

\item {} 
\href{http://github.com/davebridges/mousedb}{http://github.com/davebridges/mousedb} for the source code via Git.  If you might contribute code to the project use the source code.

\end{enumerate}

Downloading and/or unzipping will create a directory named mousedb.  You can update to the newest revision at any time either using git or downloading and re-installing the newer version.  Changing or updating software versions will not alter any saved data, but you will have to update the localsettings.py file (described below).
\begin{enumerate}
\setcounter{enumi}{2}
\item {} 
\textbf{Database software}.  Recommended to use mysql, available at \href{http://dev.mysql.com/downloads/mysql/}{http://dev.mysql.com/downloads/mysql/} .  It is also possible to use SQLite, PostgreSQL, MySQL, or Oracle.  See \href{http://docs.djangoproject.com/en/1.2/topics/install/\#database-installation}{http://docs.djangoproject.com/en/1.2/topics/install/\#database-installation} for more information.  You will also need the python bindings for your database.  If using MySQL python-mysql will be installed below.

\item {} 
\textbf{Webserver}.  Apache is recommended, available at \href{http://www.apache.org/dyn/closer.cgi}{http://www.apache.org/dyn/closer.cgi} .  It is also possible to use FastCGI, SCGI, or AJP.  See \href{http://docs.djangoproject.com/en/1.2/howto/deployment/}{http://docs.djangoproject.com/en/1.2/howto/deployment/} for more details.  The recommended way to use Apache is to download and enable mod\_wsgi.  See \href{http://code.google.com/p/modwsgi/}{http://code.google.com/p/modwsgi/} for more details.

\end{enumerate}


\section{Installation}
\label{installation:installation}\begin{enumerate}
\item {} 
Navigate into mousedb folder

\item {} 
Run \textbf{python setup.py install} to get dependencies.  If you installed via pip, this step is not necessary (but wont hurt).  This will install the dependencies South, mysql-python and django-ajax-selects.

\item {} 
Run \textbf{python bootstrap.py} to get the correct version of Django and to set up an isolated environment.  This step may take a few minutes.

\item {} 
Run \textbf{bin\textbackslash{}buildout} to generate django, test and wsgi scripts.  This step may take a few minutes.

\end{enumerate}


\section{Database Setup}
\label{installation:database-setup}\begin{enumerate}
\item {} 
Create a new database.  Check the documentation for your database software for the appropriate syntax for this step.  You need to record the user, password, host and database name.  If you are using SQLite this step is not required.

\item {} 
Go to mousedbsrcmousedblocalsettings\_empty.py and edit the settings:

\end{enumerate}
\begin{itemize}
\item {} 
ENGINE: Choose one of `django.db.backends.postgresql\_psycopg2','django.db.backends.postgresql', `django.db.backends.mysql', `django.db.backends.sqlite3', `django.db.backends.oracle' depending on the database software used.

\item {} 
NAME: database name

\item {} 
USER: database user

\item {} 
PASSWORD: database password

\item {} 
HOST: database host

\end{itemize}
\begin{enumerate}
\setcounter{enumi}{2}
\item {} 
Save this file as \textbf{localsettings.py} in the same folder as localsettings\_empty.py

\item {} 
Migrate into first mousedb directory and enter \emph{django syncdb}.  When prompted create a superuser (who will have all availabler permissions) and a password for this user.

\end{enumerate}


\section{Web Server Setup}
\label{installation:web-server-setup}
You need to set up a server to serve both the django installation and saved files.  For the saved files.  I recommend using apache for both.  The preferred setup is to use Apache2 with mod\_wsgi.  See \href{http://code.google.com/p/modwsgi/wiki/InstallationInstructions}{http://code.google.com/p/modwsgi/wiki/InstallationInstructions} for instructions on using mod\_wsgi.  The following is a httpd.conf example where the code is placed in \textbf{/usr/src/mousedb}:

\begin{Verbatim}[commandchars=\\\{\}]
Alias /robots.txt /usr/src/mousedb/src/mousedb/media/robots.txt
Alias /favicon.ico /usr/src/mousedb/src/mousedb/media/favicon.ico

Alias /mousedb-media/ /usr/src/mousedb/src/mousedb/media/
\textless{}Directory /usr/src/mousedb/src/mousedb/media\textgreater{}
     Order deny,allow
     Allow from all
\textless{}/Directory\textgreater{}

Alias /static/ /usr/src/mousedb/src/mousedb/static/
\textless{}Directory /usr/src/mousedb/src/mousedb/static\textgreater{}
     Order deny,allow
     Allow from all
\textless{}/Directory\textgreater{}

\textless{}Directory /usr/src/mousedb/bin\textgreater{}
     Order deny,allow
     Allow from all
\textless{}/Directory\textgreater{}
WSGIScriptAlias /mousedb /usr/src/mousedb/bin/django.wsgi
\end{Verbatim}

If you want to restrict access to these files, change the Allow from all directive to specific domains or ip addresses (for example Allow from 192.168.0.0/99 would allow from 192.168.0.0 to 192.168.0.99)


\section{Enabling of South for Future Migrations}
\label{installation:enabling-of-south-for-future-migrations}
Schema updates will utilize south as a way to alter database tables.  This must be enabled initially by entering the following commands from /mousedb/bin:

\begin{Verbatim}[commandchars=\\\{\}]
django schemamigration animal --initial
django schemamigration data --initial
django schemamigration groups --initial
django schemamigration timed\_mating --initial
django syncdb
django migrate
\end{Verbatim}

Future schema changes (se the UPGRADE\_NOTES.rst file for whether this is necessary) are accomplished by entering:

\begin{Verbatim}[commandchars=\\\{\}]
django schemamigration \textless{}INDICATED\_APP\textgreater{} --auto
django migrate \textless{}INDICATED\_APP\textgreater{}
\end{Verbatim}


\section{Final Configuration and User Setup}
\label{installation:final-configuration-and-user-setup}\begin{itemize}
\item {} 
Go to \emph{servername/mousedb/admin/groups/group/1} and name your research group and select a license if desired

\item {} 
Go to \emph{servername/mousedb/admin/auth/users/} and create users, selecting usernames, full names, password (or have the user set the password) and then choose group permissions.

\end{itemize}


\section{Testing}
\label{installation:testing}
From the mousedb directory run \textbf{bin\textbackslash{}test} or \textbf{bin\textbackslash{}django test} to run the test suite.  See \href{https://github.com/davebridges/mousedb/wiki/Known-Issues---Test-Suite}{https://github.com/davebridges/mousedb/wiki/Known-Issues---Test-Suite} for known issues.  Report any additional errors at the issue page at \href{https://github.com/davebridges/mousedb/issues}{https://github.com/davebridges/mousedb/issues}.


\chapter{Users and Restriction}
\label{usage:users-and-restriction}\label{usage::doc}
All pages in this database are restricted to logged-in users.  Users are defined using the standard Django \href{http://docs.djangoproject.com/en/dev/topics/auth/\#django.contrib.auth.models.User}{\code{User}} objects.  It is also recommended that data is secured by only allowing access of specific IP addresses.  For more details on this see the documentation for your webserver software (for example, for Apache see here \href{http://httpd.apache.org/docs/2.2/howto/access.html}{http://httpd.apache.org/docs/2.2/howto/access.html}).  Each database should have at least one superuser, and that user can create and designate permissions for other users.  When a user does not have the permissions to view a page or to edit something, the link to that page will not be visible and if they enter the address, they will be redirected to a login page.


\section{Creating New Users}
\label{usage:creating-new-users}
Create users by going to \textbf{../mousedb/admin/auth/user/add/} and filling in both pages of the form.  Permissions are set by going to \textbf{../mousedb/admin/auth/user/} selecting the user and manually moving the permissions from the box on the left to the box on the right.  If you want the user to have access to the admin site, select staff.  Only select superuser if you want that user to have all permissions.  Users can change passwords on the administration site as well.


\section{Removing Users}
\label{usage:removing-users}
Remove inactive users by selecting their username from the \textbf{../mousedb/admin/auth/user/} page and deselecting the active box.  Only delete a user if it was generated mistakenly and that the particular user had not been used to edit any data.


\chapter{Animal Data Entry}
\label{usage:animal-data-entry}

\section{Newborn Mice or Newly Weaned Mice}
\label{usage:newborn-mice-or-newly-weaned-mice}\begin{enumerate}
\item {} 
Go to Breeding Cages Tab

\item {} 
Click on Add/Wean Pups Button

\item {} 
Each row is a new animal.  If you accidentaly enter an extra animal, check off the delete box then submit.

\item {} 
Leave extra lines blank if you have less than 10 mice to enter

\item {} 
If you need to enter more than 10 mice, enter the first ten and submit them.  Go back and enter up to 10 more animals (10 more blank spaces will appear)

\end{enumerate}


\section{Newborn Mice}
\label{usage:newborn-mice}\begin{enumerate}
\item {} 
Enter Breeding Cage under Cage

\item {} 
Enter Strain

\item {} 
Enter Background (normally Mixed or C57BL/6-BA unless from the LY breeding cages in which case it is C57BL/6-LY5.2)

\item {} 
Enter Birthdate in format YYYY-MM-DD

\item {} 
Enter Generation and Backcross

\end{enumerate}


\section{Weaning Mice}
\label{usage:weaning-mice}\begin{enumerate}
\item {} 
If not previously entered, enter data as if newborn mice

\item {} 
Enter gender

\item {} 
Enter Wean Date in format YYYY-MM-DD

\item {} 
Enter new Cage number for Cage

\end{enumerate}


\section{Cage Changes (Not Weaning)}
\label{usage:cage-changes-not-weaning}\begin{enumerate}
\item {} 
Find mouse either from animal list or strain list

\item {} 
Click the edit mouse button

\item {} 
Change the Cage, Rack and Rack Position as Necessary

\end{enumerate}


\section{Genotyping or Ear Tagging}
\label{usage:genotyping-or-ear-tagging}\begin{enumerate}
\item {} 
Find mouse either from animal list or strain list, or through breeding cage

\item {} 
Click the edit mouse button or the Eartag/Genotype/Cage Change/Death Button

\item {} 
Enter the Ear Tag and/or select the Genotype from the Pull Down List

\end{enumerate}


\section{Marking Mice as Dead}
\label{usage:marking-mice-as-dead}

\subsection{Dead Mice (Single Mouse)}
\label{usage:dead-mice-single-mouse}\begin{enumerate}
\item {} 
Find mouse from animal list or strain list

\item {} 
Click the edit mouse button

\item {} 
Enter the death date in format YYYY-MM-DD

\item {} 
Choose Cause of Death from Pull Down List

\end{enumerate}


\subsection{Dead Mice (Several Mice)}
\label{usage:dead-mice-several-mice}\begin{enumerate}
\item {} 
Find mice from breeding cages

\item {} 
Click the Eartag/Genotype/Cage Change/Death Button

\item {} 
Enter the death date in format YYYY-MM-DD

\item {} 
Choose the Cause of Death from Pull Down List

\end{enumerate}


\chapter{Studies and Experimental Setup}
\label{usage:studies-and-experimental-setup}
Set up a new study at /mousedb/admin/data/study/ selecting animals

You must put a description and select animals in one or more treatment groups

If you have more than 2 treatment groups save the first two, then two more empty slots will appear. For animals, click on the magnifying glass then find the animal in that treatment group and click on the MouseID. The number displayed now in that field will not be the MouseID, but don't worry its just a different number to describe the mouse. To add more animals, click on the magnifying glass again and select the next animal. There should be now two numbers, separated by commas in this field. Repeat to fill all your treatment groups. You must enter a diet and environment for each treatment. The other fields are optional, and should only be used if appropriate. Ensure for pharmaceutical, you include a saline treatment group.


\chapter{Measurement Entry}
\label{usage:measurement-entry}

\section{Studies}
\label{usage:studies}
If this measurement is part of a study (ie a group of experiments) then click on the plus sign beside the study field and enter in the details about the study and treatment groups.  Unfortunately until i can figure out how to filter the treatment group animals in the admin interface, at each of the subsequent steps you will see all the animals in the database (soon hopefully it will only be the ones as part of the study group).


\section{Experiment Details}
\label{usage:experiment-details}\begin{itemize}
\item {} 
Pick experiment date, feeding state and resarchers

\item {} 
Pick animals used in this experiment (the search box will filter results)

\item {} 
Fasting state, time, injections, concentration, experimentID and notes are all optional

\end{itemize}


\section{Measurements}
\label{usage:measurements}\begin{itemize}
\item {} 
There is room to enter 14 measurements.  If you need more rows, enter the first 14 and select ``Save and Continue Editing'' and 14 more blank spots will appear.

\item {} 
Each row is a measurement, so if you have glucose and weight for some animal that is two rows entered.

\item {} 
For animals, click on the magnifying glass then find the animal in that treatment group and click on the MouseID. The number displayed now in that field will not be the MouseID, but don't worry its just a different number to describe the mouse.

\item {} 
For values, the standard units (defined by each assay) are mg for weights, mg/dL for glucose and pg/mL for insulin).  You must enter integers here (no decimal places).  If you have several measurements (ie several glucose readings during a GTT, enter them all in one measurement row, separated by commas and \emph{NO spaces}).

\end{itemize}


\chapter{Automated Documentation}
\label{api:automated-documentation}\label{api::doc}

\section{Root Package}
\label{api:root-package}\label{api:module-mousedb}\index{mousedb (module)}
MouseDB is a data management and analysis system for experimental animals.  Source code is freely available via Github (through the BSD License please see LICENSE file or \href{http://www.opensource.org/licenses/bsd-license.php}{http://www.opensource.org/licenses/bsd-license.php}), and collaboration is encouraged.  For specific details please contact Dave Bridges via Github.  MouseDB uses a web interface and a database server to store information and a web interface to access and analyse this information.  The standard setup is to use MySQL as the database and Apache as the webserver, but this can be modified if necessary.  The software was written using Django, which itself is based on the Python programming language.  Please see www.djangoproject.com and www.python.org for more information.


\subsection{Views and URLs}
\label{api:views-and-urls}\label{api:module-mousedb.views}\index{mousedb.views (module)}
This package defines simple root views.

Currently this package includes views for both the logout and home pages.
\index{ProtectedDetailView (class in mousedb.views)}

\begin{fulllineitems}
\phantomsection\label{api:mousedb.views.ProtectedDetailView}\pysiglinewithargsret{\strong{class }\code{mousedb.views.}\bfcode{ProtectedDetailView}}{\emph{**kwargs}}{}
Bases: \href{http://docs.djangoproject.com/en/dev/ref/class-based-views/\#django.views.generic.detail.DetailView}{\code{django.views.generic.detail.DetailView}}

This subclass of DetailView generates a login\_required protected version of the DetailView.

This ProtectedDetailView is then subclassed instead of using ListView for login\_required views.
\index{dispatch() (mousedb.views.ProtectedDetailView method)}

\begin{fulllineitems}
\phantomsection\label{api:mousedb.views.ProtectedDetailView.dispatch}\pysiglinewithargsret{\bfcode{dispatch}}{\emph{*args}, \emph{**kwargs}}{}
\end{fulllineitems}


\end{fulllineitems}

\index{ProtectedListView (class in mousedb.views)}

\begin{fulllineitems}
\phantomsection\label{api:mousedb.views.ProtectedListView}\pysiglinewithargsret{\strong{class }\code{mousedb.views.}\bfcode{ProtectedListView}}{\emph{**kwargs}}{}
Bases: \href{http://docs.djangoproject.com/en/dev/ref/class-based-views/\#django.views.generic.list.ListView}{\code{django.views.generic.list.ListView}}

This subclass of ListView generates a login\_required protected version of the ListView.

This ProtectedListView is then subclassed instead of using ListView for login\_required views.
\index{dispatch() (mousedb.views.ProtectedListView method)}

\begin{fulllineitems}
\phantomsection\label{api:mousedb.views.ProtectedListView.dispatch}\pysiglinewithargsret{\bfcode{dispatch}}{\emph{*args}, \emph{**kwargs}}{}
\end{fulllineitems}


\end{fulllineitems}

\index{RestrictedCreateView (class in mousedb.views)}

\begin{fulllineitems}
\phantomsection\label{api:mousedb.views.RestrictedCreateView}\pysiglinewithargsret{\strong{class }\code{mousedb.views.}\bfcode{RestrictedCreateView}}{\emph{**kwargs}}{}
Bases: \href{http://docs.djangoproject.com/en/dev/ref/class-based-views/\#django.views.generic.edit.CreateView}{\code{django.views.generic.edit.CreateView}}

Generic create view that checks permissions.

This is from \href{http://djangosnippets.org/snippets/2317/}{http://djangosnippets.org/snippets/2317/} and subclasses the UpdateView into one that requires permissions to create a particular model.
\index{dispatch() (mousedb.views.RestrictedCreateView method)}

\begin{fulllineitems}
\phantomsection\label{api:mousedb.views.RestrictedCreateView.dispatch}\pysiglinewithargsret{\bfcode{dispatch}}{\emph{request}, \emph{*args}, \emph{**kwargs}}{}
\end{fulllineitems}


\end{fulllineitems}

\index{RestrictedDeleteView (class in mousedb.views)}

\begin{fulllineitems}
\phantomsection\label{api:mousedb.views.RestrictedDeleteView}\pysiglinewithargsret{\strong{class }\code{mousedb.views.}\bfcode{RestrictedDeleteView}}{\emph{**kwargs}}{}
Bases: \href{http://docs.djangoproject.com/en/dev/ref/class-based-views/\#django.views.generic.edit.DeleteView}{\code{django.views.generic.edit.DeleteView}}

Generic delete view that checks permissions.

This is from \href{http://djangosnippets.org/snippets/2317/}{http://djangosnippets.org/snippets/2317/} and subclasses the UpdateView into one that requires permissions to delete a particular model.
\index{dispatch() (mousedb.views.RestrictedDeleteView method)}

\begin{fulllineitems}
\phantomsection\label{api:mousedb.views.RestrictedDeleteView.dispatch}\pysiglinewithargsret{\bfcode{dispatch}}{\emph{request}, \emph{*args}, \emph{**kwargs}}{}
\end{fulllineitems}


\end{fulllineitems}

\index{RestrictedUpdateView (class in mousedb.views)}

\begin{fulllineitems}
\phantomsection\label{api:mousedb.views.RestrictedUpdateView}\pysiglinewithargsret{\strong{class }\code{mousedb.views.}\bfcode{RestrictedUpdateView}}{\emph{**kwargs}}{}
Bases: \href{http://docs.djangoproject.com/en/dev/ref/class-based-views/\#django.views.generic.edit.UpdateView}{\code{django.views.generic.edit.UpdateView}}

Generic update view that checks permissions.

This is from \href{http://djangosnippets.org/snippets/2317/}{http://djangosnippets.org/snippets/2317/} and subclasses the UpdateView into one that requires permissions to update a particular model.
\index{dispatch() (mousedb.views.RestrictedUpdateView method)}

\begin{fulllineitems}
\phantomsection\label{api:mousedb.views.RestrictedUpdateView.dispatch}\pysiglinewithargsret{\bfcode{dispatch}}{\emph{request}, \emph{*args}, \emph{**kwargs}}{}
\end{fulllineitems}


\end{fulllineitems}

\index{home() (in module mousedb.views)}

\begin{fulllineitems}
\phantomsection\label{api:mousedb.views.home}\pysiglinewithargsret{\code{mousedb.views.}\bfcode{home}}{\emph{request}, \emph{*args}, \emph{**kwargs}}{}
This view generates the data for the home page.

This login restricted view passes dictionaries containing the current cages, animals and strains as well as the totals for each.  This data is passed to the template home.html

\end{fulllineitems}

\index{logout\_view() (in module mousedb.views)}

\begin{fulllineitems}
\phantomsection\label{api:mousedb.views.logout_view}\pysiglinewithargsret{\code{mousedb.views.}\bfcode{logout\_view}}{\emph{request}}{}
This view logs out the current user.

It redirects the user to the `/index/' page which in turn should redirect to the login page.

\end{fulllineitems}

\phantomsection\label{api:module-mousedb.urls}\index{mousedb.urls (module)}
Generic base url directives.

These directives will redirect requests to app specific pages, and provide redundancy in possible names.


\subsection{Test Files}
\label{api:test-files}\label{api:module-mousedb.tests}\index{mousedb.tests (module)}
This file contains tests for the root application.

These tests will verify function of the home and logout views.
\index{RootViewTests (class in mousedb.tests)}

\begin{fulllineitems}
\phantomsection\label{api:mousedb.tests.RootViewTests}\pysiglinewithargsret{\strong{class }\code{mousedb.tests.}\bfcode{RootViewTests}}{\emph{methodName='runTest'}}{}
Bases: \code{django.test.testcases.TestCase}

These are tests for the root views.  Included are tests for home and logout.
\index{fixtures (mousedb.tests.RootViewTests attribute)}

\begin{fulllineitems}
\phantomsection\label{api:mousedb.tests.RootViewTests.fixtures}\pysigline{\bfcode{fixtures}\strong{ = {[}'test\_breeding', `test\_animals', `test\_strain'{]}}}
\end{fulllineitems}

\index{setUp() (mousedb.tests.RootViewTests method)}

\begin{fulllineitems}
\phantomsection\label{api:mousedb.tests.RootViewTests.setUp}\pysiglinewithargsret{\bfcode{setUp}}{}{}
\end{fulllineitems}

\index{tearDown() (mousedb.tests.RootViewTests method)}

\begin{fulllineitems}
\phantomsection\label{api:mousedb.tests.RootViewTests.tearDown}\pysiglinewithargsret{\bfcode{tearDown}}{}{}
\end{fulllineitems}

\index{test\_home() (mousedb.tests.RootViewTests method)}

\begin{fulllineitems}
\phantomsection\label{api:mousedb.tests.RootViewTests.test_home}\pysiglinewithargsret{\bfcode{test\_home}}{}{}
This test checks the view which displays the home page.  It checks for the correct templates and status code.

\end{fulllineitems}

\index{test\_logout() (mousedb.tests.RootViewTests method)}

\begin{fulllineitems}
\phantomsection\label{api:mousedb.tests.RootViewTests.test_logout}\pysiglinewithargsret{\bfcode{test\_logout}}{}{}
This test checks the view which displays the logout page.  It checks for the correct templates and status code.

\end{fulllineitems}


\end{fulllineitems}



\section{Data Package}
\label{api:data-package}\label{api:module-mousedb.data}\index{mousedb.data (module)}
The data module describes the conditions and collection of data regarding experimental animals.

Data (or measurements) can be stored for any type of measurement.  Conceptually, several pieces of data belong to an experiment (for example several mice are measured at some time) and several experiments belong to a study.  Measurements can be stored independent of experiments and experiments can be performed outside of the context of a study.  It is however, perfered that measurements are stored within an experiment and experiments are stored within studies as this will greatly facilitate the organization of the data.


\subsection{Studies}
\label{api:studies}
In general studies are a collection of experiments.  These can be grouped together on the basis of animals and/or treatment groups.  A study must have at least one treatment group, which defines the animals and their conditions.


\subsection{Experiments}
\label{api:experiments}
An experiment is a collection of measurements for a given set of animals.  In general, an experiment is defined as a number of measurements take in a given day.


\subsection{Measurements}
\label{api:measurements}
A measurement is an animal, an assay and a measurement value.  It can be associated with an experiment, or can stand alone as an individual value.  Measurements can be viewed in the context of a study, an experiment, a treatment group or an animal by going to the appropriate page.


\subsection{Models}
\label{api:module-mousedb.data.models}\label{api:models}\index{mousedb.data.models (module)}\index{Assay (class in mousedb.data.models)}

\begin{fulllineitems}
\phantomsection\label{api:mousedb.data.models.Assay}\pysiglinewithargsret{\strong{class }\code{mousedb.data.models.}\bfcode{Assay}}{\emph{*args}, \emph{**kwargs}}{}
Bases: \code{django.db.models.base.Model}

Assay(id, assay, assay\_slug, notes, measurement\_units)
\index{Assay.DoesNotExist}

\begin{fulllineitems}
\phantomsection\label{api:mousedb.data.models.Assay.DoesNotExist}\pysigline{\strong{exception }\bfcode{DoesNotExist}}
Bases: \code{django.core.exceptions.ObjectDoesNotExist}

\end{fulllineitems}

\index{Assay.MultipleObjectsReturned}

\begin{fulllineitems}
\phantomsection\label{api:mousedb.data.models.Assay.MultipleObjectsReturned}\pysigline{\strong{exception }\code{Assay.}\bfcode{MultipleObjectsReturned}}
Bases: \code{django.core.exceptions.MultipleObjectsReturned}

\end{fulllineitems}

\index{measurement\_set (mousedb.data.models.Assay attribute)}

\begin{fulllineitems}
\phantomsection\label{api:mousedb.data.models.Assay.measurement_set}\pysigline{\code{Assay.}\bfcode{measurement\_set}}
\end{fulllineitems}

\index{objects (mousedb.data.models.Assay attribute)}

\begin{fulllineitems}
\phantomsection\label{api:mousedb.data.models.Assay.objects}\pysigline{\code{Assay.}\bfcode{objects}\strong{ = \textless{}django.db.models.manager.Manager object at 0x022FE5B0\textgreater{}}}
\end{fulllineitems}


\end{fulllineitems}

\index{Diet (class in mousedb.data.models)}

\begin{fulllineitems}
\phantomsection\label{api:mousedb.data.models.Diet}\pysiglinewithargsret{\strong{class }\code{mousedb.data.models.}\bfcode{Diet}}{\emph{*args}, \emph{**kwargs}}{}
Bases: \code{django.db.models.base.Model}

Diet(id, vendor\_id, description, product\_id, fat\_content, protein\_content, carb\_content, irradiated, notes)
\index{Diet.DoesNotExist}

\begin{fulllineitems}
\phantomsection\label{api:mousedb.data.models.Diet.DoesNotExist}\pysigline{\strong{exception }\bfcode{DoesNotExist}}
Bases: \code{django.core.exceptions.ObjectDoesNotExist}

\end{fulllineitems}

\index{Diet.MultipleObjectsReturned}

\begin{fulllineitems}
\phantomsection\label{api:mousedb.data.models.Diet.MultipleObjectsReturned}\pysigline{\strong{exception }\code{Diet.}\bfcode{MultipleObjectsReturned}}
Bases: \code{django.core.exceptions.MultipleObjectsReturned}

\end{fulllineitems}

\index{objects (mousedb.data.models.Diet attribute)}

\begin{fulllineitems}
\phantomsection\label{api:mousedb.data.models.Diet.objects}\pysigline{\code{Diet.}\bfcode{objects}\strong{ = \textless{}django.db.models.manager.Manager object at 0x02312B90\textgreater{}}}
\end{fulllineitems}

\index{treatment\_set (mousedb.data.models.Diet attribute)}

\begin{fulllineitems}
\phantomsection\label{api:mousedb.data.models.Diet.treatment_set}\pysigline{\code{Diet.}\bfcode{treatment\_set}}
\end{fulllineitems}

\index{vendor (mousedb.data.models.Diet attribute)}

\begin{fulllineitems}
\phantomsection\label{api:mousedb.data.models.Diet.vendor}\pysigline{\code{Diet.}\bfcode{vendor}}
\end{fulllineitems}


\end{fulllineitems}

\index{Environment (class in mousedb.data.models)}

\begin{fulllineitems}
\phantomsection\label{api:mousedb.data.models.Environment}\pysiglinewithargsret{\strong{class }\code{mousedb.data.models.}\bfcode{Environment}}{\emph{*args}, \emph{**kwargs}}{}
Bases: \code{django.db.models.base.Model}

Environment(id, building, room, temperature, humidity, notes)
\index{Environment.DoesNotExist}

\begin{fulllineitems}
\phantomsection\label{api:mousedb.data.models.Environment.DoesNotExist}\pysigline{\strong{exception }\bfcode{DoesNotExist}}
Bases: \code{django.core.exceptions.ObjectDoesNotExist}

\end{fulllineitems}

\index{Environment.MultipleObjectsReturned}

\begin{fulllineitems}
\phantomsection\label{api:mousedb.data.models.Environment.MultipleObjectsReturned}\pysigline{\strong{exception }\code{Environment.}\bfcode{MultipleObjectsReturned}}
Bases: \code{django.core.exceptions.MultipleObjectsReturned}

\end{fulllineitems}

\index{contact (mousedb.data.models.Environment attribute)}

\begin{fulllineitems}
\phantomsection\label{api:mousedb.data.models.Environment.contact}\pysigline{\code{Environment.}\bfcode{contact}}
\end{fulllineitems}

\index{objects (mousedb.data.models.Environment attribute)}

\begin{fulllineitems}
\phantomsection\label{api:mousedb.data.models.Environment.objects}\pysigline{\code{Environment.}\bfcode{objects}\strong{ = \textless{}django.db.models.manager.Manager object at 0x0231B4D0\textgreater{}}}
\end{fulllineitems}

\index{treatment\_set (mousedb.data.models.Environment attribute)}

\begin{fulllineitems}
\phantomsection\label{api:mousedb.data.models.Environment.treatment_set}\pysigline{\code{Environment.}\bfcode{treatment\_set}}
\end{fulllineitems}


\end{fulllineitems}

\index{Experiment (class in mousedb.data.models)}

\begin{fulllineitems}
\phantomsection\label{api:mousedb.data.models.Experiment}\pysiglinewithargsret{\strong{class }\code{mousedb.data.models.}\bfcode{Experiment}}{\emph{*args}, \emph{**kwargs}}{}
Bases: \code{django.db.models.base.Model}

Experiment(id, date, notes, experimentID, feeding\_state, fasting\_time, injection, concentration, study\_id)
\index{Experiment.DoesNotExist}

\begin{fulllineitems}
\phantomsection\label{api:mousedb.data.models.Experiment.DoesNotExist}\pysigline{\strong{exception }\bfcode{DoesNotExist}}
Bases: \code{django.core.exceptions.ObjectDoesNotExist}

\end{fulllineitems}

\index{Experiment.MultipleObjectsReturned}

\begin{fulllineitems}
\phantomsection\label{api:mousedb.data.models.Experiment.MultipleObjectsReturned}\pysigline{\strong{exception }\code{Experiment.}\bfcode{MultipleObjectsReturned}}
Bases: \code{django.core.exceptions.MultipleObjectsReturned}

\end{fulllineitems}

\index{get\_absolute\_url() (mousedb.data.models.Experiment method)}

\begin{fulllineitems}
\phantomsection\label{api:mousedb.data.models.Experiment.get_absolute_url}\pysiglinewithargsret{\code{Experiment.}\bfcode{get\_absolute\_url}}{\emph{*moreargs}, \emph{**morekwargs}}{}
\end{fulllineitems}

\index{get\_feeding\_state\_display() (mousedb.data.models.Experiment method)}

\begin{fulllineitems}
\phantomsection\label{api:mousedb.data.models.Experiment.get_feeding_state_display}\pysiglinewithargsret{\code{Experiment.}\bfcode{get\_feeding\_state\_display}}{\emph{*moreargs}, \emph{**morekwargs}}{}
\end{fulllineitems}

\index{get\_injection\_display() (mousedb.data.models.Experiment method)}

\begin{fulllineitems}
\phantomsection\label{api:mousedb.data.models.Experiment.get_injection_display}\pysiglinewithargsret{\code{Experiment.}\bfcode{get\_injection\_display}}{\emph{*moreargs}, \emph{**morekwargs}}{}
\end{fulllineitems}

\index{get\_next\_by\_date() (mousedb.data.models.Experiment method)}

\begin{fulllineitems}
\phantomsection\label{api:mousedb.data.models.Experiment.get_next_by_date}\pysiglinewithargsret{\code{Experiment.}\bfcode{get\_next\_by\_date}}{\emph{*moreargs}, \emph{**morekwargs}}{}
\end{fulllineitems}

\index{get\_previous\_by\_date() (mousedb.data.models.Experiment method)}

\begin{fulllineitems}
\phantomsection\label{api:mousedb.data.models.Experiment.get_previous_by_date}\pysiglinewithargsret{\code{Experiment.}\bfcode{get\_previous\_by\_date}}{\emph{*moreargs}, \emph{**morekwargs}}{}
\end{fulllineitems}

\index{measurement\_set (mousedb.data.models.Experiment attribute)}

\begin{fulllineitems}
\phantomsection\label{api:mousedb.data.models.Experiment.measurement_set}\pysigline{\code{Experiment.}\bfcode{measurement\_set}}
\end{fulllineitems}

\index{objects (mousedb.data.models.Experiment attribute)}

\begin{fulllineitems}
\phantomsection\label{api:mousedb.data.models.Experiment.objects}\pysigline{\code{Experiment.}\bfcode{objects}\strong{ = \textless{}django.db.models.manager.Manager object at 0x022FE2F0\textgreater{}}}
\end{fulllineitems}

\index{researchers (mousedb.data.models.Experiment attribute)}

\begin{fulllineitems}
\phantomsection\label{api:mousedb.data.models.Experiment.researchers}\pysigline{\code{Experiment.}\bfcode{researchers}}
\end{fulllineitems}

\index{study (mousedb.data.models.Experiment attribute)}

\begin{fulllineitems}
\phantomsection\label{api:mousedb.data.models.Experiment.study}\pysigline{\code{Experiment.}\bfcode{study}}
\end{fulllineitems}


\end{fulllineitems}

\index{Implantation (class in mousedb.data.models)}

\begin{fulllineitems}
\phantomsection\label{api:mousedb.data.models.Implantation}\pysiglinewithargsret{\strong{class }\code{mousedb.data.models.}\bfcode{Implantation}}{\emph{*args}, \emph{**kwargs}}{}
Bases: \code{django.db.models.base.Model}

Implantation(id, implant, vendor\_id, product\_id, notes)
\index{Implantation.DoesNotExist}

\begin{fulllineitems}
\phantomsection\label{api:mousedb.data.models.Implantation.DoesNotExist}\pysigline{\strong{exception }\bfcode{DoesNotExist}}
Bases: \code{django.core.exceptions.ObjectDoesNotExist}

\end{fulllineitems}

\index{Implantation.MultipleObjectsReturned}

\begin{fulllineitems}
\phantomsection\label{api:mousedb.data.models.Implantation.MultipleObjectsReturned}\pysigline{\strong{exception }\code{Implantation.}\bfcode{MultipleObjectsReturned}}
Bases: \code{django.core.exceptions.MultipleObjectsReturned}

\end{fulllineitems}

\index{objects (mousedb.data.models.Implantation attribute)}

\begin{fulllineitems}
\phantomsection\label{api:mousedb.data.models.Implantation.objects}\pysigline{\code{Implantation.}\bfcode{objects}\strong{ = \textless{}django.db.models.manager.Manager object at 0x0231BE70\textgreater{}}}
\end{fulllineitems}

\index{surgeon (mousedb.data.models.Implantation attribute)}

\begin{fulllineitems}
\phantomsection\label{api:mousedb.data.models.Implantation.surgeon}\pysigline{\code{Implantation.}\bfcode{surgeon}}
\end{fulllineitems}

\index{treatment\_set (mousedb.data.models.Implantation attribute)}

\begin{fulllineitems}
\phantomsection\label{api:mousedb.data.models.Implantation.treatment_set}\pysigline{\code{Implantation.}\bfcode{treatment\_set}}
\end{fulllineitems}

\index{vendor (mousedb.data.models.Implantation attribute)}

\begin{fulllineitems}
\phantomsection\label{api:mousedb.data.models.Implantation.vendor}\pysigline{\code{Implantation.}\bfcode{vendor}}
\end{fulllineitems}


\end{fulllineitems}

\index{Measurement (class in mousedb.data.models)}

\begin{fulllineitems}
\phantomsection\label{api:mousedb.data.models.Measurement}\pysiglinewithargsret{\strong{class }\code{mousedb.data.models.}\bfcode{Measurement}}{\emph{*args}, \emph{**kwargs}}{}
Bases: \code{django.db.models.base.Model}

Measurement(id, animal\_id, experiment\_id, assay\_id, values)
\index{Measurement.DoesNotExist}

\begin{fulllineitems}
\phantomsection\label{api:mousedb.data.models.Measurement.DoesNotExist}\pysigline{\strong{exception }\bfcode{DoesNotExist}}
Bases: \code{django.core.exceptions.ObjectDoesNotExist}

\end{fulllineitems}

\index{Measurement.MultipleObjectsReturned}

\begin{fulllineitems}
\phantomsection\label{api:mousedb.data.models.Measurement.MultipleObjectsReturned}\pysigline{\strong{exception }\code{Measurement.}\bfcode{MultipleObjectsReturned}}
Bases: \code{django.core.exceptions.MultipleObjectsReturned}

\end{fulllineitems}

\index{age() (mousedb.data.models.Measurement method)}

\begin{fulllineitems}
\phantomsection\label{api:mousedb.data.models.Measurement.age}\pysiglinewithargsret{\code{Measurement.}\bfcode{age}}{}{}
\end{fulllineitems}

\index{animal (mousedb.data.models.Measurement attribute)}

\begin{fulllineitems}
\phantomsection\label{api:mousedb.data.models.Measurement.animal}\pysigline{\code{Measurement.}\bfcode{animal}}
\end{fulllineitems}

\index{assay (mousedb.data.models.Measurement attribute)}

\begin{fulllineitems}
\phantomsection\label{api:mousedb.data.models.Measurement.assay}\pysigline{\code{Measurement.}\bfcode{assay}}
\end{fulllineitems}

\index{experiment (mousedb.data.models.Measurement attribute)}

\begin{fulllineitems}
\phantomsection\label{api:mousedb.data.models.Measurement.experiment}\pysigline{\code{Measurement.}\bfcode{experiment}}
\end{fulllineitems}

\index{get\_absolute\_url() (mousedb.data.models.Measurement method)}

\begin{fulllineitems}
\phantomsection\label{api:mousedb.data.models.Measurement.get_absolute_url}\pysiglinewithargsret{\code{Measurement.}\bfcode{get\_absolute\_url}}{\emph{*moreargs}, \emph{**morekwargs}}{}
\end{fulllineitems}

\index{objects (mousedb.data.models.Measurement attribute)}

\begin{fulllineitems}
\phantomsection\label{api:mousedb.data.models.Measurement.objects}\pysigline{\code{Measurement.}\bfcode{objects}\strong{ = \textless{}django.db.models.manager.Manager object at 0x022FEA50\textgreater{}}}
\end{fulllineitems}


\end{fulllineitems}

\index{Pharmaceutical (class in mousedb.data.models)}

\begin{fulllineitems}
\phantomsection\label{api:mousedb.data.models.Pharmaceutical}\pysiglinewithargsret{\strong{class }\code{mousedb.data.models.}\bfcode{Pharmaceutical}}{\emph{*args}, \emph{**kwargs}}{}
Bases: \code{django.db.models.base.Model}

Pharmaceutical(id, drug, dose, recurrance, mode, vendor\_id, notes)
\index{Pharmaceutical.DoesNotExist}

\begin{fulllineitems}
\phantomsection\label{api:mousedb.data.models.Pharmaceutical.DoesNotExist}\pysigline{\strong{exception }\bfcode{DoesNotExist}}
Bases: \code{django.core.exceptions.ObjectDoesNotExist}

\end{fulllineitems}

\index{Pharmaceutical.MultipleObjectsReturned}

\begin{fulllineitems}
\phantomsection\label{api:mousedb.data.models.Pharmaceutical.MultipleObjectsReturned}\pysigline{\strong{exception }\code{Pharmaceutical.}\bfcode{MultipleObjectsReturned}}
Bases: \code{django.core.exceptions.MultipleObjectsReturned}

\end{fulllineitems}

\index{get\_mode\_display() (mousedb.data.models.Pharmaceutical method)}

\begin{fulllineitems}
\phantomsection\label{api:mousedb.data.models.Pharmaceutical.get_mode_display}\pysiglinewithargsret{\code{Pharmaceutical.}\bfcode{get\_mode\_display}}{\emph{*moreargs}, \emph{**morekwargs}}{}
\end{fulllineitems}

\index{objects (mousedb.data.models.Pharmaceutical attribute)}

\begin{fulllineitems}
\phantomsection\label{api:mousedb.data.models.Pharmaceutical.objects}\pysigline{\code{Pharmaceutical.}\bfcode{objects}\strong{ = \textless{}django.db.models.manager.Manager object at 0x02324550\textgreater{}}}
\end{fulllineitems}

\index{treatment\_set (mousedb.data.models.Pharmaceutical attribute)}

\begin{fulllineitems}
\phantomsection\label{api:mousedb.data.models.Pharmaceutical.treatment_set}\pysigline{\code{Pharmaceutical.}\bfcode{treatment\_set}}
\end{fulllineitems}

\index{vendor (mousedb.data.models.Pharmaceutical attribute)}

\begin{fulllineitems}
\phantomsection\label{api:mousedb.data.models.Pharmaceutical.vendor}\pysigline{\code{Pharmaceutical.}\bfcode{vendor}}
\end{fulllineitems}


\end{fulllineitems}

\index{Researcher (class in mousedb.data.models)}

\begin{fulllineitems}
\phantomsection\label{api:mousedb.data.models.Researcher}\pysiglinewithargsret{\strong{class }\code{mousedb.data.models.}\bfcode{Researcher}}{\emph{*args}, \emph{**kwargs}}{}
Bases: \code{django.db.models.base.Model}

Researcher(id, first\_name, last\_name, name\_slug, email, active)
\index{Researcher.DoesNotExist}

\begin{fulllineitems}
\phantomsection\label{api:mousedb.data.models.Researcher.DoesNotExist}\pysigline{\strong{exception }\bfcode{DoesNotExist}}
Bases: \code{django.core.exceptions.ObjectDoesNotExist}

\end{fulllineitems}

\index{Researcher.MultipleObjectsReturned}

\begin{fulllineitems}
\phantomsection\label{api:mousedb.data.models.Researcher.MultipleObjectsReturned}\pysigline{\strong{exception }\code{Researcher.}\bfcode{MultipleObjectsReturned}}
Bases: \code{django.core.exceptions.MultipleObjectsReturned}

\end{fulllineitems}

\index{environment\_set (mousedb.data.models.Researcher attribute)}

\begin{fulllineitems}
\phantomsection\label{api:mousedb.data.models.Researcher.environment_set}\pysigline{\code{Researcher.}\bfcode{environment\_set}}
\end{fulllineitems}

\index{experiment\_set (mousedb.data.models.Researcher attribute)}

\begin{fulllineitems}
\phantomsection\label{api:mousedb.data.models.Researcher.experiment_set}\pysigline{\code{Researcher.}\bfcode{experiment\_set}}
\end{fulllineitems}

\index{implantation\_set (mousedb.data.models.Researcher attribute)}

\begin{fulllineitems}
\phantomsection\label{api:mousedb.data.models.Researcher.implantation_set}\pysigline{\code{Researcher.}\bfcode{implantation\_set}}
\end{fulllineitems}

\index{objects (mousedb.data.models.Researcher attribute)}

\begin{fulllineitems}
\phantomsection\label{api:mousedb.data.models.Researcher.objects}\pysigline{\code{Researcher.}\bfcode{objects}\strong{ = \textless{}django.db.models.manager.Manager object at 0x022FED50\textgreater{}}}
\end{fulllineitems}

\index{transplantation\_set (mousedb.data.models.Researcher attribute)}

\begin{fulllineitems}
\phantomsection\label{api:mousedb.data.models.Researcher.transplantation_set}\pysigline{\code{Researcher.}\bfcode{transplantation\_set}}
\end{fulllineitems}

\index{treatment\_set (mousedb.data.models.Researcher attribute)}

\begin{fulllineitems}
\phantomsection\label{api:mousedb.data.models.Researcher.treatment_set}\pysigline{\code{Researcher.}\bfcode{treatment\_set}}
\end{fulllineitems}


\end{fulllineitems}

\index{Study (class in mousedb.data.models)}

\begin{fulllineitems}
\phantomsection\label{api:mousedb.data.models.Study}\pysiglinewithargsret{\strong{class }\code{mousedb.data.models.}\bfcode{Study}}{\emph{*args}, \emph{**kwargs}}{}
Bases: \code{django.db.models.base.Model}

Study(id, description, start\_date, stop\_date, notes)
\index{Study.DoesNotExist}

\begin{fulllineitems}
\phantomsection\label{api:mousedb.data.models.Study.DoesNotExist}\pysigline{\strong{exception }\bfcode{DoesNotExist}}
Bases: \code{django.core.exceptions.ObjectDoesNotExist}

\end{fulllineitems}

\index{Study.MultipleObjectsReturned}

\begin{fulllineitems}
\phantomsection\label{api:mousedb.data.models.Study.MultipleObjectsReturned}\pysigline{\strong{exception }\code{Study.}\bfcode{MultipleObjectsReturned}}
Bases: \code{django.core.exceptions.MultipleObjectsReturned}

\end{fulllineitems}

\index{experiment\_set (mousedb.data.models.Study attribute)}

\begin{fulllineitems}
\phantomsection\label{api:mousedb.data.models.Study.experiment_set}\pysigline{\code{Study.}\bfcode{experiment\_set}}
\end{fulllineitems}

\index{get\_absolute\_url() (mousedb.data.models.Study method)}

\begin{fulllineitems}
\phantomsection\label{api:mousedb.data.models.Study.get_absolute_url}\pysiglinewithargsret{\code{Study.}\bfcode{get\_absolute\_url}}{\emph{*moreargs}, \emph{**morekwargs}}{}
\end{fulllineitems}

\index{objects (mousedb.data.models.Study attribute)}

\begin{fulllineitems}
\phantomsection\label{api:mousedb.data.models.Study.objects}\pysigline{\code{Study.}\bfcode{objects}\strong{ = \textless{}django.db.models.manager.Manager object at 0x02304AD0\textgreater{}}}
\end{fulllineitems}

\index{strain (mousedb.data.models.Study attribute)}

\begin{fulllineitems}
\phantomsection\label{api:mousedb.data.models.Study.strain}\pysigline{\code{Study.}\bfcode{strain}}
\end{fulllineitems}

\index{treatment\_set (mousedb.data.models.Study attribute)}

\begin{fulllineitems}
\phantomsection\label{api:mousedb.data.models.Study.treatment_set}\pysigline{\code{Study.}\bfcode{treatment\_set}}
\end{fulllineitems}


\end{fulllineitems}

\index{Transplantation (class in mousedb.data.models)}

\begin{fulllineitems}
\phantomsection\label{api:mousedb.data.models.Transplantation}\pysiglinewithargsret{\strong{class }\code{mousedb.data.models.}\bfcode{Transplantation}}{\emph{*args}, \emph{**kwargs}}{}
Bases: \code{django.db.models.base.Model}

Transplantation(id, tissue, transplant\_date, notes)
\index{Transplantation.DoesNotExist}

\begin{fulllineitems}
\phantomsection\label{api:mousedb.data.models.Transplantation.DoesNotExist}\pysigline{\strong{exception }\bfcode{DoesNotExist}}
Bases: \code{django.core.exceptions.ObjectDoesNotExist}

\end{fulllineitems}

\index{Transplantation.MultipleObjectsReturned}

\begin{fulllineitems}
\phantomsection\label{api:mousedb.data.models.Transplantation.MultipleObjectsReturned}\pysigline{\strong{exception }\code{Transplantation.}\bfcode{MultipleObjectsReturned}}
Bases: \code{django.core.exceptions.MultipleObjectsReturned}

\end{fulllineitems}

\index{donor (mousedb.data.models.Transplantation attribute)}

\begin{fulllineitems}
\phantomsection\label{api:mousedb.data.models.Transplantation.donor}\pysigline{\code{Transplantation.}\bfcode{donor}}
\end{fulllineitems}

\index{get\_next\_by\_transplant\_date() (mousedb.data.models.Transplantation method)}

\begin{fulllineitems}
\phantomsection\label{api:mousedb.data.models.Transplantation.get_next_by_transplant_date}\pysiglinewithargsret{\code{Transplantation.}\bfcode{get\_next\_by\_transplant\_date}}{\emph{*moreargs}, \emph{**morekwargs}}{}
\end{fulllineitems}

\index{get\_previous\_by\_transplant\_date() (mousedb.data.models.Transplantation method)}

\begin{fulllineitems}
\phantomsection\label{api:mousedb.data.models.Transplantation.get_previous_by_transplant_date}\pysiglinewithargsret{\code{Transplantation.}\bfcode{get\_previous\_by\_transplant\_date}}{\emph{*moreargs}, \emph{**morekwargs}}{}
\end{fulllineitems}

\index{objects (mousedb.data.models.Transplantation attribute)}

\begin{fulllineitems}
\phantomsection\label{api:mousedb.data.models.Transplantation.objects}\pysigline{\code{Transplantation.}\bfcode{objects}\strong{ = \textless{}django.db.models.manager.Manager object at 0x0232A7B0\textgreater{}}}
\end{fulllineitems}

\index{surgeon (mousedb.data.models.Transplantation attribute)}

\begin{fulllineitems}
\phantomsection\label{api:mousedb.data.models.Transplantation.surgeon}\pysigline{\code{Transplantation.}\bfcode{surgeon}}
\end{fulllineitems}

\index{treatment\_set (mousedb.data.models.Transplantation attribute)}

\begin{fulllineitems}
\phantomsection\label{api:mousedb.data.models.Transplantation.treatment_set}\pysigline{\code{Transplantation.}\bfcode{treatment\_set}}
\end{fulllineitems}


\end{fulllineitems}

\index{Treatment (class in mousedb.data.models)}

\begin{fulllineitems}
\phantomsection\label{api:mousedb.data.models.Treatment}\pysiglinewithargsret{\strong{class }\code{mousedb.data.models.}\bfcode{Treatment}}{\emph{*args}, \emph{**kwargs}}{}
Bases: \code{django.db.models.base.Model}

Treatment(id, treatment, study\_id, diet\_id, environment\_id, transplantation\_id, notes)
\index{Treatment.DoesNotExist}

\begin{fulllineitems}
\phantomsection\label{api:mousedb.data.models.Treatment.DoesNotExist}\pysigline{\strong{exception }\bfcode{DoesNotExist}}
Bases: \code{django.core.exceptions.ObjectDoesNotExist}

\end{fulllineitems}

\index{Treatment.MultipleObjectsReturned}

\begin{fulllineitems}
\phantomsection\label{api:mousedb.data.models.Treatment.MultipleObjectsReturned}\pysigline{\strong{exception }\code{Treatment.}\bfcode{MultipleObjectsReturned}}
Bases: \code{django.core.exceptions.MultipleObjectsReturned}

\end{fulllineitems}

\index{animals (mousedb.data.models.Treatment attribute)}

\begin{fulllineitems}
\phantomsection\label{api:mousedb.data.models.Treatment.animals}\pysigline{\code{Treatment.}\bfcode{animals}}
\end{fulllineitems}

\index{diet (mousedb.data.models.Treatment attribute)}

\begin{fulllineitems}
\phantomsection\label{api:mousedb.data.models.Treatment.diet}\pysigline{\code{Treatment.}\bfcode{diet}}
\end{fulllineitems}

\index{environment (mousedb.data.models.Treatment attribute)}

\begin{fulllineitems}
\phantomsection\label{api:mousedb.data.models.Treatment.environment}\pysigline{\code{Treatment.}\bfcode{environment}}
\end{fulllineitems}

\index{get\_absolute\_url() (mousedb.data.models.Treatment method)}

\begin{fulllineitems}
\phantomsection\label{api:mousedb.data.models.Treatment.get_absolute_url}\pysiglinewithargsret{\code{Treatment.}\bfcode{get\_absolute\_url}}{\emph{*moreargs}, \emph{**morekwargs}}{}
\end{fulllineitems}

\index{implantation (mousedb.data.models.Treatment attribute)}

\begin{fulllineitems}
\phantomsection\label{api:mousedb.data.models.Treatment.implantation}\pysigline{\code{Treatment.}\bfcode{implantation}}
\end{fulllineitems}

\index{objects (mousedb.data.models.Treatment attribute)}

\begin{fulllineitems}
\phantomsection\label{api:mousedb.data.models.Treatment.objects}\pysigline{\code{Treatment.}\bfcode{objects}\strong{ = \textless{}django.db.models.manager.Manager object at 0x023123F0\textgreater{}}}
\end{fulllineitems}

\index{pharmaceutical (mousedb.data.models.Treatment attribute)}

\begin{fulllineitems}
\phantomsection\label{api:mousedb.data.models.Treatment.pharmaceutical}\pysigline{\code{Treatment.}\bfcode{pharmaceutical}}
\end{fulllineitems}

\index{researchers (mousedb.data.models.Treatment attribute)}

\begin{fulllineitems}
\phantomsection\label{api:mousedb.data.models.Treatment.researchers}\pysigline{\code{Treatment.}\bfcode{researchers}}
\end{fulllineitems}

\index{study (mousedb.data.models.Treatment attribute)}

\begin{fulllineitems}
\phantomsection\label{api:mousedb.data.models.Treatment.study}\pysigline{\code{Treatment.}\bfcode{study}}
\end{fulllineitems}

\index{transplantation (mousedb.data.models.Treatment attribute)}

\begin{fulllineitems}
\phantomsection\label{api:mousedb.data.models.Treatment.transplantation}\pysigline{\code{Treatment.}\bfcode{transplantation}}
\end{fulllineitems}


\end{fulllineitems}

\index{Vendor (class in mousedb.data.models)}

\begin{fulllineitems}
\phantomsection\label{api:mousedb.data.models.Vendor}\pysiglinewithargsret{\strong{class }\code{mousedb.data.models.}\bfcode{Vendor}}{\emph{*args}, \emph{**kwargs}}{}
Bases: \code{django.db.models.base.Model}

Vendor(id, vendor, website, email, ordering, notes)
\index{Vendor.DoesNotExist}

\begin{fulllineitems}
\phantomsection\label{api:mousedb.data.models.Vendor.DoesNotExist}\pysigline{\strong{exception }\bfcode{DoesNotExist}}
Bases: \code{django.core.exceptions.ObjectDoesNotExist}

\end{fulllineitems}

\index{Vendor.MultipleObjectsReturned}

\begin{fulllineitems}
\phantomsection\label{api:mousedb.data.models.Vendor.MultipleObjectsReturned}\pysigline{\strong{exception }\code{Vendor.}\bfcode{MultipleObjectsReturned}}
Bases: \code{django.core.exceptions.MultipleObjectsReturned}

\end{fulllineitems}

\index{diet\_set (mousedb.data.models.Vendor attribute)}

\begin{fulllineitems}
\phantomsection\label{api:mousedb.data.models.Vendor.diet_set}\pysigline{\code{Vendor.}\bfcode{diet\_set}}
\end{fulllineitems}

\index{get\_ordering\_display() (mousedb.data.models.Vendor method)}

\begin{fulllineitems}
\phantomsection\label{api:mousedb.data.models.Vendor.get_ordering_display}\pysiglinewithargsret{\code{Vendor.}\bfcode{get\_ordering\_display}}{\emph{*moreargs}, \emph{**morekwargs}}{}
\end{fulllineitems}

\index{implantation\_set (mousedb.data.models.Vendor attribute)}

\begin{fulllineitems}
\phantomsection\label{api:mousedb.data.models.Vendor.implantation_set}\pysigline{\code{Vendor.}\bfcode{implantation\_set}}
\end{fulllineitems}

\index{objects (mousedb.data.models.Vendor attribute)}

\begin{fulllineitems}
\phantomsection\label{api:mousedb.data.models.Vendor.objects}\pysigline{\code{Vendor.}\bfcode{objects}\strong{ = \textless{}django.db.models.manager.Manager object at 0x023127B0\textgreater{}}}
\end{fulllineitems}

\index{pharmaceutical\_set (mousedb.data.models.Vendor attribute)}

\begin{fulllineitems}
\phantomsection\label{api:mousedb.data.models.Vendor.pharmaceutical_set}\pysigline{\code{Vendor.}\bfcode{pharmaceutical\_set}}
\end{fulllineitems}


\end{fulllineitems}



\subsection{Forms}
\label{api:module-mousedb.data.forms}\label{api:forms}\index{mousedb.data.forms (module)}\index{ExperimentForm (class in mousedb.data.forms)}

\begin{fulllineitems}
\phantomsection\label{api:mousedb.data.forms.ExperimentForm}\pysiglinewithargsret{\strong{class }\code{mousedb.data.forms.}\bfcode{ExperimentForm}}{\emph{data=None}, \emph{files=None}, \emph{auto\_id='id\_\%s'}, \emph{prefix=None}, \emph{initial=None}, \emph{error\_class=\textless{}class `django.forms.util.ErrorList'\textgreater{}}, \emph{label\_suffix=':'}, \emph{empty\_permitted=False}, \emph{instance=None}}{}
Bases: \code{django.forms.models.ModelForm}

This is the configuration for the experiment form.

This form is used to set up and modify an experiment.  It uses a datepicker widget for the date.
\index{ExperimentForm.Meta (class in mousedb.data.forms)}

\begin{fulllineitems}
\phantomsection\label{api:mousedb.data.forms.ExperimentForm.Meta}\pysigline{\strong{class }\bfcode{Meta}}~\index{model (mousedb.data.forms.ExperimentForm.Meta attribute)}

\begin{fulllineitems}
\phantomsection\label{api:mousedb.data.forms.ExperimentForm.Meta.model}\pysigline{\bfcode{model}}
alias of \code{Experiment}

\end{fulllineitems}


\end{fulllineitems}

\index{base\_fields (mousedb.data.forms.ExperimentForm attribute)}

\begin{fulllineitems}
\phantomsection\label{api:mousedb.data.forms.ExperimentForm.base_fields}\pysigline{\code{ExperimentForm.}\bfcode{base\_fields}\strong{ = \{`date': \textless{}django.forms.fields.DateField object at 0x023F4090\textgreater{}, `notes': \textless{}django.forms.fields.CharField object at 0x02623970\textgreater{}, `experimentID': \textless{}django.forms.fields.SlugField object at 0x023F4130\textgreater{}, `feeding\_state': \textless{}django.forms.fields.TypedChoiceField object at 0x02623AB0\textgreater{}, `fasting\_time': \textless{}django.forms.fields.IntegerField object at 0x023C7590\textgreater{}, `injection': \textless{}django.forms.fields.TypedChoiceField object at 0x023C7830\textgreater{}, `concentration': \textless{}django.forms.fields.CharField object at 0x023C7610\textgreater{}, `study': \textless{}django.forms.models.ModelChoiceField object at 0x023C7A90\textgreater{}, `researchers': \textless{}django.forms.models.ModelMultipleChoiceField object at 0x023C7030\textgreater{}\}}}
\end{fulllineitems}

\index{declared\_fields (mousedb.data.forms.ExperimentForm attribute)}

\begin{fulllineitems}
\phantomsection\label{api:mousedb.data.forms.ExperimentForm.declared_fields}\pysigline{\code{ExperimentForm.}\bfcode{declared\_fields}\strong{ = \{\}}}
\end{fulllineitems}

\index{media (mousedb.data.forms.ExperimentForm attribute)}

\begin{fulllineitems}
\phantomsection\label{api:mousedb.data.forms.ExperimentForm.media}\pysigline{\code{ExperimentForm.}\bfcode{media}}
\end{fulllineitems}


\end{fulllineitems}

\index{MeasurementForm (class in mousedb.data.forms)}

\begin{fulllineitems}
\phantomsection\label{api:mousedb.data.forms.MeasurementForm}\pysiglinewithargsret{\strong{class }\code{mousedb.data.forms.}\bfcode{MeasurementForm}}{\emph{data=None}, \emph{files=None}, \emph{auto\_id='id\_\%s'}, \emph{prefix=None}, \emph{initial=None}, \emph{error\_class=\textless{}class `django.forms.util.ErrorList'\textgreater{}}, \emph{label\_suffix=':'}, \emph{empty\_permitted=False}, \emph{instance=None}}{}
Bases: \code{django.forms.models.ModelForm}

Form definition for adding and editing measurements.

This form is used for adding or modifying single measurements from within an experiment.  It has an autocomplete field for animal.
\index{MeasurementForm.Media (class in mousedb.data.forms)}

\begin{fulllineitems}
\phantomsection\label{api:mousedb.data.forms.MeasurementForm.Media}\pysigline{\strong{class }\bfcode{Media}}~\index{css (mousedb.data.forms.MeasurementForm.Media attribute)}

\begin{fulllineitems}
\phantomsection\label{api:mousedb.data.forms.MeasurementForm.Media.css}\pysigline{\bfcode{css}\strong{ = \{`all': (`javascript/jquery-autocomplete/jquery.autocomplete.css', `css/autocomplete.css')\}}}
\end{fulllineitems}

\index{js (mousedb.data.forms.MeasurementForm.Media attribute)}

\begin{fulllineitems}
\phantomsection\label{api:mousedb.data.forms.MeasurementForm.Media.js}\pysigline{\bfcode{js}\strong{ = (`javascript/jquery-ui/js/jquery-ui-1.8.2.custom.min.js', `javascript/jquery-autocomplete/jquery.autocomplete.js')}}
\end{fulllineitems}


\end{fulllineitems}

\index{MeasurementForm.Meta (class in mousedb.data.forms)}

\begin{fulllineitems}
\phantomsection\label{api:mousedb.data.forms.MeasurementForm.Meta}\pysigline{\strong{class }\code{MeasurementForm.}\bfcode{Meta}}~\index{model (mousedb.data.forms.MeasurementForm.Meta attribute)}

\begin{fulllineitems}
\phantomsection\label{api:mousedb.data.forms.MeasurementForm.Meta.model}\pysigline{\bfcode{model}}
alias of \code{Measurement}

\end{fulllineitems}


\end{fulllineitems}

\index{base\_fields (mousedb.data.forms.MeasurementForm attribute)}

\begin{fulllineitems}
\phantomsection\label{api:mousedb.data.forms.MeasurementForm.base_fields}\pysigline{\code{MeasurementForm.}\bfcode{base\_fields}\strong{ = \{`animal': \textless{}ajax\_select.fields.AutoCompleteSelectField object at 0x0261ED50\textgreater{}, `experiment': \textless{}django.forms.models.ModelChoiceField object at 0x02392250\textgreater{}, `assay': \textless{}django.forms.models.ModelChoiceField object at 0x02392930\textgreater{}, `values': \textless{}django.forms.fields.CharField object at 0x02392F30\textgreater{}\}}}
\end{fulllineitems}

\index{declared\_fields (mousedb.data.forms.MeasurementForm attribute)}

\begin{fulllineitems}
\phantomsection\label{api:mousedb.data.forms.MeasurementForm.declared_fields}\pysigline{\code{MeasurementForm.}\bfcode{declared\_fields}\strong{ = \{`animal': \textless{}ajax\_select.fields.AutoCompleteSelectField object at 0x0261ED50\textgreater{}\}}}
\end{fulllineitems}

\index{media (mousedb.data.forms.MeasurementForm attribute)}

\begin{fulllineitems}
\phantomsection\label{api:mousedb.data.forms.MeasurementForm.media}\pysigline{\code{MeasurementForm.}\bfcode{media}}
\end{fulllineitems}


\end{fulllineitems}

\index{StudyExperimentForm (class in mousedb.data.forms)}

\begin{fulllineitems}
\phantomsection\label{api:mousedb.data.forms.StudyExperimentForm}\pysiglinewithargsret{\strong{class }\code{mousedb.data.forms.}\bfcode{StudyExperimentForm}}{\emph{data=None}, \emph{files=None}, \emph{auto\_id='id\_\%s'}, \emph{prefix=None}, \emph{initial=None}, \emph{error\_class=\textless{}class `django.forms.util.ErrorList'\textgreater{}}, \emph{label\_suffix=':'}, \emph{empty\_permitted=False}, \emph{instance=None}}{}
Bases: \code{django.forms.models.ModelForm}

This is the configuration for a study form (From an experiment).

This hides the study field which will be automatically set upon save.
\index{StudyExperimentForm.Meta (class in mousedb.data.forms)}

\begin{fulllineitems}
\phantomsection\label{api:mousedb.data.forms.StudyExperimentForm.Meta}\pysigline{\strong{class }\bfcode{Meta}}~\index{exclude (mousedb.data.forms.StudyExperimentForm.Meta attribute)}

\begin{fulllineitems}
\phantomsection\label{api:mousedb.data.forms.StudyExperimentForm.Meta.exclude}\pysigline{\bfcode{exclude}\strong{ = {[}'study'{]}}}
\end{fulllineitems}

\index{model (mousedb.data.forms.StudyExperimentForm.Meta attribute)}

\begin{fulllineitems}
\phantomsection\label{api:mousedb.data.forms.StudyExperimentForm.Meta.model}\pysigline{\bfcode{model}}
alias of \code{Experiment}

\end{fulllineitems}


\end{fulllineitems}

\index{base\_fields (mousedb.data.forms.StudyExperimentForm attribute)}

\begin{fulllineitems}
\phantomsection\label{api:mousedb.data.forms.StudyExperimentForm.base_fields}\pysigline{\code{StudyExperimentForm.}\bfcode{base\_fields}\strong{ = \{`date': \textless{}django.forms.fields.DateField object at 0x023C74D0\textgreater{}, `notes': \textless{}django.forms.fields.CharField object at 0x023C7F10\textgreater{}, `experimentID': \textless{}django.forms.fields.SlugField object at 0x023C7170\textgreater{}, `feeding\_state': \textless{}django.forms.fields.TypedChoiceField object at 0x023C74B0\textgreater{}, `fasting\_time': \textless{}django.forms.fields.IntegerField object at 0x023C70D0\textgreater{}, `injection': \textless{}django.forms.fields.TypedChoiceField object at 0x023C7070\textgreater{}, `concentration': \textless{}django.forms.fields.CharField object at 0x023C7950\textgreater{}, `researchers': \textless{}django.forms.models.ModelMultipleChoiceField object at 0x023C7530\textgreater{}\}}}
\end{fulllineitems}

\index{declared\_fields (mousedb.data.forms.StudyExperimentForm attribute)}

\begin{fulllineitems}
\phantomsection\label{api:mousedb.data.forms.StudyExperimentForm.declared_fields}\pysigline{\code{StudyExperimentForm.}\bfcode{declared\_fields}\strong{ = \{\}}}
\end{fulllineitems}

\index{media (mousedb.data.forms.StudyExperimentForm attribute)}

\begin{fulllineitems}
\phantomsection\label{api:mousedb.data.forms.StudyExperimentForm.media}\pysigline{\code{StudyExperimentForm.}\bfcode{media}}
\end{fulllineitems}


\end{fulllineitems}

\index{StudyForm (class in mousedb.data.forms)}

\begin{fulllineitems}
\phantomsection\label{api:mousedb.data.forms.StudyForm}\pysiglinewithargsret{\strong{class }\code{mousedb.data.forms.}\bfcode{StudyForm}}{\emph{data=None}, \emph{files=None}, \emph{auto\_id='id\_\%s'}, \emph{prefix=None}, \emph{initial=None}, \emph{error\_class=\textless{}class `django.forms.util.ErrorList'\textgreater{}}, \emph{label\_suffix=':'}, \emph{empty\_permitted=False}, \emph{instance=None}}{}
Bases: \code{django.forms.models.ModelForm}

This is the configuration for the study form.

This form is used to create and modify studies.  It uses an autocomplete widget for the animals.
\index{StudyForm.Meta (class in mousedb.data.forms)}

\begin{fulllineitems}
\phantomsection\label{api:mousedb.data.forms.StudyForm.Meta}\pysigline{\strong{class }\bfcode{Meta}}~\index{model (mousedb.data.forms.StudyForm.Meta attribute)}

\begin{fulllineitems}
\phantomsection\label{api:mousedb.data.forms.StudyForm.Meta.model}\pysigline{\bfcode{model}}
alias of \code{Study}

\end{fulllineitems}


\end{fulllineitems}

\index{base\_fields (mousedb.data.forms.StudyForm attribute)}

\begin{fulllineitems}
\phantomsection\label{api:mousedb.data.forms.StudyForm.base_fields}\pysigline{\code{StudyForm.}\bfcode{base\_fields}\strong{ = \{`description': \textless{}django.forms.fields.CharField object at 0x02392970\textgreater{}, `start\_date': \textless{}django.forms.fields.DateField object at 0x021E4250\textgreater{}, `stop\_date': \textless{}django.forms.fields.DateField object at 0x02392DD0\textgreater{}, `notes': \textless{}django.forms.fields.CharField object at 0x02392FB0\textgreater{}, `strain': \textless{}django.forms.models.ModelMultipleChoiceField object at 0x02237510\textgreater{}\}}}
\end{fulllineitems}

\index{declared\_fields (mousedb.data.forms.StudyForm attribute)}

\begin{fulllineitems}
\phantomsection\label{api:mousedb.data.forms.StudyForm.declared_fields}\pysigline{\code{StudyForm.}\bfcode{declared\_fields}\strong{ = \{\}}}
\end{fulllineitems}

\index{media (mousedb.data.forms.StudyForm attribute)}

\begin{fulllineitems}
\phantomsection\label{api:mousedb.data.forms.StudyForm.media}\pysigline{\code{StudyForm.}\bfcode{media}}
\end{fulllineitems}


\end{fulllineitems}

\index{TreatmentForm (class in mousedb.data.forms)}

\begin{fulllineitems}
\phantomsection\label{api:mousedb.data.forms.TreatmentForm}\pysiglinewithargsret{\strong{class }\code{mousedb.data.forms.}\bfcode{TreatmentForm}}{\emph{data=None}, \emph{files=None}, \emph{auto\_id='id\_\%s'}, \emph{prefix=None}, \emph{initial=None}, \emph{error\_class=\textless{}class `django.forms.util.ErrorList'\textgreater{}}, \emph{label\_suffix=':'}, \emph{empty\_permitted=False}, \emph{instance=None}}{}
Bases: \code{django.forms.models.ModelForm}

Form class for study treatment groups.

In the case of studies, animals are defined in the treatment group rather than in the study group.  A treatment consists of a study, a set of animals and the conditions which define that treatment.  This includes related fields for environment, diet, implants and transplants.
\index{TreatmentForm.Media (class in mousedb.data.forms)}

\begin{fulllineitems}
\phantomsection\label{api:mousedb.data.forms.TreatmentForm.Media}\pysigline{\strong{class }\bfcode{Media}}~\index{css (mousedb.data.forms.TreatmentForm.Media attribute)}

\begin{fulllineitems}
\phantomsection\label{api:mousedb.data.forms.TreatmentForm.Media.css}\pysigline{\bfcode{css}\strong{ = \{`all': (`javascript/jquery-autocomplete/jquery.autocomplete.css', `css/autocomplete.css')\}}}
\end{fulllineitems}

\index{js (mousedb.data.forms.TreatmentForm.Media attribute)}

\begin{fulllineitems}
\phantomsection\label{api:mousedb.data.forms.TreatmentForm.Media.js}\pysigline{\bfcode{js}\strong{ = (`javascript/jquery-ui/js/jquery-ui-1.8.2.custom.min.js', `javascript/jquery-autocomplete/jquery.autocomplete.js')}}
\end{fulllineitems}


\end{fulllineitems}

\index{TreatmentForm.Meta (class in mousedb.data.forms)}

\begin{fulllineitems}
\phantomsection\label{api:mousedb.data.forms.TreatmentForm.Meta}\pysigline{\strong{class }\code{TreatmentForm.}\bfcode{Meta}}~\index{model (mousedb.data.forms.TreatmentForm.Meta attribute)}

\begin{fulllineitems}
\phantomsection\label{api:mousedb.data.forms.TreatmentForm.Meta.model}\pysigline{\bfcode{model}}
alias of \code{Treatment}

\end{fulllineitems}


\end{fulllineitems}

\index{base\_fields (mousedb.data.forms.TreatmentForm attribute)}

\begin{fulllineitems}
\phantomsection\label{api:mousedb.data.forms.TreatmentForm.base_fields}\pysigline{\code{TreatmentForm.}\bfcode{base\_fields}\strong{ = \{`treatment': \textless{}django.forms.fields.CharField object at 0x02230F30\textgreater{}, `study': \textless{}django.forms.models.ModelChoiceField object at 0x018C9690\textgreater{}, `diet': \textless{}django.forms.models.ModelChoiceField object at 0x021805F0\textgreater{}, `environment': \textless{}django.forms.models.ModelChoiceField object at 0x021806D0\textgreater{}, `transplantation': \textless{}django.forms.models.ModelChoiceField object at 0x021806B0\textgreater{}, `notes': \textless{}django.forms.fields.CharField object at 0x01EFA310\textgreater{}, `animals': \textless{}ajax\_select.fields.AutoCompleteSelectMultipleField object at 0x02392CB0\textgreater{}, `implantation': \textless{}django.forms.models.ModelMultipleChoiceField object at 0x0269C2D0\textgreater{}, `pharmaceutical': \textless{}django.forms.models.ModelMultipleChoiceField object at 0x0269C3D0\textgreater{}, `researchers': \textless{}django.forms.models.ModelMultipleChoiceField object at 0x0269C530\textgreater{}\}}}
\end{fulllineitems}

\index{declared\_fields (mousedb.data.forms.TreatmentForm attribute)}

\begin{fulllineitems}
\phantomsection\label{api:mousedb.data.forms.TreatmentForm.declared_fields}\pysigline{\code{TreatmentForm.}\bfcode{declared\_fields}\strong{ = \{`animals': \textless{}ajax\_select.fields.AutoCompleteSelectMultipleField object at 0x02392CB0\textgreater{}\}}}
\end{fulllineitems}

\index{media (mousedb.data.forms.TreatmentForm attribute)}

\begin{fulllineitems}
\phantomsection\label{api:mousedb.data.forms.TreatmentForm.media}\pysigline{\code{TreatmentForm.}\bfcode{media}}
\end{fulllineitems}


\end{fulllineitems}



\subsection{Views and URLs}
\label{api:id1}\phantomsection\label{api:module-mousedb.data.views}\index{mousedb.data.views (module)}\index{add\_measurement() (in module mousedb.data.views)}

\begin{fulllineitems}
\phantomsection\label{api:mousedb.data.views.add_measurement}\pysiglinewithargsret{\code{mousedb.data.views.}\bfcode{add\_measurement}}{\emph{request}, \emph{*args}, \emph{**kwargs}}{}
This is a view to display a form to add single measurements to an experiment.

It calls the object MeasurementForm, which has an autocomplete field for animal.

\end{fulllineitems}

\index{aging\_csv() (in module mousedb.data.views)}

\begin{fulllineitems}
\phantomsection\label{api:mousedb.data.views.aging_csv}\pysiglinewithargsret{\code{mousedb.data.views.}\bfcode{aging\_csv}}{\emph{request}, \emph{*args}, \emph{**kwargs}}{}
This view generates a csv output file of all animal data for use in aging analysis.

The view writes to a csv table the animal, strain, genotype, age (in days), and cause of death.

\end{fulllineitems}

\index{experiment\_detail() (in module mousedb.data.views)}

\begin{fulllineitems}
\phantomsection\label{api:mousedb.data.views.experiment_detail}\pysiglinewithargsret{\code{mousedb.data.views.}\bfcode{experiment\_detail}}{\emph{request}, \emph{*args}, \emph{**kwargs}}{}
\end{fulllineitems}

\index{experiment\_detail\_all() (in module mousedb.data.views)}

\begin{fulllineitems}
\phantomsection\label{api:mousedb.data.views.experiment_detail_all}\pysiglinewithargsret{\code{mousedb.data.views.}\bfcode{experiment\_detail\_all}}{\emph{request}, \emph{*args}, \emph{**kwargs}}{}
\end{fulllineitems}

\index{experiment\_details\_csv() (in module mousedb.data.views)}

\begin{fulllineitems}
\phantomsection\label{api:mousedb.data.views.experiment_details_csv}\pysiglinewithargsret{\code{mousedb.data.views.}\bfcode{experiment\_details\_csv}}{\emph{request}, \emph{*args}, \emph{**kwargs}}{}
This view generates a csv output file of an experiment.

The view writes to a csv table the animal, genotype, age (in days), assay and values.

\end{fulllineitems}

\index{experiment\_list() (in module mousedb.data.views)}

\begin{fulllineitems}
\phantomsection\label{api:mousedb.data.views.experiment_list}\pysiglinewithargsret{\code{mousedb.data.views.}\bfcode{experiment\_list}}{\emph{request}, \emph{*args}, \emph{**kwargs}}{}
\end{fulllineitems}

\index{litters\_csv() (in module mousedb.data.views)}

\begin{fulllineitems}
\phantomsection\label{api:mousedb.data.views.litters_csv}\pysiglinewithargsret{\code{mousedb.data.views.}\bfcode{litters\_csv}}{\emph{request}, \emph{*args}, \emph{**kwargs}}{}
This view generates a csv output file of all animal data for use in litter analysis.

The view writes to a csv table the birthdate, breeding cage and strain.

\end{fulllineitems}

\index{study\_experiment() (in module mousedb.data.views)}

\begin{fulllineitems}
\phantomsection\label{api:mousedb.data.views.study_experiment}\pysiglinewithargsret{\code{mousedb.data.views.}\bfcode{study\_experiment}}{\emph{request}, \emph{*args}, \emph{**kwargs}}{}
\end{fulllineitems}

\phantomsection\label{api:module-mousedb.data.urls}\index{mousedb.data.urls (module)}

\subsection{Administrative Site Configuration}
\label{api:module-mousedb.data.admin}\label{api:administrative-site-configuration}\index{mousedb.data.admin (module)}

\subsection{Test Files}
\label{api:id2}\phantomsection\label{api:module-mousedb.data.tests}\index{mousedb.data.tests (module)}
This file contains tests for the data application.

These tests will verify generation of new experiment, measurement, assay, researcher, study, treatment, vendor, diet, environment, implantation, transplantation and pharnaceutical objects.
\index{StudyModelTests (class in mousedb.data.tests)}

\begin{fulllineitems}
\phantomsection\label{api:mousedb.data.tests.StudyModelTests}\pysiglinewithargsret{\strong{class }\code{mousedb.data.tests.}\bfcode{StudyModelTests}}{\emph{methodName='runTest'}}{}
Bases: \code{django.test.testcases.TestCase}

Test the creation and modification of Study objects.
\index{setUp() (mousedb.data.tests.StudyModelTests method)}

\begin{fulllineitems}
\phantomsection\label{api:mousedb.data.tests.StudyModelTests.setUp}\pysiglinewithargsret{\bfcode{setUp}}{}{}
Instantiate the test client.

\end{fulllineitems}

\index{tearDown() (mousedb.data.tests.StudyModelTests method)}

\begin{fulllineitems}
\phantomsection\label{api:mousedb.data.tests.StudyModelTests.tearDown}\pysiglinewithargsret{\bfcode{tearDown}}{}{}
Depopulate created model instances from test database.

\end{fulllineitems}

\index{test\_create\_studey\_detailed() (mousedb.data.tests.StudyModelTests method)}

\begin{fulllineitems}
\phantomsection\label{api:mousedb.data.tests.StudyModelTests.test_create_studey_detailed}\pysiglinewithargsret{\bfcode{test\_create\_studey\_detailed}}{}{}
This is a test for creating a new study object, with all fields being entered.  It also verifies that unicode is set correctly.  This test is dependent on the ability to create a new Strain object (see animal.tests.StrainModelTests.test\_create\_minimal\_strain).

\end{fulllineitems}

\index{test\_create\_study\_minimal() (mousedb.data.tests.StudyModelTests method)}

\begin{fulllineitems}
\phantomsection\label{api:mousedb.data.tests.StudyModelTests.test_create_study_minimal}\pysiglinewithargsret{\bfcode{test\_create\_study\_minimal}}{}{}
This is a test for creating a new study object, with only the minimum being entered.  It also verifies that unicode is set correctly.

\end{fulllineitems}

\index{test\_study\_absolute\_url() (mousedb.data.tests.StudyModelTests method)}

\begin{fulllineitems}
\phantomsection\label{api:mousedb.data.tests.StudyModelTests.test_study_absolute_url}\pysiglinewithargsret{\bfcode{test\_study\_absolute\_url}}{}{}
This test verifies that the absolute url of a study object is set correctly.  This study is dependend on a positive result on test\_create\_study\_minimal.

\end{fulllineitems}


\end{fulllineitems}

\index{StudyViewTests (class in mousedb.data.tests)}

\begin{fulllineitems}
\phantomsection\label{api:mousedb.data.tests.StudyViewTests}\pysiglinewithargsret{\strong{class }\code{mousedb.data.tests.}\bfcode{StudyViewTests}}{\emph{methodName='runTest'}}{}
Bases: \code{django.test.testcases.TestCase}

These tests test the views associated with Study objects.
\index{setUp() (mousedb.data.tests.StudyViewTests method)}

\begin{fulllineitems}
\phantomsection\label{api:mousedb.data.tests.StudyViewTests.setUp}\pysiglinewithargsret{\bfcode{setUp}}{}{}
This function sets up the test client, and creates a test study.

\end{fulllineitems}

\index{tearDown() (mousedb.data.tests.StudyViewTests method)}

\begin{fulllineitems}
\phantomsection\label{api:mousedb.data.tests.StudyViewTests.tearDown}\pysiglinewithargsret{\bfcode{tearDown}}{}{}
Depopulate created model instances from test database.

\end{fulllineitems}

\index{test\_study\_delete() (mousedb.data.tests.StudyViewTests method)}

\begin{fulllineitems}
\phantomsection\label{api:mousedb.data.tests.StudyViewTests.test_study_delete}\pysiglinewithargsret{\bfcode{test\_study\_delete}}{}{}
This test checks the view which displays a study detail page.  It checks for the correct templates and status code.

\end{fulllineitems}

\index{test\_study\_detail() (mousedb.data.tests.StudyViewTests method)}

\begin{fulllineitems}
\phantomsection\label{api:mousedb.data.tests.StudyViewTests.test_study_detail}\pysiglinewithargsret{\bfcode{test\_study\_detail}}{}{}
This test checks the view which displays a study detail page.  It checks for the correct templates and status code.

\end{fulllineitems}

\index{test\_study\_edit() (mousedb.data.tests.StudyViewTests method)}

\begin{fulllineitems}
\phantomsection\label{api:mousedb.data.tests.StudyViewTests.test_study_edit}\pysiglinewithargsret{\bfcode{test\_study\_edit}}{}{}
This test checks the view which displays a study edit page.  It checks for the correct templates and status code.

\end{fulllineitems}

\index{test\_study\_list() (mousedb.data.tests.StudyViewTests method)}

\begin{fulllineitems}
\phantomsection\label{api:mousedb.data.tests.StudyViewTests.test_study_list}\pysiglinewithargsret{\bfcode{test\_study\_list}}{}{}
This test checks the status code, and templates for study lists.

\end{fulllineitems}

\index{test\_study\_new() (mousedb.data.tests.StudyViewTests method)}

\begin{fulllineitems}
\phantomsection\label{api:mousedb.data.tests.StudyViewTests.test_study_new}\pysiglinewithargsret{\bfcode{test\_study\_new}}{}{}
This test checks the view which displays a study creation page.  It checks for the correct templates and status code.

\end{fulllineitems}


\end{fulllineitems}

\index{TreatmentViewTests (class in mousedb.data.tests)}

\begin{fulllineitems}
\phantomsection\label{api:mousedb.data.tests.TreatmentViewTests}\pysiglinewithargsret{\strong{class }\code{mousedb.data.tests.}\bfcode{TreatmentViewTests}}{\emph{methodName='runTest'}}{}
Bases: \code{django.test.testcases.TestCase}

These tests test the views associated with Treatment objects.
\index{fixtures (mousedb.data.tests.TreatmentViewTests attribute)}

\begin{fulllineitems}
\phantomsection\label{api:mousedb.data.tests.TreatmentViewTests.fixtures}\pysigline{\bfcode{fixtures}\strong{ = {[}'test\_treatment'{]}}}
\end{fulllineitems}

\index{setUp() (mousedb.data.tests.TreatmentViewTests method)}

\begin{fulllineitems}
\phantomsection\label{api:mousedb.data.tests.TreatmentViewTests.setUp}\pysiglinewithargsret{\bfcode{setUp}}{}{}
\end{fulllineitems}

\index{tearDown() (mousedb.data.tests.TreatmentViewTests method)}

\begin{fulllineitems}
\phantomsection\label{api:mousedb.data.tests.TreatmentViewTests.tearDown}\pysiglinewithargsret{\bfcode{tearDown}}{}{}
\end{fulllineitems}

\index{test\_treatment\_detail() (mousedb.data.tests.TreatmentViewTests method)}

\begin{fulllineitems}
\phantomsection\label{api:mousedb.data.tests.TreatmentViewTests.test_treatment_detail}\pysiglinewithargsret{\bfcode{test\_treatment\_detail}}{}{}
This test checks the view which displays a treatment-detail page.  It checks for the correct templates and status code.

\end{fulllineitems}


\end{fulllineitems}



\section{Animals Package}
\label{api:animals-package}\label{api:module-mousedb.animal}\index{mousedb.animal (module)}
The animal app contains and controls the display of data about animals.

Animals are tracked as individual entities, and given associations to breeding cages to follow ancestry, and strains.


\subsection{Animal}
\label{api:animal}
Most parameters about an animal are set within the animal object.  Here is where the animals strain, breeding, parentage and many other parameters are included.  Animals have foreignkey relationships with both Strain and Breeding, so an animal may only belong to one of each of those.  As an example, a mouse cannot come from more than one Breeding set, and cannot belong to more than one strain.


\subsubsection{Backcrosses and Generations}
\label{api:backcrosses-and-generations}
For this software, optional tracking of backcrosses and generations is available and is stored as an attribute of an animal.  When an inbred cross is made against a pure background, the backcross increases by 1.  When a heterozygote cross is made, the generation increases by one.  As an example, for every time a mouse in a C57/BL6 background is crossed against a wildtype C57/B6 mouse, the backcross (but not the generation) increases by one.  For every time a mutant strain is crosses against itself (either vs a heterozygote or homozygote of that strain), the generation will increase by one.  Backcrosses should typically be performed against a separate colony of purebred mouse, rather than against wild-type alleles of the mutant strain.


\subsection{Breeding Cages}
\label{api:breeding-cages}
A breeding cage is defined as a set of one or more male and one or more female mice.  Because of this, it is not always clear who the precise parentage of an animal is.  If the parentage is known, then the Mother and Father fields can be set for a particular animal.


\subsection{Strains}
\label{api:strains}
A strain is a set of mice with a similar genetics.  Importantly strains are separated from Backgrounds.  For example, one might have mice with the genotype ob/ob but these mice may be in either a C57-Black6 or a mixed background.  This difference is set at the individual animal level.  
The result of this is that a query for a particular strain may then need to be filtered to a specific background.


\subsection{Models}
\label{api:id3}\phantomsection\label{api:module-mousedb.animal.models}\index{mousedb.animal.models (module)}
This module describes the Strain, Animal, Breeding and Cage data models.

This module stores all data regarding a particular laboratory animal.  Information about experimental data and timed matings are stored in the data and timed\_matings packages.  This module describes the database structure for each data model.
\index{Animal (class in mousedb.animal.models)}

\begin{fulllineitems}
\phantomsection\label{api:mousedb.animal.models.Animal}\pysiglinewithargsret{\strong{class }\code{mousedb.animal.models.}\bfcode{Animal}}{\emph{*args}, \emph{**kwargs}}{}
Bases: \code{django.db.models.base.Model}

A data model describing an animal.

This data model describes a wide variety of parameters of an experimental animal.  This model is linked to the Strain.  If the parentage of a mouse is known, this can be identified (the breeding set may not be clear on this matter). Mice are automatically marked as not alive when a Death date is provided and the object is saved.  Strain, Background and Genotype are required fields.  By default, querysets are ordered first by strain then by MouseID.
\index{Breeding (mousedb.animal.models.Animal attribute)}

\begin{fulllineitems}
\phantomsection\label{api:mousedb.animal.models.Animal.Breeding}\pysigline{\bfcode{Breeding}}
\end{fulllineitems}

\index{Animal.DoesNotExist}

\begin{fulllineitems}
\phantomsection\label{api:mousedb.animal.models.Animal.DoesNotExist}\pysigline{\strong{exception }\bfcode{DoesNotExist}}
Bases: \code{django.core.exceptions.ObjectDoesNotExist}

\end{fulllineitems}

\index{Father (mousedb.animal.models.Animal attribute)}

\begin{fulllineitems}
\phantomsection\label{api:mousedb.animal.models.Animal.Father}\pysigline{\code{Animal.}\bfcode{Father}}
\end{fulllineitems}

\index{Mother (mousedb.animal.models.Animal attribute)}

\begin{fulllineitems}
\phantomsection\label{api:mousedb.animal.models.Animal.Mother}\pysigline{\code{Animal.}\bfcode{Mother}}
\end{fulllineitems}

\index{Animal.MultipleObjectsReturned}

\begin{fulllineitems}
\phantomsection\label{api:mousedb.animal.models.Animal.MultipleObjectsReturned}\pysigline{\strong{exception }\code{Animal.}\bfcode{MultipleObjectsReturned}}
Bases: \code{django.core.exceptions.MultipleObjectsReturned}

\end{fulllineitems}

\index{PlugFemale (mousedb.animal.models.Animal attribute)}

\begin{fulllineitems}
\phantomsection\label{api:mousedb.animal.models.Animal.PlugFemale}\pysigline{\code{Animal.}\bfcode{PlugFemale}}
\end{fulllineitems}

\index{PlugMale (mousedb.animal.models.Animal attribute)}

\begin{fulllineitems}
\phantomsection\label{api:mousedb.animal.models.Animal.PlugMale}\pysigline{\code{Animal.}\bfcode{PlugMale}}
\end{fulllineitems}

\index{Strain (mousedb.animal.models.Animal attribute)}

\begin{fulllineitems}
\phantomsection\label{api:mousedb.animal.models.Animal.Strain}\pysigline{\code{Animal.}\bfcode{Strain}}
\end{fulllineitems}

\index{age() (mousedb.animal.models.Animal method)}

\begin{fulllineitems}
\phantomsection\label{api:mousedb.animal.models.Animal.age}\pysiglinewithargsret{\code{Animal.}\bfcode{age}}{}{}
Calculates the animals age, relative to the current date (if alive) or the date of death (if not).

\end{fulllineitems}

\index{breeding\_female\_location\_type() (mousedb.animal.models.Animal method)}

\begin{fulllineitems}
\phantomsection\label{api:mousedb.animal.models.Animal.breeding_female_location_type}\pysiglinewithargsret{\code{Animal.}\bfcode{breeding\_female\_location\_type}}{}{}
This attribute defines whether a female's current location is the same as the breeding cage to which it belongs.

This attribute is used to color breeding table entries such that male mice which are currently in a different cage can quickly be identified.
The location is relative to the first breeding cage an animal is assigned to.

\end{fulllineitems}

\index{breeding\_females (mousedb.animal.models.Animal attribute)}

\begin{fulllineitems}
\phantomsection\label{api:mousedb.animal.models.Animal.breeding_females}\pysigline{\code{Animal.}\bfcode{breeding\_females}}
\end{fulllineitems}

\index{breeding\_male\_location\_type() (mousedb.animal.models.Animal method)}

\begin{fulllineitems}
\phantomsection\label{api:mousedb.animal.models.Animal.breeding_male_location_type}\pysiglinewithargsret{\code{Animal.}\bfcode{breeding\_male\_location\_type}}{}{}
This attribute defines whether a male's current location is the same as the breeding cage to which it belongs.

This attribute is used to color breeding table entries such that male mice which are currently in a different cage can quickly be identified.
The location is relative to the first breeding cage an animal is assigned to.

\end{fulllineitems}

\index{breeding\_males (mousedb.animal.models.Animal attribute)}

\begin{fulllineitems}
\phantomsection\label{api:mousedb.animal.models.Animal.breeding_males}\pysigline{\code{Animal.}\bfcode{breeding\_males}}
\end{fulllineitems}

\index{father (mousedb.animal.models.Animal attribute)}

\begin{fulllineitems}
\phantomsection\label{api:mousedb.animal.models.Animal.father}\pysigline{\code{Animal.}\bfcode{father}}
\end{fulllineitems}

\index{get\_Background\_display() (mousedb.animal.models.Animal method)}

\begin{fulllineitems}
\phantomsection\label{api:mousedb.animal.models.Animal.get_Background_display}\pysiglinewithargsret{\code{Animal.}\bfcode{get\_Background\_display}}{\emph{*moreargs}, \emph{**morekwargs}}{}
\end{fulllineitems}

\index{get\_Cause\_of\_Death\_display() (mousedb.animal.models.Animal method)}

\begin{fulllineitems}
\phantomsection\label{api:mousedb.animal.models.Animal.get_Cause_of_Death_display}\pysiglinewithargsret{\code{Animal.}\bfcode{get\_Cause\_of\_Death\_display}}{\emph{*moreargs}, \emph{**morekwargs}}{}
\end{fulllineitems}

\index{get\_Gender\_display() (mousedb.animal.models.Animal method)}

\begin{fulllineitems}
\phantomsection\label{api:mousedb.animal.models.Animal.get_Gender_display}\pysiglinewithargsret{\code{Animal.}\bfcode{get\_Gender\_display}}{\emph{*moreargs}, \emph{**morekwargs}}{}
\end{fulllineitems}

\index{get\_Genotype\_display() (mousedb.animal.models.Animal method)}

\begin{fulllineitems}
\phantomsection\label{api:mousedb.animal.models.Animal.get_Genotype_display}\pysiglinewithargsret{\code{Animal.}\bfcode{get\_Genotype\_display}}{\emph{*moreargs}, \emph{**morekwargs}}{}
\end{fulllineitems}

\index{get\_absolute\_url() (mousedb.animal.models.Animal method)}

\begin{fulllineitems}
\phantomsection\label{api:mousedb.animal.models.Animal.get_absolute_url}\pysiglinewithargsret{\code{Animal.}\bfcode{get\_absolute\_url}}{\emph{*moreargs}, \emph{**morekwargs}}{}
\end{fulllineitems}

\index{measurement\_set (mousedb.animal.models.Animal attribute)}

\begin{fulllineitems}
\phantomsection\label{api:mousedb.animal.models.Animal.measurement_set}\pysigline{\code{Animal.}\bfcode{measurement\_set}}
\end{fulllineitems}

\index{mother (mousedb.animal.models.Animal attribute)}

\begin{fulllineitems}
\phantomsection\label{api:mousedb.animal.models.Animal.mother}\pysigline{\code{Animal.}\bfcode{mother}}
\end{fulllineitems}

\index{objects (mousedb.animal.models.Animal attribute)}

\begin{fulllineitems}
\phantomsection\label{api:mousedb.animal.models.Animal.objects}\pysigline{\code{Animal.}\bfcode{objects}\strong{ = \textless{}django.db.models.manager.Manager object at 0x021C5190\textgreater{}}}
\end{fulllineitems}

\index{save() (mousedb.animal.models.Animal method)}

\begin{fulllineitems}
\phantomsection\label{api:mousedb.animal.models.Animal.save}\pysiglinewithargsret{\code{Animal.}\bfcode{save}}{}{}
The save method for Animal class is over-ridden to set Alive=False when a Death date is entered.  This is not the case for a cause of death.

\end{fulllineitems}

\index{transplantation\_set (mousedb.animal.models.Animal attribute)}

\begin{fulllineitems}
\phantomsection\label{api:mousedb.animal.models.Animal.transplantation_set}\pysigline{\code{Animal.}\bfcode{transplantation\_set}}
\end{fulllineitems}

\index{treatment\_set (mousedb.animal.models.Animal attribute)}

\begin{fulllineitems}
\phantomsection\label{api:mousedb.animal.models.Animal.treatment_set}\pysigline{\code{Animal.}\bfcode{treatment\_set}}
\end{fulllineitems}


\end{fulllineitems}

\index{Breeding (class in mousedb.animal.models)}

\begin{fulllineitems}
\phantomsection\label{api:mousedb.animal.models.Breeding}\pysiglinewithargsret{\strong{class }\code{mousedb.animal.models.}\bfcode{Breeding}}{\emph{*args}, \emph{**kwargs}}{}
Bases: \code{django.db.models.base.Model}

This data model stores information about a particular breeding set

A breeding set may contain one ore more males and females and must be defined via the progeny strain.  For example, in the case of generating a new strain, the strain indicates the new strain not the parental strains.  A breeding cage is defined as one male with one or more females.  If the breeding set is part of a timed mating experiment, then Timed\_Mating must be selected.  Breeding cages are automatically inactivated upon saving when a End date is provided.  The only required field is Strain.  By default, querysets are ordered by Strain, then Start.
\index{Breeding.DoesNotExist}

\begin{fulllineitems}
\phantomsection\label{api:mousedb.animal.models.Breeding.DoesNotExist}\pysigline{\strong{exception }\bfcode{DoesNotExist}}
Bases: \code{django.core.exceptions.ObjectDoesNotExist}

\end{fulllineitems}

\index{Females (mousedb.animal.models.Breeding attribute)}

\begin{fulllineitems}
\phantomsection\label{api:mousedb.animal.models.Breeding.Females}\pysigline{\code{Breeding.}\bfcode{Females}}
\end{fulllineitems}

\index{Male (mousedb.animal.models.Breeding attribute)}

\begin{fulllineitems}
\phantomsection\label{api:mousedb.animal.models.Breeding.Male}\pysigline{\code{Breeding.}\bfcode{Male}}
\end{fulllineitems}

\index{Breeding.MultipleObjectsReturned}

\begin{fulllineitems}
\phantomsection\label{api:mousedb.animal.models.Breeding.MultipleObjectsReturned}\pysigline{\strong{exception }\code{Breeding.}\bfcode{MultipleObjectsReturned}}
Bases: \code{django.core.exceptions.MultipleObjectsReturned}

\end{fulllineitems}

\index{Strain (mousedb.animal.models.Breeding attribute)}

\begin{fulllineitems}
\phantomsection\label{api:mousedb.animal.models.Breeding.Strain}\pysigline{\code{Breeding.}\bfcode{Strain}}
\end{fulllineitems}

\index{animal\_set (mousedb.animal.models.Breeding attribute)}

\begin{fulllineitems}
\phantomsection\label{api:mousedb.animal.models.Breeding.animal_set}\pysigline{\code{Breeding.}\bfcode{animal\_set}}
\end{fulllineitems}

\index{duration() (mousedb.animal.models.Breeding method)}

\begin{fulllineitems}
\phantomsection\label{api:mousedb.animal.models.Breeding.duration}\pysiglinewithargsret{\code{Breeding.}\bfcode{duration}}{}{}
Calculates the breeding cage's duration.

This is relative to the current date (if alive) or the date of inactivation (if not).
The duration is formatted in days.

\end{fulllineitems}

\index{get\_Crosstype\_display() (mousedb.animal.models.Breeding method)}

\begin{fulllineitems}
\phantomsection\label{api:mousedb.animal.models.Breeding.get_Crosstype_display}\pysiglinewithargsret{\code{Breeding.}\bfcode{get\_Crosstype\_display}}{\emph{*moreargs}, \emph{**morekwargs}}{}
\end{fulllineitems}

\index{get\_absolute\_url() (mousedb.animal.models.Breeding method)}

\begin{fulllineitems}
\phantomsection\label{api:mousedb.animal.models.Breeding.get_absolute_url}\pysiglinewithargsret{\code{Breeding.}\bfcode{get\_absolute\_url}}{\emph{*moreargs}, \emph{**morekwargs}}{}
\end{fulllineitems}

\index{get\_background\_display() (mousedb.animal.models.Breeding method)}

\begin{fulllineitems}
\phantomsection\label{api:mousedb.animal.models.Breeding.get_background_display}\pysiglinewithargsret{\code{Breeding.}\bfcode{get\_background\_display}}{\emph{*moreargs}, \emph{**morekwargs}}{}
\end{fulllineitems}

\index{get\_genotype\_display() (mousedb.animal.models.Breeding method)}

\begin{fulllineitems}
\phantomsection\label{api:mousedb.animal.models.Breeding.get_genotype_display}\pysiglinewithargsret{\code{Breeding.}\bfcode{get\_genotype\_display}}{\emph{*moreargs}, \emph{**morekwargs}}{}
\end{fulllineitems}

\index{male\_breeding\_location\_type() (mousedb.animal.models.Breeding method)}

\begin{fulllineitems}
\phantomsection\label{api:mousedb.animal.models.Breeding.male_breeding_location_type}\pysiglinewithargsret{\code{Breeding.}\bfcode{male\_breeding\_location\_type}}{}{}
This attribute defines whether a breeding male's current location is the same as the breeding cage.

This attribute is used to color breeding table entries such that male mice which are currently in a different cage can quickly be identified.

\end{fulllineitems}

\index{objects (mousedb.animal.models.Breeding attribute)}

\begin{fulllineitems}
\phantomsection\label{api:mousedb.animal.models.Breeding.objects}\pysigline{\code{Breeding.}\bfcode{objects}\strong{ = \textless{}django.db.models.manager.Manager object at 0x021CE6F0\textgreater{}}}
\end{fulllineitems}

\index{plugevents\_set (mousedb.animal.models.Breeding attribute)}

\begin{fulllineitems}
\phantomsection\label{api:mousedb.animal.models.Breeding.plugevents_set}\pysigline{\code{Breeding.}\bfcode{plugevents\_set}}
\end{fulllineitems}

\index{save() (mousedb.animal.models.Breeding method)}

\begin{fulllineitems}
\phantomsection\label{api:mousedb.animal.models.Breeding.save}\pysiglinewithargsret{\code{Breeding.}\bfcode{save}}{}{}
The save function for a breeding cage has to automatic over-rides, Active and the Cage for the Breeder.

In the case of Active, if an End field is specified, then the Active field is set to False.
In the case of Cage, if a Cage is provided, and animals are specified under Male or Females for a Breeding object, then the Cage field for those animals is set to that of the breeding cage.  The same is true for both Rack and Rack Position.

\end{fulllineitems}

\index{unweaned() (mousedb.animal.models.Breeding method)}

\begin{fulllineitems}
\phantomsection\label{api:mousedb.animal.models.Breeding.unweaned}\pysiglinewithargsret{\code{Breeding.}\bfcode{unweaned}}{}{}
This attribute generates a queryset of unweaned animals for this breeding cage.  It is filtered for only Alive animals.

\end{fulllineitems}


\end{fulllineitems}

\index{Strain (class in mousedb.animal.models)}

\begin{fulllineitems}
\phantomsection\label{api:mousedb.animal.models.Strain}\pysiglinewithargsret{\strong{class }\code{mousedb.animal.models.}\bfcode{Strain}}{\emph{*args}, \emph{**kwargs}}{}
Bases: \code{django.db.models.base.Model}

A data model describing a mouse strain.

This is separate from the background of a mouse.  For example a ob/ob mouse on a mixed or a black-6 background still have the same strain.  The background is defined in the animal and breeding cages.  Strain and Strain\_slug are required.
\index{Strain.DoesNotExist}

\begin{fulllineitems}
\phantomsection\label{api:mousedb.animal.models.Strain.DoesNotExist}\pysigline{\strong{exception }\bfcode{DoesNotExist}}
Bases: \code{django.core.exceptions.ObjectDoesNotExist}

\end{fulllineitems}

\index{Strain.MultipleObjectsReturned}

\begin{fulllineitems}
\phantomsection\label{api:mousedb.animal.models.Strain.MultipleObjectsReturned}\pysigline{\strong{exception }\code{Strain.}\bfcode{MultipleObjectsReturned}}
Bases: \code{django.core.exceptions.MultipleObjectsReturned}

\end{fulllineitems}

\index{animal\_set (mousedb.animal.models.Strain attribute)}

\begin{fulllineitems}
\phantomsection\label{api:mousedb.animal.models.Strain.animal_set}\pysigline{\code{Strain.}\bfcode{animal\_set}}
\end{fulllineitems}

\index{breeding\_set (mousedb.animal.models.Strain attribute)}

\begin{fulllineitems}
\phantomsection\label{api:mousedb.animal.models.Strain.breeding_set}\pysigline{\code{Strain.}\bfcode{breeding\_set}}
\end{fulllineitems}

\index{get\_absolute\_url() (mousedb.animal.models.Strain method)}

\begin{fulllineitems}
\phantomsection\label{api:mousedb.animal.models.Strain.get_absolute_url}\pysiglinewithargsret{\code{Strain.}\bfcode{get\_absolute\_url}}{\emph{*moreargs}, \emph{**morekwargs}}{}
For a Strain object, the permalinked absolute url is \emph{/strain/strain-slug}.

\end{fulllineitems}

\index{objects (mousedb.animal.models.Strain attribute)}

\begin{fulllineitems}
\phantomsection\label{api:mousedb.animal.models.Strain.objects}\pysigline{\code{Strain.}\bfcode{objects}\strong{ = \textless{}django.db.models.manager.Manager object at 0x021C2830\textgreater{}}}
\end{fulllineitems}

\index{study\_set (mousedb.animal.models.Strain attribute)}

\begin{fulllineitems}
\phantomsection\label{api:mousedb.animal.models.Strain.study_set}\pysigline{\code{Strain.}\bfcode{study\_set}}
\end{fulllineitems}


\end{fulllineitems}

\index{Strain (class in mousedb.animal.models)}

\begin{fulllineitems}
\pysiglinewithargsret{\strong{class }\code{mousedb.animal.models.}\bfcode{Strain}}{\emph{*args}, \emph{**kwargs}}{}
Bases: \code{django.db.models.base.Model}

A data model describing a mouse strain.

This is separate from the background of a mouse.  For example a ob/ob mouse on a mixed or a black-6 background still have the same strain.  The background is defined in the animal and breeding cages.  Strain and Strain\_slug are required.

\end{fulllineitems}

\index{Animal (class in mousedb.animal.models)}

\begin{fulllineitems}
\pysiglinewithargsret{\strong{class }\code{mousedb.animal.models.}\bfcode{Animal}}{\emph{*args}, \emph{**kwargs}}{}
Bases: \code{django.db.models.base.Model}

A data model describing an animal.

This data model describes a wide variety of parameters of an experimental animal.  This model is linked to the Strain.  If the parentage of a mouse is known, this can be identified (the breeding set may not be clear on this matter). Mice are automatically marked as not alive when a Death date is provided and the object is saved.  Strain, Background and Genotype are required fields.  By default, querysets are ordered first by strain then by MouseID.

\end{fulllineitems}

\index{Breeding (class in mousedb.animal.models)}

\begin{fulllineitems}
\pysiglinewithargsret{\strong{class }\code{mousedb.animal.models.}\bfcode{Breeding}}{\emph{*args}, \emph{**kwargs}}{}
Bases: \code{django.db.models.base.Model}

This data model stores information about a particular breeding set

A breeding set may contain one ore more males and females and must be defined via the progeny strain.  For example, in the case of generating a new strain, the strain indicates the new strain not the parental strains.  A breeding cage is defined as one male with one or more females.  If the breeding set is part of a timed mating experiment, then Timed\_Mating must be selected.  Breeding cages are automatically inactivated upon saving when a End date is provided.  The only required field is Strain.  By default, querysets are ordered by Strain, then Start.

\end{fulllineitems}



\subsection{Forms}
\label{api:id4}\phantomsection\label{api:module-mousedb.animal.forms}\index{mousedb.animal.forms (module)}
Forms for use in manipulating objects in the animal app.
\index{AnimalForm (class in mousedb.animal.forms)}

\begin{fulllineitems}
\phantomsection\label{api:mousedb.animal.forms.AnimalForm}\pysiglinewithargsret{\strong{class }\code{mousedb.animal.forms.}\bfcode{AnimalForm}}{\emph{data=None}, \emph{files=None}, \emph{auto\_id='id\_\%s'}, \emph{prefix=None}, \emph{initial=None}, \emph{error\_class=\textless{}class `django.forms.util.ErrorList'\textgreater{}}, \emph{label\_suffix=':'}, \emph{empty\_permitted=False}, \emph{instance=None}}{}
Bases: \code{django.forms.models.ModelForm}

This modelform provides fields for modifying animal data.

This form also automatically loads javascript and css for the datepicker jquery-ui widget.  It also includes auto
\index{AnimalForm.Media (class in mousedb.animal.forms)}

\begin{fulllineitems}
\phantomsection\label{api:mousedb.animal.forms.AnimalForm.Media}\pysigline{\strong{class }\bfcode{Media}}~\index{css (mousedb.animal.forms.AnimalForm.Media attribute)}

\begin{fulllineitems}
\phantomsection\label{api:mousedb.animal.forms.AnimalForm.Media.css}\pysigline{\bfcode{css}\strong{ = \{`all': (`javascript/jquery-autocomplete/jquery.autocomplete.css', `css/autocomplete.css')\}}}
\end{fulllineitems}

\index{js (mousedb.animal.forms.AnimalForm.Media attribute)}

\begin{fulllineitems}
\phantomsection\label{api:mousedb.animal.forms.AnimalForm.Media.js}\pysigline{\bfcode{js}\strong{ = (`javascript/jquery-ui/js/jquery-ui-1.8.2.custom.min.js', `javascript/jquery-autocomplete/jquery.autocomplete.js')}}
\end{fulllineitems}


\end{fulllineitems}

\index{AnimalForm.Meta (class in mousedb.animal.forms)}

\begin{fulllineitems}
\phantomsection\label{api:mousedb.animal.forms.AnimalForm.Meta}\pysigline{\strong{class }\code{AnimalForm.}\bfcode{Meta}}~\index{model (mousedb.animal.forms.AnimalForm.Meta attribute)}

\begin{fulllineitems}
\phantomsection\label{api:mousedb.animal.forms.AnimalForm.Meta.model}\pysigline{\bfcode{model}}
alias of \code{Animal}

\end{fulllineitems}


\end{fulllineitems}

\index{base\_fields (mousedb.animal.forms.AnimalForm attribute)}

\begin{fulllineitems}
\phantomsection\label{api:mousedb.animal.forms.AnimalForm.base_fields}\pysigline{\code{AnimalForm.}\bfcode{base\_fields}\strong{ = \{`MouseID': \textless{}django.forms.fields.IntegerField object at 0x0287C4F0\textgreater{}, `Cage': \textless{}django.forms.fields.IntegerField object at 0x0287C5D0\textgreater{}, `Rack': \textless{}django.forms.fields.CharField object at 0x0287C570\textgreater{}, `Rack\_Position': \textless{}django.forms.fields.CharField object at 0x0287C650\textgreater{}, `Strain': \textless{}django.forms.models.ModelChoiceField object at 0x0287C710\textgreater{}, `Background': \textless{}django.forms.fields.TypedChoiceField object at 0x0282DCB0\textgreater{}, `Genotype': \textless{}django.forms.fields.TypedChoiceField object at 0x0282DD50\textgreater{}, `Gender': \textless{}django.forms.fields.TypedChoiceField object at 0x0282D950\textgreater{}, `Born': \textless{}django.forms.fields.DateField object at 0x029819B0\textgreater{}, `Weaned': \textless{}django.forms.fields.DateField object at 0x0282DDB0\textgreater{}, `Death': \textless{}django.forms.fields.DateField object at 0x0282D910\textgreater{}, `Cause\_of\_Death': \textless{}django.forms.fields.TypedChoiceField object at 0x02237CD0\textgreater{}, `Backcross': \textless{}django.forms.fields.IntegerField object at 0x026E6050\textgreater{}, `Generation': \textless{}django.forms.fields.IntegerField object at 0x026E6B30\textgreater{}, `Breeding': \textless{}django.forms.models.ModelChoiceField object at 0x0279B170\textgreater{}, `Father': \textless{}ajax\_select.fields.AutoCompleteSelectField object at 0x02981670\textgreater{}, `Mother': \textless{}ajax\_select.fields.AutoCompleteSelectField object at 0x0287CD30\textgreater{}, `Markings': \textless{}django.forms.fields.CharField object at 0x0279B7D0\textgreater{}, `Notes': \textless{}django.forms.fields.CharField object at 0x0279B030\textgreater{}, `Alive': \textless{}django.forms.fields.BooleanField object at 0x0279B070\textgreater{}\}}}
\end{fulllineitems}

\index{declared\_fields (mousedb.animal.forms.AnimalForm attribute)}

\begin{fulllineitems}
\phantomsection\label{api:mousedb.animal.forms.AnimalForm.declared_fields}\pysigline{\code{AnimalForm.}\bfcode{declared\_fields}\strong{ = \{`Father': \textless{}ajax\_select.fields.AutoCompleteSelectField object at 0x02981670\textgreater{}, `Mother': \textless{}ajax\_select.fields.AutoCompleteSelectField object at 0x0287CD30\textgreater{}\}}}
\end{fulllineitems}

\index{media (mousedb.animal.forms.AnimalForm attribute)}

\begin{fulllineitems}
\phantomsection\label{api:mousedb.animal.forms.AnimalForm.media}\pysigline{\code{AnimalForm.}\bfcode{media}}
\end{fulllineitems}


\end{fulllineitems}

\index{BreedingForm (class in mousedb.animal.forms)}

\begin{fulllineitems}
\phantomsection\label{api:mousedb.animal.forms.BreedingForm}\pysiglinewithargsret{\strong{class }\code{mousedb.animal.forms.}\bfcode{BreedingForm}}{\emph{data=None}, \emph{files=None}, \emph{auto\_id='id\_\%s'}, \emph{prefix=None}, \emph{initial=None}, \emph{error\_class=\textless{}class `django.forms.util.ErrorList'\textgreater{}}, \emph{label\_suffix=':'}, \emph{empty\_permitted=False}, \emph{instance=None}}{}
Bases: \code{django.forms.models.ModelForm}

This form provides most fields for creating and entring breeding cage data.

This form is used from the url /mousedb/breeding/new and is a generic create view.  This view includes a datepicker widget for Stat and End dates and autocomplete fields for the Females and Male fields
\index{BreedingForm.Media (class in mousedb.animal.forms)}

\begin{fulllineitems}
\phantomsection\label{api:mousedb.animal.forms.BreedingForm.Media}\pysigline{\strong{class }\bfcode{Media}}~\index{css (mousedb.animal.forms.BreedingForm.Media attribute)}

\begin{fulllineitems}
\phantomsection\label{api:mousedb.animal.forms.BreedingForm.Media.css}\pysigline{\bfcode{css}\strong{ = \{`all': (`javascript/jquery-autocomplete/jquery.autocomplete.css', `css/autocomplete.css')\}}}
\end{fulllineitems}

\index{js (mousedb.animal.forms.BreedingForm.Media attribute)}

\begin{fulllineitems}
\phantomsection\label{api:mousedb.animal.forms.BreedingForm.Media.js}\pysigline{\bfcode{js}\strong{ = (`javascript/jquery-ui/js/jquery-ui-1.8.2.custom.min.js', `javascript/jquery-autocomplete/jquery.autocomplete.js')}}
\end{fulllineitems}


\end{fulllineitems}

\index{BreedingForm.Meta (class in mousedb.animal.forms)}

\begin{fulllineitems}
\phantomsection\label{api:mousedb.animal.forms.BreedingForm.Meta}\pysigline{\strong{class }\code{BreedingForm.}\bfcode{Meta}}~\index{model (mousedb.animal.forms.BreedingForm.Meta attribute)}

\begin{fulllineitems}
\phantomsection\label{api:mousedb.animal.forms.BreedingForm.Meta.model}\pysigline{\bfcode{model}}
alias of \code{Breeding}

\end{fulllineitems}


\end{fulllineitems}

\index{base\_fields (mousedb.animal.forms.BreedingForm attribute)}

\begin{fulllineitems}
\phantomsection\label{api:mousedb.animal.forms.BreedingForm.base_fields}\pysigline{\code{BreedingForm.}\bfcode{base\_fields}\strong{ = \{`Strain': \textless{}django.forms.models.ModelChoiceField object at 0x02748BB0\textgreater{}, `Cage': \textless{}django.forms.fields.CharField object at 0x02748AB0\textgreater{}, `BreedingName': \textless{}django.forms.fields.CharField object at 0x02748BF0\textgreater{}, `Start': \textless{}django.forms.fields.DateField object at 0x02748C30\textgreater{}, `End': \textless{}django.forms.fields.DateField object at 0x02748D50\textgreater{}, `Crosstype': \textless{}django.forms.fields.TypedChoiceField object at 0x02748DB0\textgreater{}, `Notes': \textless{}django.forms.fields.CharField object at 0x02748C90\textgreater{}, `Rack': \textless{}django.forms.fields.CharField object at 0x02748B70\textgreater{}, `Rack\_Position': \textless{}django.forms.fields.CharField object at 0x02748FF0\textgreater{}, `Active': \textless{}django.forms.fields.BooleanField object at 0x02748D30\textgreater{}, `Timed\_Mating': \textless{}django.forms.fields.BooleanField object at 0x02748D90\textgreater{}, `genotype': \textless{}django.forms.fields.TypedChoiceField object at 0x027489B0\textgreater{}, `background': \textless{}django.forms.fields.TypedChoiceField object at 0x02748E90\textgreater{}, `backcross': \textless{}django.forms.fields.IntegerField object at 0x02748D10\textgreater{}, `generation': \textless{}django.forms.fields.IntegerField object at 0x02748AF0\textgreater{}, `Females': \textless{}ajax\_select.fields.AutoCompleteSelectMultipleField object at 0x02748870\textgreater{}, `Male': \textless{}ajax\_select.fields.AutoCompleteSelectMultipleField object at 0x027B0130\textgreater{}\}}}
\end{fulllineitems}

\index{declared\_fields (mousedb.animal.forms.BreedingForm attribute)}

\begin{fulllineitems}
\phantomsection\label{api:mousedb.animal.forms.BreedingForm.declared_fields}\pysigline{\code{BreedingForm.}\bfcode{declared\_fields}\strong{ = \{`Male': \textless{}ajax\_select.fields.AutoCompleteSelectMultipleField object at 0x027B0130\textgreater{}, `Females': \textless{}ajax\_select.fields.AutoCompleteSelectMultipleField object at 0x02748870\textgreater{}\}}}
\end{fulllineitems}

\index{media (mousedb.animal.forms.BreedingForm attribute)}

\begin{fulllineitems}
\phantomsection\label{api:mousedb.animal.forms.BreedingForm.media}\pysigline{\code{BreedingForm.}\bfcode{media}}
\end{fulllineitems}


\end{fulllineitems}

\index{MultipleAnimalForm (class in mousedb.animal.forms)}

\begin{fulllineitems}
\phantomsection\label{api:mousedb.animal.forms.MultipleAnimalForm}\pysiglinewithargsret{\strong{class }\code{mousedb.animal.forms.}\bfcode{MultipleAnimalForm}}{\emph{data=None}, \emph{files=None}, \emph{auto\_id='id\_\%s'}, \emph{prefix=None}, \emph{initial=None}, \emph{error\_class=\textless{}class `django.forms.util.ErrorList'\textgreater{}}, \emph{label\_suffix=':'}, \emph{empty\_permitted=False}, \emph{instance=None}}{}
Bases: \code{django.forms.models.ModelForm}

This modelform provides fields for entering multiple identical copies of a set of mice.

This form only includes the required fields Background and Strain.
\index{MultipleAnimalForm.Meta (class in mousedb.animal.forms)}

\begin{fulllineitems}
\phantomsection\label{api:mousedb.animal.forms.MultipleAnimalForm.Meta}\pysigline{\strong{class }\bfcode{Meta}}~\index{fields (mousedb.animal.forms.MultipleAnimalForm.Meta attribute)}

\begin{fulllineitems}
\phantomsection\label{api:mousedb.animal.forms.MultipleAnimalForm.Meta.fields}\pysigline{\bfcode{fields}\strong{ = {[}'Background', `Strain', `Breeding', `Cage', `Rack', `Rack\_Position', `Strain', `Background', `Genotype', `Gender', `Born', `Weaned', `Backcross', `Generation', `Breeding', `Father', `Mother', `Markings', `Notes'{]}}}
\end{fulllineitems}

\index{model (mousedb.animal.forms.MultipleAnimalForm.Meta attribute)}

\begin{fulllineitems}
\phantomsection\label{api:mousedb.animal.forms.MultipleAnimalForm.Meta.model}\pysigline{\bfcode{model}}
alias of \code{Animal}

\end{fulllineitems}


\end{fulllineitems}

\index{base\_fields (mousedb.animal.forms.MultipleAnimalForm attribute)}

\begin{fulllineitems}
\phantomsection\label{api:mousedb.animal.forms.MultipleAnimalForm.base_fields}\pysigline{\code{MultipleAnimalForm.}\bfcode{base\_fields}\strong{ = \{`Background': \textless{}django.forms.fields.TypedChoiceField object at 0x0279B9D0\textgreater{}, `Strain': \textless{}django.forms.models.ModelChoiceField object at 0x0279BA10\textgreater{}, `Breeding': \textless{}django.forms.models.ModelChoiceField object at 0x027B0F30\textgreater{}, `Cage': \textless{}django.forms.fields.IntegerField object at 0x0279B430\textgreater{}, `Rack': \textless{}django.forms.fields.CharField object at 0x0279B9B0\textgreater{}, `Rack\_Position': \textless{}django.forms.fields.CharField object at 0x0279B1B0\textgreater{}, `Genotype': \textless{}django.forms.fields.TypedChoiceField object at 0x027B0E30\textgreater{}, `Gender': \textless{}django.forms.fields.TypedChoiceField object at 0x027B0F10\textgreater{}, `Born': \textless{}django.forms.fields.DateField object at 0x027B0D90\textgreater{}, `Weaned': \textless{}django.forms.fields.DateField object at 0x027B0DD0\textgreater{}, `Backcross': \textless{}django.forms.fields.IntegerField object at 0x027B0FB0\textgreater{}, `Generation': \textless{}django.forms.fields.IntegerField object at 0x027B0FD0\textgreater{}, `Father': \textless{}django.forms.models.ModelChoiceField object at 0x0279BF90\textgreater{}, `Mother': \textless{}django.forms.models.ModelChoiceField object at 0x0279B5D0\textgreater{}, `Markings': \textless{}django.forms.fields.CharField object at 0x027B0D50\textgreater{}, `Notes': \textless{}django.forms.fields.CharField object at 0x027B0F50\textgreater{}, `count': \textless{}django.forms.fields.IntegerField object at 0x0287CCD0\textgreater{}\}}}
\end{fulllineitems}

\index{declared\_fields (mousedb.animal.forms.MultipleAnimalForm attribute)}

\begin{fulllineitems}
\phantomsection\label{api:mousedb.animal.forms.MultipleAnimalForm.declared_fields}\pysigline{\code{MultipleAnimalForm.}\bfcode{declared\_fields}\strong{ = \{`count': \textless{}django.forms.fields.IntegerField object at 0x0287CCD0\textgreater{}\}}}
\end{fulllineitems}

\index{media (mousedb.animal.forms.MultipleAnimalForm attribute)}

\begin{fulllineitems}
\phantomsection\label{api:mousedb.animal.forms.MultipleAnimalForm.media}\pysigline{\code{MultipleAnimalForm.}\bfcode{media}}
\end{fulllineitems}


\end{fulllineitems}

\index{MultipleBreedingAnimalForm (class in mousedb.animal.forms)}

\begin{fulllineitems}
\phantomsection\label{api:mousedb.animal.forms.MultipleBreedingAnimalForm}\pysiglinewithargsret{\strong{class }\code{mousedb.animal.forms.}\bfcode{MultipleBreedingAnimalForm}}{\emph{data=None}, \emph{files=None}, \emph{auto\_id='id\_\%s'}, \emph{prefix=None}, \emph{initial=None}, \emph{error\_class=\textless{}class `django.forms.util.ErrorList'\textgreater{}}, \emph{label\_suffix=':'}, \emph{empty\_permitted=False}, \emph{instance=None}}{}
Bases: \code{django.forms.models.ModelForm}

This modelform provides fields for entering multiple pups within a breeding set.

The only fields presented are Born, Weaned, Gender and Count.  Several other fields will be automatically entered based on the Breeding Set entries.
\index{MultipleBreedingAnimalForm.Meta (class in mousedb.animal.forms)}

\begin{fulllineitems}
\phantomsection\label{api:mousedb.animal.forms.MultipleBreedingAnimalForm.Meta}\pysigline{\strong{class }\bfcode{Meta}}~\index{fields (mousedb.animal.forms.MultipleBreedingAnimalForm.Meta attribute)}

\begin{fulllineitems}
\phantomsection\label{api:mousedb.animal.forms.MultipleBreedingAnimalForm.Meta.fields}\pysigline{\bfcode{fields}\strong{ = {[}'Born', `Weaned', `Gender'{]}}}
\end{fulllineitems}

\index{model (mousedb.animal.forms.MultipleBreedingAnimalForm.Meta attribute)}

\begin{fulllineitems}
\phantomsection\label{api:mousedb.animal.forms.MultipleBreedingAnimalForm.Meta.model}\pysigline{\bfcode{model}}
alias of \code{Animal}

\end{fulllineitems}


\end{fulllineitems}

\index{base\_fields (mousedb.animal.forms.MultipleBreedingAnimalForm attribute)}

\begin{fulllineitems}
\phantomsection\label{api:mousedb.animal.forms.MultipleBreedingAnimalForm.base_fields}\pysigline{\code{MultipleBreedingAnimalForm.}\bfcode{base\_fields}\strong{ = \{`Born': \textless{}django.forms.fields.DateField object at 0x027B0CD0\textgreater{}, `Weaned': \textless{}django.forms.fields.DateField object at 0x027B0150\textgreater{}, `Gender': \textless{}django.forms.fields.TypedChoiceField object at 0x027B0170\textgreater{}, `count': \textless{}django.forms.fields.IntegerField object at 0x0279B130\textgreater{}\}}}
\end{fulllineitems}

\index{declared\_fields (mousedb.animal.forms.MultipleBreedingAnimalForm attribute)}

\begin{fulllineitems}
\phantomsection\label{api:mousedb.animal.forms.MultipleBreedingAnimalForm.declared_fields}\pysigline{\code{MultipleBreedingAnimalForm.}\bfcode{declared\_fields}\strong{ = \{`count': \textless{}django.forms.fields.IntegerField object at 0x0279B130\textgreater{}\}}}
\end{fulllineitems}

\index{media (mousedb.animal.forms.MultipleBreedingAnimalForm attribute)}

\begin{fulllineitems}
\phantomsection\label{api:mousedb.animal.forms.MultipleBreedingAnimalForm.media}\pysigline{\code{MultipleBreedingAnimalForm.}\bfcode{media}}
\end{fulllineitems}


\end{fulllineitems}



\subsection{Views and URLs}
\label{api:id5}\phantomsection\label{api:module-mousedb.animal.views}\index{mousedb.animal.views (module)}
These views define template redirects for the animal app.

This module contains all views for this app as class based views.
\index{AnimalCreate (class in mousedb.animal.views)}

\begin{fulllineitems}
\phantomsection\label{api:mousedb.animal.views.AnimalCreate}\pysiglinewithargsret{\strong{class }\code{mousedb.animal.views.}\bfcode{AnimalCreate}}{\emph{**kwargs}}{}
Bases: \href{http://docs.djangoproject.com/en/dev/ref/class-based-views/\#django.views.generic.edit.CreateView}{\code{django.views.generic.edit.CreateView}}

This class generates the animal-new view.

This permission restricted view takes a url in the form \emph{/animal/new} and generates an empty animal\_form.html.
This view is restricted to those with the animal.create\_animal permission.
\index{dispatch() (mousedb.animal.views.AnimalCreate method)}

\begin{fulllineitems}
\phantomsection\label{api:mousedb.animal.views.AnimalCreate.dispatch}\pysiglinewithargsret{\bfcode{dispatch}}{\emph{*args}, \emph{**kwargs}}{}
This decorator sets this view to have restricted permissions.

\end{fulllineitems}

\index{form\_class (mousedb.animal.views.AnimalCreate attribute)}

\begin{fulllineitems}
\phantomsection\label{api:mousedb.animal.views.AnimalCreate.form_class}\pysigline{\bfcode{form\_class}}
alias of \code{AnimalForm}

\end{fulllineitems}

\index{model (mousedb.animal.views.AnimalCreate attribute)}

\begin{fulllineitems}
\phantomsection\label{api:mousedb.animal.views.AnimalCreate.model}\pysigline{\bfcode{model}}
alias of \code{Animal}

\end{fulllineitems}

\index{template\_name (mousedb.animal.views.AnimalCreate attribute)}

\begin{fulllineitems}
\phantomsection\label{api:mousedb.animal.views.AnimalCreate.template_name}\pysigline{\bfcode{template\_name}\strong{ = `animal\_form.html'}}
\end{fulllineitems}


\end{fulllineitems}

\index{AnimalDelete (class in mousedb.animal.views)}

\begin{fulllineitems}
\phantomsection\label{api:mousedb.animal.views.AnimalDelete}\pysiglinewithargsret{\strong{class }\code{mousedb.animal.views.}\bfcode{AnimalDelete}}{\emph{**kwargs}}{}
Bases: \href{http://docs.djangoproject.com/en/dev/ref/class-based-views/\#django.views.generic.edit.DeleteView}{\code{django.views.generic.edit.DeleteView}}

This class generates the animal-delete view.

This permission restricted view takes a url in the form \emph{/animal/\#/delete} and passes that object to the confirm\_delete.html page.
This view is restricted to those with the animal.delete\_animal permission.
\index{context\_object\_name (mousedb.animal.views.AnimalDelete attribute)}

\begin{fulllineitems}
\phantomsection\label{api:mousedb.animal.views.AnimalDelete.context_object_name}\pysigline{\bfcode{context\_object\_name}\strong{ = `animal'}}
\end{fulllineitems}

\index{dispatch() (mousedb.animal.views.AnimalDelete method)}

\begin{fulllineitems}
\phantomsection\label{api:mousedb.animal.views.AnimalDelete.dispatch}\pysiglinewithargsret{\bfcode{dispatch}}{\emph{*args}, \emph{**kwargs}}{}
This decorator sets this view to have restricted permissions.

\end{fulllineitems}

\index{model (mousedb.animal.views.AnimalDelete attribute)}

\begin{fulllineitems}
\phantomsection\label{api:mousedb.animal.views.AnimalDelete.model}\pysigline{\bfcode{model}}
alias of \code{Animal}

\end{fulllineitems}

\index{success\_url (mousedb.animal.views.AnimalDelete attribute)}

\begin{fulllineitems}
\phantomsection\label{api:mousedb.animal.views.AnimalDelete.success_url}\pysigline{\bfcode{success\_url}\strong{ = `/animal/'}}
\end{fulllineitems}

\index{template\_name (mousedb.animal.views.AnimalDelete attribute)}

\begin{fulllineitems}
\phantomsection\label{api:mousedb.animal.views.AnimalDelete.template_name}\pysigline{\bfcode{template\_name}\strong{ = `confirm\_delete.html'}}
\end{fulllineitems}


\end{fulllineitems}

\index{AnimalDetail (class in mousedb.animal.views)}

\begin{fulllineitems}
\phantomsection\label{api:mousedb.animal.views.AnimalDetail}\pysiglinewithargsret{\strong{class }\code{mousedb.animal.views.}\bfcode{AnimalDetail}}{\emph{**kwargs}}{}
Bases: {\hyperref[api:mousedb.views.ProtectedDetailView]{\code{mousedb.views.ProtectedDetailView}}}

This view displays specific details about an animal as the animal-detail.

It takes a request in the form \emph{animal/(id)}, \emph{mice/(id)} or \emph{mouse/(id)} and renders the detail page for that mouse.  The request is defined for id not MouseID (or barcode) because this allows for details to be displayed for mice without barcode identification.
Therefore care must be taken that animal/4932 is id=4932 and not barcode=4932.  The animal name is defined at the top of the page.
This page is restricted to logged-in users.
\index{context\_object\_name (mousedb.animal.views.AnimalDetail attribute)}

\begin{fulllineitems}
\phantomsection\label{api:mousedb.animal.views.AnimalDetail.context_object_name}\pysigline{\bfcode{context\_object\_name}\strong{ = `animal'}}
\end{fulllineitems}

\index{model (mousedb.animal.views.AnimalDetail attribute)}

\begin{fulllineitems}
\phantomsection\label{api:mousedb.animal.views.AnimalDetail.model}\pysigline{\bfcode{model}}
alias of \code{Animal}

\end{fulllineitems}

\index{template\_name (mousedb.animal.views.AnimalDetail attribute)}

\begin{fulllineitems}
\phantomsection\label{api:mousedb.animal.views.AnimalDetail.template_name}\pysigline{\bfcode{template\_name}\strong{ = `animal\_detail.html'}}
\end{fulllineitems}


\end{fulllineitems}

\index{AnimalList (class in mousedb.animal.views)}

\begin{fulllineitems}
\phantomsection\label{api:mousedb.animal.views.AnimalList}\pysiglinewithargsret{\strong{class }\code{mousedb.animal.views.}\bfcode{AnimalList}}{\emph{**kwargs}}{}
Bases: {\hyperref[api:mousedb.views.ProtectedListView]{\code{mousedb.views.ProtectedListView}}}

This view generates a list of animals as animal-list

This view responds to a url in the form \emph{/animal}
It sends a variable animal containing all animals to animal\_list.html.
This view is login protected.
\index{allow\_empty (mousedb.animal.views.AnimalList attribute)}

\begin{fulllineitems}
\phantomsection\label{api:mousedb.animal.views.AnimalList.allow_empty}\pysigline{\bfcode{allow\_empty}\strong{ = True}}
\end{fulllineitems}

\index{context\_object\_name (mousedb.animal.views.AnimalList attribute)}

\begin{fulllineitems}
\phantomsection\label{api:mousedb.animal.views.AnimalList.context_object_name}\pysigline{\bfcode{context\_object\_name}\strong{ = `animal\_list'}}
\end{fulllineitems}

\index{dispatch() (mousedb.animal.views.AnimalList method)}

\begin{fulllineitems}
\phantomsection\label{api:mousedb.animal.views.AnimalList.dispatch}\pysiglinewithargsret{\bfcode{dispatch}}{\emph{*args}, \emph{**kwargs}}{}
This decorator sets this view to have restricted permissions.

\end{fulllineitems}

\index{model (mousedb.animal.views.AnimalList attribute)}

\begin{fulllineitems}
\phantomsection\label{api:mousedb.animal.views.AnimalList.model}\pysigline{\bfcode{model}}
alias of \code{Animal}

\end{fulllineitems}

\index{template\_name (mousedb.animal.views.AnimalList attribute)}

\begin{fulllineitems}
\phantomsection\label{api:mousedb.animal.views.AnimalList.template_name}\pysigline{\bfcode{template\_name}\strong{ = `animal\_list.html'}}
\end{fulllineitems}


\end{fulllineitems}

\index{AnimalListAlive (class in mousedb.animal.views)}

\begin{fulllineitems}
\phantomsection\label{api:mousedb.animal.views.AnimalListAlive}\pysiglinewithargsret{\strong{class }\code{mousedb.animal.views.}\bfcode{AnimalListAlive}}{\emph{**kwargs}}{}
Bases: {\hyperref[api:mousedb.animal.views.AnimalList]{\code{mousedb.animal.views.AnimalList}}}

This view generates a list of alive animals or animal-list-alive.

The main use for this view is to take a url in the form \emph{/animal/all} and to return a list of all alive animals to animal\_list.html in the context animal.  It also adds an extra context variable, ``list type'' as Alive.  
This view is login protected.
\index{get\_context\_data() (mousedb.animal.views.AnimalListAlive method)}

\begin{fulllineitems}
\phantomsection\label{api:mousedb.animal.views.AnimalListAlive.get_context_data}\pysiglinewithargsret{\bfcode{get\_context\_data}}{\emph{**kwargs}}{}
This add in the context of list\_type and returns this as Alive.

\end{fulllineitems}

\index{queryset (mousedb.animal.views.AnimalListAlive attribute)}

\begin{fulllineitems}
\phantomsection\label{api:mousedb.animal.views.AnimalListAlive.queryset}\pysigline{\bfcode{queryset}}
\end{fulllineitems}


\end{fulllineitems}

\index{AnimalMonthArchive (class in mousedb.animal.views)}

\begin{fulllineitems}
\phantomsection\label{api:mousedb.animal.views.AnimalMonthArchive}\pysiglinewithargsret{\strong{class }\code{mousedb.animal.views.}\bfcode{AnimalMonthArchive}}{\emph{**kwargs}}{}
Bases: \href{http://docs.djangoproject.com/en/dev/ref/class-based-views/\#django.views.generic.dates.MonthArchiveView}{\code{django.views.generic.dates.MonthArchiveView}}

This view generates a list of animals born within the specified year.

It takes a url in the form of \textbf{/date/\#\#\#\#} where \#\#\#\# is the four digit code of the year.
This view is restricted to logged in users.
\index{context\_object\_name (mousedb.animal.views.AnimalMonthArchive attribute)}

\begin{fulllineitems}
\phantomsection\label{api:mousedb.animal.views.AnimalMonthArchive.context_object_name}\pysigline{\bfcode{context\_object\_name}\strong{ = `animal\_list'}}
\end{fulllineitems}

\index{date\_field (mousedb.animal.views.AnimalMonthArchive attribute)}

\begin{fulllineitems}
\phantomsection\label{api:mousedb.animal.views.AnimalMonthArchive.date_field}\pysigline{\bfcode{date\_field}\strong{ = `Born'}}
\end{fulllineitems}

\index{dispatch() (mousedb.animal.views.AnimalMonthArchive method)}

\begin{fulllineitems}
\phantomsection\label{api:mousedb.animal.views.AnimalMonthArchive.dispatch}\pysiglinewithargsret{\bfcode{dispatch}}{\emph{*args}, \emph{**kwargs}}{}
This decorator sets this view to have restricted permissions.

\end{fulllineitems}

\index{make\_object\_list (mousedb.animal.views.AnimalMonthArchive attribute)}

\begin{fulllineitems}
\phantomsection\label{api:mousedb.animal.views.AnimalMonthArchive.make_object_list}\pysigline{\bfcode{make\_object\_list}\strong{ = True}}
\end{fulllineitems}

\index{model (mousedb.animal.views.AnimalMonthArchive attribute)}

\begin{fulllineitems}
\phantomsection\label{api:mousedb.animal.views.AnimalMonthArchive.model}\pysigline{\bfcode{model}}
alias of \code{Animal}

\end{fulllineitems}

\index{month\_format (mousedb.animal.views.AnimalMonthArchive attribute)}

\begin{fulllineitems}
\phantomsection\label{api:mousedb.animal.views.AnimalMonthArchive.month_format}\pysigline{\bfcode{month\_format}\strong{ = `\%m'}}
\end{fulllineitems}

\index{template\_name (mousedb.animal.views.AnimalMonthArchive attribute)}

\begin{fulllineitems}
\phantomsection\label{api:mousedb.animal.views.AnimalMonthArchive.template_name}\pysigline{\bfcode{template\_name}\strong{ = `animal\_list.html'}}
\end{fulllineitems}


\end{fulllineitems}

\index{AnimalUpdate (class in mousedb.animal.views)}

\begin{fulllineitems}
\phantomsection\label{api:mousedb.animal.views.AnimalUpdate}\pysiglinewithargsret{\strong{class }\code{mousedb.animal.views.}\bfcode{AnimalUpdate}}{\emph{**kwargs}}{}
Bases: \href{http://docs.djangoproject.com/en/dev/ref/class-based-views/\#django.views.generic.edit.UpdateView}{\code{django.views.generic.edit.UpdateView}}

This class generates the animal-edit view.

This permission restricted view takes a url in the form \emph{/animal/\#/edit} and generates a animal\_form.html with that object.
This view is restricted to those with the animal.update\_animal permission.
\index{context\_object\_name (mousedb.animal.views.AnimalUpdate attribute)}

\begin{fulllineitems}
\phantomsection\label{api:mousedb.animal.views.AnimalUpdate.context_object_name}\pysigline{\bfcode{context\_object\_name}\strong{ = `animal'}}
\end{fulllineitems}

\index{dispatch() (mousedb.animal.views.AnimalUpdate method)}

\begin{fulllineitems}
\phantomsection\label{api:mousedb.animal.views.AnimalUpdate.dispatch}\pysiglinewithargsret{\bfcode{dispatch}}{\emph{*args}, \emph{**kwargs}}{}
This decorator sets this view to have restricted permissions.

\end{fulllineitems}

\index{form\_class (mousedb.animal.views.AnimalUpdate attribute)}

\begin{fulllineitems}
\phantomsection\label{api:mousedb.animal.views.AnimalUpdate.form_class}\pysigline{\bfcode{form\_class}}
alias of \code{AnimalForm}

\end{fulllineitems}

\index{model (mousedb.animal.views.AnimalUpdate attribute)}

\begin{fulllineitems}
\phantomsection\label{api:mousedb.animal.views.AnimalUpdate.model}\pysigline{\bfcode{model}}
alias of \code{Animal}

\end{fulllineitems}

\index{template\_name (mousedb.animal.views.AnimalUpdate attribute)}

\begin{fulllineitems}
\phantomsection\label{api:mousedb.animal.views.AnimalUpdate.template_name}\pysigline{\bfcode{template\_name}\strong{ = `animal\_form.html'}}
\end{fulllineitems}


\end{fulllineitems}

\index{AnimalYearArchive (class in mousedb.animal.views)}

\begin{fulllineitems}
\phantomsection\label{api:mousedb.animal.views.AnimalYearArchive}\pysiglinewithargsret{\strong{class }\code{mousedb.animal.views.}\bfcode{AnimalYearArchive}}{\emph{**kwargs}}{}
Bases: \href{http://docs.djangoproject.com/en/dev/ref/class-based-views/\#django.views.generic.dates.YearArchiveView}{\code{django.views.generic.dates.YearArchiveView}}

This view generates a list of animals born within the specified year.

It takes a url in the form of \textbf{/date/\#\#\#\#} where \#\#\#\# is the four digit code of the year.
This view is restricted to logged in users.
\index{context\_object\_name (mousedb.animal.views.AnimalYearArchive attribute)}

\begin{fulllineitems}
\phantomsection\label{api:mousedb.animal.views.AnimalYearArchive.context_object_name}\pysigline{\bfcode{context\_object\_name}\strong{ = `animal\_list'}}
\end{fulllineitems}

\index{date\_field (mousedb.animal.views.AnimalYearArchive attribute)}

\begin{fulllineitems}
\phantomsection\label{api:mousedb.animal.views.AnimalYearArchive.date_field}\pysigline{\bfcode{date\_field}\strong{ = `Born'}}
\end{fulllineitems}

\index{dispatch() (mousedb.animal.views.AnimalYearArchive method)}

\begin{fulllineitems}
\phantomsection\label{api:mousedb.animal.views.AnimalYearArchive.dispatch}\pysiglinewithargsret{\bfcode{dispatch}}{\emph{*args}, \emph{**kwargs}}{}
This decorator sets this view to have restricted permissions.

\end{fulllineitems}

\index{make\_object\_list (mousedb.animal.views.AnimalYearArchive attribute)}

\begin{fulllineitems}
\phantomsection\label{api:mousedb.animal.views.AnimalYearArchive.make_object_list}\pysigline{\bfcode{make\_object\_list}\strong{ = True}}
\end{fulllineitems}

\index{model (mousedb.animal.views.AnimalYearArchive attribute)}

\begin{fulllineitems}
\phantomsection\label{api:mousedb.animal.views.AnimalYearArchive.model}\pysigline{\bfcode{model}}
alias of \code{Animal}

\end{fulllineitems}

\index{template\_name (mousedb.animal.views.AnimalYearArchive attribute)}

\begin{fulllineitems}
\phantomsection\label{api:mousedb.animal.views.AnimalYearArchive.template_name}\pysigline{\bfcode{template\_name}\strong{ = `animal\_list.html'}}
\end{fulllineitems}


\end{fulllineitems}

\index{BreedingCreate (class in mousedb.animal.views)}

\begin{fulllineitems}
\phantomsection\label{api:mousedb.animal.views.BreedingCreate}\pysiglinewithargsret{\strong{class }\code{mousedb.animal.views.}\bfcode{BreedingCreate}}{\emph{**kwargs}}{}
Bases: \href{http://docs.djangoproject.com/en/dev/ref/class-based-views/\#django.views.generic.edit.CreateView}{\code{django.views.generic.edit.CreateView}}

This class generates the breeding-new view.

This permission restricted view takes a url in the form \emph{/breeding/new} and generates an empty plugevents\_form.html.
\index{dispatch() (mousedb.animal.views.BreedingCreate method)}

\begin{fulllineitems}
\phantomsection\label{api:mousedb.animal.views.BreedingCreate.dispatch}\pysiglinewithargsret{\bfcode{dispatch}}{\emph{*args}, \emph{**kwargs}}{}
This decorator sets this view to have restricted permissions.

\end{fulllineitems}

\index{form\_class (mousedb.animal.views.BreedingCreate attribute)}

\begin{fulllineitems}
\phantomsection\label{api:mousedb.animal.views.BreedingCreate.form_class}\pysigline{\bfcode{form\_class}}
alias of \code{BreedingForm}

\end{fulllineitems}

\index{model (mousedb.animal.views.BreedingCreate attribute)}

\begin{fulllineitems}
\phantomsection\label{api:mousedb.animal.views.BreedingCreate.model}\pysigline{\bfcode{model}}
alias of \code{Breeding}

\end{fulllineitems}

\index{template\_name (mousedb.animal.views.BreedingCreate attribute)}

\begin{fulllineitems}
\phantomsection\label{api:mousedb.animal.views.BreedingCreate.template_name}\pysigline{\bfcode{template\_name}\strong{ = `breeding\_form.html'}}
\end{fulllineitems}


\end{fulllineitems}

\index{BreedingDelete (class in mousedb.animal.views)}

\begin{fulllineitems}
\phantomsection\label{api:mousedb.animal.views.BreedingDelete}\pysiglinewithargsret{\strong{class }\code{mousedb.animal.views.}\bfcode{BreedingDelete}}{\emph{**kwargs}}{}
Bases: \href{http://docs.djangoproject.com/en/dev/ref/class-based-views/\#django.views.generic.edit.DeleteView}{\code{django.views.generic.edit.DeleteView}}

This class generates the breeding-delete view.

This permission restricted view takes a url in the form \emph{/breeding/\#/delete} and passes that object to the confirm\_delete.html page.
\index{context\_object\_name (mousedb.animal.views.BreedingDelete attribute)}

\begin{fulllineitems}
\phantomsection\label{api:mousedb.animal.views.BreedingDelete.context_object_name}\pysigline{\bfcode{context\_object\_name}\strong{ = `breeding'}}
\end{fulllineitems}

\index{dispatch() (mousedb.animal.views.BreedingDelete method)}

\begin{fulllineitems}
\phantomsection\label{api:mousedb.animal.views.BreedingDelete.dispatch}\pysiglinewithargsret{\bfcode{dispatch}}{\emph{*args}, \emph{**kwargs}}{}
This decorator sets this view to have restricted permissions.

\end{fulllineitems}

\index{model (mousedb.animal.views.BreedingDelete attribute)}

\begin{fulllineitems}
\phantomsection\label{api:mousedb.animal.views.BreedingDelete.model}\pysigline{\bfcode{model}}
alias of \code{Breeding}

\end{fulllineitems}

\index{success\_url (mousedb.animal.views.BreedingDelete attribute)}

\begin{fulllineitems}
\phantomsection\label{api:mousedb.animal.views.BreedingDelete.success_url}\pysigline{\bfcode{success\_url}\strong{ = `/breeding/'}}
\end{fulllineitems}

\index{template\_name (mousedb.animal.views.BreedingDelete attribute)}

\begin{fulllineitems}
\phantomsection\label{api:mousedb.animal.views.BreedingDelete.template_name}\pysigline{\bfcode{template\_name}\strong{ = `confirm\_delete.html'}}
\end{fulllineitems}


\end{fulllineitems}

\index{BreedingDetail (class in mousedb.animal.views)}

\begin{fulllineitems}
\phantomsection\label{api:mousedb.animal.views.BreedingDetail}\pysiglinewithargsret{\strong{class }\code{mousedb.animal.views.}\bfcode{BreedingDetail}}{\emph{**kwargs}}{}
Bases: {\hyperref[api:mousedb.views.ProtectedDetailView]{\code{mousedb.views.ProtectedDetailView}}}

This view displays specific details about a breeding set.

It takes a request in the form \emph{/breeding/(breeding\_id)} and renders the detail page for that breeding set.
The breeding\_id refers to the background id of the breeding set, and not the breeding cage barcode.
This page is restricted to logged-in users.
\index{context\_object\_name (mousedb.animal.views.BreedingDetail attribute)}

\begin{fulllineitems}
\phantomsection\label{api:mousedb.animal.views.BreedingDetail.context_object_name}\pysigline{\bfcode{context\_object\_name}\strong{ = `breeding'}}
\end{fulllineitems}

\index{model (mousedb.animal.views.BreedingDetail attribute)}

\begin{fulllineitems}
\phantomsection\label{api:mousedb.animal.views.BreedingDetail.model}\pysigline{\bfcode{model}}
alias of \code{Breeding}

\end{fulllineitems}

\index{template\_name (mousedb.animal.views.BreedingDetail attribute)}

\begin{fulllineitems}
\phantomsection\label{api:mousedb.animal.views.BreedingDetail.template_name}\pysigline{\bfcode{template\_name}\strong{ = `breeding\_detail.html'}}
\end{fulllineitems}


\end{fulllineitems}

\index{BreedingList (class in mousedb.animal.views)}

\begin{fulllineitems}
\phantomsection\label{api:mousedb.animal.views.BreedingList}\pysiglinewithargsret{\strong{class }\code{mousedb.animal.views.}\bfcode{BreedingList}}{\emph{**kwargs}}{}
Bases: {\hyperref[api:mousedb.views.ProtectedListView]{\code{mousedb.views.ProtectedListView}}}

This class generates an object list for active Breeding objects.

This login protected view takes all Breeding objects and sends them to strain\_list.html as a strain\_list dictionary.  It also passes a strain\_list\_alive and cages dictionary to show the numbers for total cages and total strains.
The url for this view is \emph{/strain/}
\index{context\_object\_name (mousedb.animal.views.BreedingList attribute)}

\begin{fulllineitems}
\phantomsection\label{api:mousedb.animal.views.BreedingList.context_object_name}\pysigline{\bfcode{context\_object\_name}\strong{ = `breeding\_list'}}
\end{fulllineitems}

\index{get\_context\_data() (mousedb.animal.views.BreedingList method)}

\begin{fulllineitems}
\phantomsection\label{api:mousedb.animal.views.BreedingList.get_context_data}\pysiglinewithargsret{\bfcode{get\_context\_data}}{\emph{**kwargs}}{}
This adds into the context of breeding\_type and sets it to Active.

\end{fulllineitems}

\index{queryset (mousedb.animal.views.BreedingList attribute)}

\begin{fulllineitems}
\phantomsection\label{api:mousedb.animal.views.BreedingList.queryset}\pysigline{\bfcode{queryset}}
\end{fulllineitems}

\index{template\_name (mousedb.animal.views.BreedingList attribute)}

\begin{fulllineitems}
\phantomsection\label{api:mousedb.animal.views.BreedingList.template_name}\pysigline{\bfcode{template\_name}\strong{ = `breeding\_list.html'}}
\end{fulllineitems}


\end{fulllineitems}

\index{BreedingListAll (class in mousedb.animal.views)}

\begin{fulllineitems}
\phantomsection\label{api:mousedb.animal.views.BreedingListAll}\pysiglinewithargsret{\strong{class }\code{mousedb.animal.views.}\bfcode{BreedingListAll}}{\emph{**kwargs}}{}
Bases: {\hyperref[api:mousedb.animal.views.BreedingList]{\code{mousedb.animal.views.BreedingList}}}

This class generates a view for all breeding objects.

This class is a subclass of BreedingList, changing the queryset and the  breeding\_type context.
\index{get\_context\_data() (mousedb.animal.views.BreedingListAll method)}

\begin{fulllineitems}
\phantomsection\label{api:mousedb.animal.views.BreedingListAll.get_context_data}\pysiglinewithargsret{\bfcode{get\_context\_data}}{\emph{**kwargs}}{}
This add in the context of breeding\_type and sets it to All.

\end{fulllineitems}

\index{queryset (mousedb.animal.views.BreedingListAll attribute)}

\begin{fulllineitems}
\phantomsection\label{api:mousedb.animal.views.BreedingListAll.queryset}\pysigline{\bfcode{queryset}}
\end{fulllineitems}


\end{fulllineitems}

\index{BreedingListTimedMating (class in mousedb.animal.views)}

\begin{fulllineitems}
\phantomsection\label{api:mousedb.animal.views.BreedingListTimedMating}\pysiglinewithargsret{\strong{class }\code{mousedb.animal.views.}\bfcode{BreedingListTimedMating}}{\emph{**kwargs}}{}
Bases: {\hyperref[api:mousedb.animal.views.BreedingList]{\code{mousedb.animal.views.BreedingList}}}

This class generates a view for breeding objects, showing only timed mating cages.

This class is a subclass of BreedingList, changing the queryset and the  breeding\_type context.
\index{get\_context\_data() (mousedb.animal.views.BreedingListTimedMating method)}

\begin{fulllineitems}
\phantomsection\label{api:mousedb.animal.views.BreedingListTimedMating.get_context_data}\pysiglinewithargsret{\bfcode{get\_context\_data}}{\emph{**kwargs}}{}
This add in the context of breeding\_type and sets it to Timed\_Matings.

\end{fulllineitems}

\index{queryset (mousedb.animal.views.BreedingListTimedMating attribute)}

\begin{fulllineitems}
\phantomsection\label{api:mousedb.animal.views.BreedingListTimedMating.queryset}\pysigline{\bfcode{queryset}}
\end{fulllineitems}


\end{fulllineitems}

\index{BreedingSearch (class in mousedb.animal.views)}

\begin{fulllineitems}
\phantomsection\label{api:mousedb.animal.views.BreedingSearch}\pysiglinewithargsret{\strong{class }\code{mousedb.animal.views.}\bfcode{BreedingSearch}}{\emph{**kwargs}}{}
Bases: {\hyperref[api:mousedb.animal.views.BreedingList]{\code{mousedb.animal.views.BreedingList}}}

This class generates a view for breeding objects, showing the results of a search query for cage number.

This class is a subclass of BreedingList, changing the queryset and the  breeding\_type context as well as providing the search query and search results if available.
\index{get\_context\_data() (mousedb.animal.views.BreedingSearch method)}

\begin{fulllineitems}
\phantomsection\label{api:mousedb.animal.views.BreedingSearch.get_context_data}\pysiglinewithargsret{\bfcode{get\_context\_data}}{\emph{**kwargs}}{}
This add in the context of breeding\_type and sets it to Search it also returns the query and the queryset.

\end{fulllineitems}

\index{template\_name (mousedb.animal.views.BreedingSearch attribute)}

\begin{fulllineitems}
\phantomsection\label{api:mousedb.animal.views.BreedingSearch.template_name}\pysigline{\bfcode{template\_name}\strong{ = `breeding\_search.html'}}
\end{fulllineitems}


\end{fulllineitems}

\index{BreedingUpdate (class in mousedb.animal.views)}

\begin{fulllineitems}
\phantomsection\label{api:mousedb.animal.views.BreedingUpdate}\pysiglinewithargsret{\strong{class }\code{mousedb.animal.views.}\bfcode{BreedingUpdate}}{\emph{**kwargs}}{}
Bases: \href{http://docs.djangoproject.com/en/dev/ref/class-based-views/\#django.views.generic.edit.UpdateView}{\code{django.views.generic.edit.UpdateView}}

This class generates the breeding-edit view.

This permission restricted view takes a url in the form \emph{/breeding/\#/edit} and generates a breeding\_form.html with that object.
\index{context\_object\_name (mousedb.animal.views.BreedingUpdate attribute)}

\begin{fulllineitems}
\phantomsection\label{api:mousedb.animal.views.BreedingUpdate.context_object_name}\pysigline{\bfcode{context\_object\_name}\strong{ = `breeding'}}
\end{fulllineitems}

\index{dispatch() (mousedb.animal.views.BreedingUpdate method)}

\begin{fulllineitems}
\phantomsection\label{api:mousedb.animal.views.BreedingUpdate.dispatch}\pysiglinewithargsret{\bfcode{dispatch}}{\emph{*args}, \emph{**kwargs}}{}
This decorator sets this view to have restricted permissions.

\end{fulllineitems}

\index{form\_class (mousedb.animal.views.BreedingUpdate attribute)}

\begin{fulllineitems}
\phantomsection\label{api:mousedb.animal.views.BreedingUpdate.form_class}\pysigline{\bfcode{form\_class}}
alias of \code{BreedingForm}

\end{fulllineitems}

\index{model (mousedb.animal.views.BreedingUpdate attribute)}

\begin{fulllineitems}
\phantomsection\label{api:mousedb.animal.views.BreedingUpdate.model}\pysigline{\bfcode{model}}
alias of \code{Breeding}

\end{fulllineitems}

\index{template\_name (mousedb.animal.views.BreedingUpdate attribute)}

\begin{fulllineitems}
\phantomsection\label{api:mousedb.animal.views.BreedingUpdate.template_name}\pysigline{\bfcode{template\_name}\strong{ = `breeding\_form.html'}}
\end{fulllineitems}


\end{fulllineitems}

\index{EarTagList (class in mousedb.animal.views)}

\begin{fulllineitems}
\phantomsection\label{api:mousedb.animal.views.EarTagList}\pysiglinewithargsret{\strong{class }\code{mousedb.animal.views.}\bfcode{EarTagList}}{\emph{**kwargs}}{}
Bases: {\hyperref[api:mousedb.animal.views.AnimalList]{\code{mousedb.animal.views.AnimalList}}}

This view is for showing animals which need to be eartagged.

This list shows animals that do not have an eartag (MouseID) and are older than the age set by WEAN\_AGE in localsettings.py (default is 14 days).
It takes a view \textbf{/todo/eartag}.
This view is login protected.
\index{queryset (mousedb.animal.views.EarTagList attribute)}

\begin{fulllineitems}
\phantomsection\label{api:mousedb.animal.views.EarTagList.queryset}\pysigline{\bfcode{queryset}}
\end{fulllineitems}


\end{fulllineitems}

\index{GenotypeList (class in mousedb.animal.views)}

\begin{fulllineitems}
\phantomsection\label{api:mousedb.animal.views.GenotypeList}\pysiglinewithargsret{\strong{class }\code{mousedb.animal.views.}\bfcode{GenotypeList}}{\emph{**kwargs}}{}
Bases: {\hyperref[api:mousedb.animal.views.AnimalList]{\code{mousedb.animal.views.AnimalList}}}

This view is for showing animals which need to be genotyped.

This list shows animals that do not have a genotype (ie N.D. or ?) and are older than GENOTYPE\_AGE as designated in localsettings.py (default is 21 days).
It takes a view \textbf{/todo/genotype}.
This view is login protected.
\index{queryset (mousedb.animal.views.GenotypeList attribute)}

\begin{fulllineitems}
\phantomsection\label{api:mousedb.animal.views.GenotypeList.queryset}\pysigline{\bfcode{queryset}}
\end{fulllineitems}


\end{fulllineitems}

\index{NoCageList (class in mousedb.animal.views)}

\begin{fulllineitems}
\phantomsection\label{api:mousedb.animal.views.NoCageList}\pysiglinewithargsret{\strong{class }\code{mousedb.animal.views.}\bfcode{NoCageList}}{\emph{**kwargs}}{}
Bases: {\hyperref[api:mousedb.animal.views.AnimalList]{\code{mousedb.animal.views.AnimalList}}}

This view is for showing animals which need to have a cage entered.

This list shows animals that have no cage number and are alive.
It takes a view \textbf{/todo/no\_cage}.
This view is login protected.
\index{queryset (mousedb.animal.views.NoCageList attribute)}

\begin{fulllineitems}
\phantomsection\label{api:mousedb.animal.views.NoCageList.queryset}\pysigline{\bfcode{queryset}}
\end{fulllineitems}


\end{fulllineitems}

\index{NoRackList (class in mousedb.animal.views)}

\begin{fulllineitems}
\phantomsection\label{api:mousedb.animal.views.NoRackList}\pysiglinewithargsret{\strong{class }\code{mousedb.animal.views.}\bfcode{NoRackList}}{\emph{**kwargs}}{}
Bases: {\hyperref[api:mousedb.animal.views.AnimalList]{\code{mousedb.animal.views.AnimalList}}}

This view is for showing animals which need to have a cage entered.

This list shows animals that have no cage number and are alive.
It takes a view \textbf{/todo/no\_rack}.
This view is login protected.
\index{queryset (mousedb.animal.views.NoRackList attribute)}

\begin{fulllineitems}
\phantomsection\label{api:mousedb.animal.views.NoRackList.queryset}\pysigline{\bfcode{queryset}}
\end{fulllineitems}


\end{fulllineitems}

\index{StrainCreate (class in mousedb.animal.views)}

\begin{fulllineitems}
\phantomsection\label{api:mousedb.animal.views.StrainCreate}\pysiglinewithargsret{\strong{class }\code{mousedb.animal.views.}\bfcode{StrainCreate}}{\emph{**kwargs}}{}
Bases: \href{http://docs.djangoproject.com/en/dev/ref/class-based-views/\#django.views.generic.edit.CreateView}{\code{django.views.generic.edit.CreateView}}

This class generates the strain-new view.

This permission restricted view takes a url in the form \emph{/strain/new} and generates an empty strain\_form.html.
This view is restricted to those with the animal.create\_strain permission.
\index{dispatch() (mousedb.animal.views.StrainCreate method)}

\begin{fulllineitems}
\phantomsection\label{api:mousedb.animal.views.StrainCreate.dispatch}\pysiglinewithargsret{\bfcode{dispatch}}{\emph{*args}, \emph{**kwargs}}{}
This decorator sets this view to have restricted permissions.

\end{fulllineitems}

\index{model (mousedb.animal.views.StrainCreate attribute)}

\begin{fulllineitems}
\phantomsection\label{api:mousedb.animal.views.StrainCreate.model}\pysigline{\bfcode{model}}
alias of \code{Strain}

\end{fulllineitems}

\index{template\_name (mousedb.animal.views.StrainCreate attribute)}

\begin{fulllineitems}
\phantomsection\label{api:mousedb.animal.views.StrainCreate.template_name}\pysigline{\bfcode{template\_name}\strong{ = `strain\_form.html'}}
\end{fulllineitems}


\end{fulllineitems}

\index{StrainDelete (class in mousedb.animal.views)}

\begin{fulllineitems}
\phantomsection\label{api:mousedb.animal.views.StrainDelete}\pysiglinewithargsret{\strong{class }\code{mousedb.animal.views.}\bfcode{StrainDelete}}{\emph{**kwargs}}{}
Bases: \href{http://docs.djangoproject.com/en/dev/ref/class-based-views/\#django.views.generic.edit.DeleteView}{\code{django.views.generic.edit.DeleteView}}

This class generates the strain-delete view.

This permission restricted view takes a url in the form \emph{/strain/\#/delete} and passes that object to the confirm\_delete.html page.
This view is restricted to those with the animal.delete\_strain permission.
\index{context\_object\_name (mousedb.animal.views.StrainDelete attribute)}

\begin{fulllineitems}
\phantomsection\label{api:mousedb.animal.views.StrainDelete.context_object_name}\pysigline{\bfcode{context\_object\_name}\strong{ = `strain'}}
\end{fulllineitems}

\index{dispatch() (mousedb.animal.views.StrainDelete method)}

\begin{fulllineitems}
\phantomsection\label{api:mousedb.animal.views.StrainDelete.dispatch}\pysiglinewithargsret{\bfcode{dispatch}}{\emph{*args}, \emph{**kwargs}}{}
This decorator sets this view to have restricted permissions.

\end{fulllineitems}

\index{model (mousedb.animal.views.StrainDelete attribute)}

\begin{fulllineitems}
\phantomsection\label{api:mousedb.animal.views.StrainDelete.model}\pysigline{\bfcode{model}}
alias of \code{Strain}

\end{fulllineitems}

\index{success\_url (mousedb.animal.views.StrainDelete attribute)}

\begin{fulllineitems}
\phantomsection\label{api:mousedb.animal.views.StrainDelete.success_url}\pysigline{\bfcode{success\_url}\strong{ = `/strain/'}}
\end{fulllineitems}

\index{template\_name (mousedb.animal.views.StrainDelete attribute)}

\begin{fulllineitems}
\phantomsection\label{api:mousedb.animal.views.StrainDelete.template_name}\pysigline{\bfcode{template\_name}\strong{ = `confirm\_delete.html'}}
\end{fulllineitems}


\end{fulllineitems}

\index{StrainDetail (class in mousedb.animal.views)}

\begin{fulllineitems}
\phantomsection\label{api:mousedb.animal.views.StrainDetail}\pysiglinewithargsret{\strong{class }\code{mousedb.animal.views.}\bfcode{StrainDetail}}{\emph{**kwargs}}{}
Bases: {\hyperref[api:mousedb.views.ProtectedDetailView]{\code{mousedb.views.ProtectedDetailView}}}

This view displays specific details about a strain showing \emph{only current} related objects.

It takes a request in the form \emph{strain/(strain\_slug)/} and renders the detail page for that strain.
This view also passes along a dictionary of alive animals belonging to that strain.
This page is restricted to logged-in users.
\index{context\_object\_name (mousedb.animal.views.StrainDetail attribute)}

\begin{fulllineitems}
\phantomsection\label{api:mousedb.animal.views.StrainDetail.context_object_name}\pysigline{\bfcode{context\_object\_name}\strong{ = `strain'}}
\end{fulllineitems}

\index{get\_context\_data() (mousedb.animal.views.StrainDetail method)}

\begin{fulllineitems}
\phantomsection\label{api:mousedb.animal.views.StrainDetail.get_context_data}\pysiglinewithargsret{\bfcode{get\_context\_data}}{\emph{**kwargs}}{}
This add in the context of strain\_list\_alive (which filters for only alive animals and active) and cages which filters for the number of current cages.

\end{fulllineitems}

\index{queryset (mousedb.animal.views.StrainDetail attribute)}

\begin{fulllineitems}
\phantomsection\label{api:mousedb.animal.views.StrainDetail.queryset}\pysigline{\bfcode{queryset}}
\end{fulllineitems}

\index{slug\_field (mousedb.animal.views.StrainDetail attribute)}

\begin{fulllineitems}
\phantomsection\label{api:mousedb.animal.views.StrainDetail.slug_field}\pysigline{\bfcode{slug\_field}\strong{ = `Strain\_slug'}}
\end{fulllineitems}

\index{template\_name (mousedb.animal.views.StrainDetail attribute)}

\begin{fulllineitems}
\phantomsection\label{api:mousedb.animal.views.StrainDetail.template_name}\pysigline{\bfcode{template\_name}\strong{ = `strain\_detail.html'}}
\end{fulllineitems}


\end{fulllineitems}

\index{StrainDetailAll (class in mousedb.animal.views)}

\begin{fulllineitems}
\phantomsection\label{api:mousedb.animal.views.StrainDetailAll}\pysiglinewithargsret{\strong{class }\code{mousedb.animal.views.}\bfcode{StrainDetailAll}}{\emph{**kwargs}}{}
Bases: {\hyperref[api:mousedb.animal.views.StrainDetail]{\code{mousedb.animal.views.StrainDetail}}}

This view displays specific details about a strain showing \emph{all} related objects.

This view subclasses StrainDetail and modifies the get\_context\_data to show associated active objects.
It takes a request in the form \emph{strain/(strain\_slug)/all} and renders the detail page for that strain.
This view also passes along a dictionary of alive animals belonging to that strain.
This page is restricted to logged-in users.
\index{get\_context\_data() (mousedb.animal.views.StrainDetailAll method)}

\begin{fulllineitems}
\phantomsection\label{api:mousedb.animal.views.StrainDetailAll.get_context_data}\pysiglinewithargsret{\bfcode{get\_context\_data}}{\emph{**kwargs}}{}
This adds into the context of strain\_list\_all (which filters for all alive animals and active cages) and cages which filters for the number of current cages.

\end{fulllineitems}


\end{fulllineitems}

\index{StrainList (class in mousedb.animal.views)}

\begin{fulllineitems}
\phantomsection\label{api:mousedb.animal.views.StrainList}\pysiglinewithargsret{\strong{class }\code{mousedb.animal.views.}\bfcode{StrainList}}{\emph{**kwargs}}{}
Bases: {\hyperref[api:mousedb.views.ProtectedListView]{\code{mousedb.views.ProtectedListView}}}

This class generates an object list for Strain objects.

This login protected view takes all Strain objects and sends them to strain\_list.html as a strain\_list dictionary.  It also passes a strain\_list\_alive and cages dictionary to show the numbers for total cages and total strains.
The url for this view is \textbf{/strain/}
\index{context\_object\_name (mousedb.animal.views.StrainList attribute)}

\begin{fulllineitems}
\phantomsection\label{api:mousedb.animal.views.StrainList.context_object_name}\pysigline{\bfcode{context\_object\_name}\strong{ = `strain\_list'}}
\end{fulllineitems}

\index{get\_context\_data() (mousedb.animal.views.StrainList method)}

\begin{fulllineitems}
\phantomsection\label{api:mousedb.animal.views.StrainList.get_context_data}\pysiglinewithargsret{\bfcode{get\_context\_data}}{\emph{**kwargs}}{}
This add in the context of strain\_list\_alive (which filters for all alive animals) and cages which filters for the number of current cages.

\end{fulllineitems}

\index{model (mousedb.animal.views.StrainList attribute)}

\begin{fulllineitems}
\phantomsection\label{api:mousedb.animal.views.StrainList.model}\pysigline{\bfcode{model}}
alias of \code{Strain}

\end{fulllineitems}

\index{template\_name (mousedb.animal.views.StrainList attribute)}

\begin{fulllineitems}
\phantomsection\label{api:mousedb.animal.views.StrainList.template_name}\pysigline{\bfcode{template\_name}\strong{ = `strain\_list.html'}}
\end{fulllineitems}


\end{fulllineitems}

\index{StrainUpdate (class in mousedb.animal.views)}

\begin{fulllineitems}
\phantomsection\label{api:mousedb.animal.views.StrainUpdate}\pysiglinewithargsret{\strong{class }\code{mousedb.animal.views.}\bfcode{StrainUpdate}}{\emph{**kwargs}}{}
Bases: \href{http://docs.djangoproject.com/en/dev/ref/class-based-views/\#django.views.generic.edit.UpdateView}{\code{django.views.generic.edit.UpdateView}}

This class generates the strain-edit view.

This permission restricted view takes a url in the form \emph{/strain/\#/edit} and generates a strain\_form.html with that object.
This view is restricted to those with the animal.update\_strain permission.
\index{context\_object\_name (mousedb.animal.views.StrainUpdate attribute)}

\begin{fulllineitems}
\phantomsection\label{api:mousedb.animal.views.StrainUpdate.context_object_name}\pysigline{\bfcode{context\_object\_name}\strong{ = `strain'}}
\end{fulllineitems}

\index{dispatch() (mousedb.animal.views.StrainUpdate method)}

\begin{fulllineitems}
\phantomsection\label{api:mousedb.animal.views.StrainUpdate.dispatch}\pysiglinewithargsret{\bfcode{dispatch}}{\emph{*args}, \emph{**kwargs}}{}
This decorator sets this view to have restricted permissions.

\end{fulllineitems}

\index{model (mousedb.animal.views.StrainUpdate attribute)}

\begin{fulllineitems}
\phantomsection\label{api:mousedb.animal.views.StrainUpdate.model}\pysigline{\bfcode{model}}
alias of \code{Strain}

\end{fulllineitems}

\index{template\_name (mousedb.animal.views.StrainUpdate attribute)}

\begin{fulllineitems}
\phantomsection\label{api:mousedb.animal.views.StrainUpdate.template_name}\pysigline{\bfcode{template\_name}\strong{ = `strain\_form.html'}}
\end{fulllineitems}


\end{fulllineitems}

\index{WeanList (class in mousedb.animal.views)}

\begin{fulllineitems}
\phantomsection\label{api:mousedb.animal.views.WeanList}\pysiglinewithargsret{\strong{class }\code{mousedb.animal.views.}\bfcode{WeanList}}{\emph{**kwargs}}{}
Bases: {\hyperref[api:mousedb.animal.views.AnimalList]{\code{mousedb.animal.views.AnimalList}}}

This view is for showing animals which need to be weaned.

This list shows animals that need to be weaned.  They are animals that are older than the WEAN\_AGE and are alive.
It takes a view \textbf{/todo/wean}.
This view is login protected.
\index{queryset (mousedb.animal.views.WeanList attribute)}

\begin{fulllineitems}
\phantomsection\label{api:mousedb.animal.views.WeanList.queryset}\pysigline{\bfcode{queryset}}
\end{fulllineitems}


\end{fulllineitems}

\index{breeding\_change() (in module mousedb.animal.views)}

\begin{fulllineitems}
\phantomsection\label{api:mousedb.animal.views.breeding_change}\pysiglinewithargsret{\code{mousedb.animal.views.}\bfcode{breeding\_change}}{\emph{request}, \emph{*args}, \emph{**kwargs}}{}
This view is used to generate a form by which to change pups which belong to a particular breeding set.

This view typically is used to modify existing pups.  This might include marking animals as sacrificed, entering genotype or marking information or entering movement of mice to another cage.  It is used to show and modify several animals at once.
It takes a request in the form /breeding/(breeding\_id)/change/ and returns a form specific to the breeding set defined in breeding\_id.  breeding\_id is the background identification number of the breeding set and does not refer to the barcode of any breeding cage.
This view returns a formset in which one row represents one animal.  To add extra animals to a breeding set use /breeding/(breeding\_id)/pups/.
This view is restricted to those with the permission animal.change\_animal.

\end{fulllineitems}

\index{breeding\_pups() (in module mousedb.animal.views)}

\begin{fulllineitems}
\phantomsection\label{api:mousedb.animal.views.breeding_pups}\pysiglinewithargsret{\code{mousedb.animal.views.}\bfcode{breeding\_pups}}{\emph{request}, \emph{*args}, \emph{**kwargs}}{}
This view is used to generate a form by which to add pups which belong to a particular breeding set.

This view is intended to be used to add initial information about pups, including eartag, genotype, gender and birth or weaning information.
It takes a request in the form /breeding/(breeding\_id)/pups/ and returns a form specific to the breeding set defined in breeding\_id.  breeding\_id is the background identification number of the breeding set and does not refer to the barcode of any breeding cage.
This view is restricted to those with the permission animal.add\_animal.

\end{fulllineitems}

\index{breeding\_wean() (in module mousedb.animal.views)}

\begin{fulllineitems}
\phantomsection\label{api:mousedb.animal.views.breeding_wean}\pysiglinewithargsret{\code{mousedb.animal.views.}\bfcode{breeding\_wean}}{\emph{request}, \emph{*args}, \emph{**kwargs}}{}
This view is used to generate a form by which to wean pups which belong to a particular breeding set.

This view typically is used to wean existing pups.  This includes the MouseID, Cage, Markings, Gender and Wean Date fields.  For other fields use the breeding-change page.
It takes a request in the form /breeding/(breeding\_id)/wean/ and returns a form specific to the breeding set defined in breeding\_id.  breeding\_id is the background identification number of the breeding set and does not refer to the barcode of any breeding cage.
This view returns a formset in which one row represents one animal.  To add extra animals to a breeding set use /breeding/(breeding\_id)/pups/.
This view is restricted to those with the permission animal.change\_animal.

\end{fulllineitems}

\index{date\_archive\_year() (in module mousedb.animal.views)}

\begin{fulllineitems}
\phantomsection\label{api:mousedb.animal.views.date_archive_year}\pysiglinewithargsret{\code{mousedb.animal.views.}\bfcode{date\_archive\_year}}{\emph{request}}{}
This view will generate a table of the number of mice born on an annual basis.

This view is associated with the url name archive-home, and returns an dictionary of a date and a animal count.

\end{fulllineitems}

\index{multiple\_breeding\_pups() (in module mousedb.animal.views)}

\begin{fulllineitems}
\phantomsection\label{api:mousedb.animal.views.multiple_breeding_pups}\pysiglinewithargsret{\code{mousedb.animal.views.}\bfcode{multiple\_breeding\_pups}}{\emph{request}, \emph{breeding\_id}}{}
This view is used to enter multiple animals at the same time from a breeding cage.
\begin{description}
\item[{It will generate a form containing animal information and a number of mice.  It is intended to create several identical animals with the same attributes.}] \leavevmode
This view requres an input of a breeding\_id to generate the correct form.

\end{description}

\end{fulllineitems}

\index{multiple\_pups() (in module mousedb.animal.views)}

\begin{fulllineitems}
\phantomsection\label{api:mousedb.animal.views.multiple_pups}\pysiglinewithargsret{\code{mousedb.animal.views.}\bfcode{multiple\_pups}}{\emph{request}}{}
This view is used to enter multiple animals at the same time.

It will generate a form containing animal information and a number of mice.  It is intended to create several identical animals with the same attributes.

\end{fulllineitems}

\index{todo() (in module mousedb.animal.views)}

\begin{fulllineitems}
\phantomsection\label{api:mousedb.animal.views.todo}\pysiglinewithargsret{\code{mousedb.animal.views.}\bfcode{todo}}{\emph{request}, \emph{*args}, \emph{**kwargs}}{}
This view generates a summary of the todo lists.

The login restricted view passes lists for ear tagging, genotyping and weaning and passes them to the template todo.html.

\end{fulllineitems}

\phantomsection\label{api:module-mousedb.animal.urls}\index{mousedb.animal.urls (module)}
This package contains all url dispatchers for the animal app.

It is broken down into animal, strain, breeding, cage and date url dispatchers for clarity.
Each of these takes a different subfolder directive (ie the animal module is for \emph{/animal...} requests.


\subsection{Administrative Site Configuration}
\label{api:id6}\phantomsection\label{api:module-mousedb.animal.admin}\index{mousedb.animal.admin (module)}
Admin site settings for the animal app.
\index{AnimalAdmin (class in mousedb.animal.admin)}

\begin{fulllineitems}
\phantomsection\label{api:mousedb.animal.admin.AnimalAdmin}\pysiglinewithargsret{\strong{class }\code{mousedb.animal.admin.}\bfcode{AnimalAdmin}}{\emph{model}, \emph{admin\_site}}{}
Bases: \code{django.contrib.admin.options.ModelAdmin}

Provides parameters for animal objects within the admin interface.
\index{actions (mousedb.animal.admin.AnimalAdmin attribute)}

\begin{fulllineitems}
\phantomsection\label{api:mousedb.animal.admin.AnimalAdmin.actions}\pysigline{\bfcode{actions}\strong{ = {[}'mark\_sacrificed'{]}}}
\end{fulllineitems}

\index{fieldsets (mousedb.animal.admin.AnimalAdmin attribute)}

\begin{fulllineitems}
\phantomsection\label{api:mousedb.animal.admin.AnimalAdmin.fieldsets}\pysigline{\bfcode{fieldsets}\strong{ = ((None, \{`fields': (`Strain', `Background', `MouseID', `Cage', `Genotype', `Gender', `Born', `Weaned', `Backcross', `Generation', `Markings', `Notes', `Breeding', `Rack', `Rack\_Position')\}), (`Animal Death Information', \{`fields': (`Death', `Cause\_of\_Death', `Alive'), `classes': (`collapse',)\}))}}
\end{fulllineitems}

\index{list\_display (mousedb.animal.admin.AnimalAdmin attribute)}

\begin{fulllineitems}
\phantomsection\label{api:mousedb.animal.admin.AnimalAdmin.list_display}\pysigline{\bfcode{list\_display}\strong{ = (`MouseID', `Rack', `Rack\_Position', `Cage', `Markings', `Gender', `Genotype', `Strain', `Background', `Generation', `Backcross', `Born', `Alive', `Death')}}
\end{fulllineitems}

\index{list\_filter (mousedb.animal.admin.AnimalAdmin attribute)}

\begin{fulllineitems}
\phantomsection\label{api:mousedb.animal.admin.AnimalAdmin.list_filter}\pysigline{\bfcode{list\_filter}\strong{ = (`Alive', `Strain', `Background', `Gender', `Genotype', `Backcross')}}
\end{fulllineitems}

\index{mark\_sacrificed() (mousedb.animal.admin.AnimalAdmin method)}

\begin{fulllineitems}
\phantomsection\label{api:mousedb.animal.admin.AnimalAdmin.mark_sacrificed}\pysiglinewithargsret{\bfcode{mark\_sacrificed}}{\emph{request}, \emph{queryset}}{}
An admin action for marking several animals as sacrificed.

This action sets the selected animals as Alive=False, Death=today and Cause\_of\_Death as sacrificed.  To use other paramters, mice muse be individually marked as sacrificed.
This admin action also shows as the output the number of mice sacrificed.

\end{fulllineitems}

\index{media (mousedb.animal.admin.AnimalAdmin attribute)}

\begin{fulllineitems}
\phantomsection\label{api:mousedb.animal.admin.AnimalAdmin.media}\pysigline{\bfcode{media}}
\end{fulllineitems}

\index{radio\_fields (mousedb.animal.admin.AnimalAdmin attribute)}

\begin{fulllineitems}
\phantomsection\label{api:mousedb.animal.admin.AnimalAdmin.radio_fields}\pysigline{\bfcode{radio\_fields}\strong{ = \{`Gender': 1, `Cause\_of\_Death': 1, `Background': 1, `Strain': 1\}}}
\end{fulllineitems}

\index{raw\_id\_fields (mousedb.animal.admin.AnimalAdmin attribute)}

\begin{fulllineitems}
\phantomsection\label{api:mousedb.animal.admin.AnimalAdmin.raw_id_fields}\pysigline{\bfcode{raw\_id\_fields}\strong{ = (`Breeding',)}}
\end{fulllineitems}

\index{search\_fields (mousedb.animal.admin.AnimalAdmin attribute)}

\begin{fulllineitems}
\phantomsection\label{api:mousedb.animal.admin.AnimalAdmin.search_fields}\pysigline{\bfcode{search\_fields}\strong{ = {[}'MouseID', `Cage'{]}}}
\end{fulllineitems}


\end{fulllineitems}

\index{AnimalInline (class in mousedb.animal.admin)}

\begin{fulllineitems}
\phantomsection\label{api:mousedb.animal.admin.AnimalInline}\pysiglinewithargsret{\strong{class }\code{mousedb.animal.admin.}\bfcode{AnimalInline}}{\emph{parent\_model}, \emph{admin\_site}}{}
Bases: \code{django.contrib.admin.options.TabularInline}

Provides an inline tabular formset for animal objects.

Currently used with the breeding admin page.
\index{fields (mousedb.animal.admin.AnimalInline attribute)}

\begin{fulllineitems}
\phantomsection\label{api:mousedb.animal.admin.AnimalInline.fields}\pysigline{\bfcode{fields}\strong{ = (`Strain', `Background', `MouseID', `Cage', `Genotype', `Gender', `Born', `Weaned', `Generation', `Markings', `Notes', `Rack', `Rack\_Position')}}
\end{fulllineitems}

\index{media (mousedb.animal.admin.AnimalInline attribute)}

\begin{fulllineitems}
\phantomsection\label{api:mousedb.animal.admin.AnimalInline.media}\pysigline{\bfcode{media}}
\end{fulllineitems}

\index{model (mousedb.animal.admin.AnimalInline attribute)}

\begin{fulllineitems}
\phantomsection\label{api:mousedb.animal.admin.AnimalInline.model}\pysigline{\bfcode{model}}
alias of \code{Animal}

\end{fulllineitems}

\index{radio\_fields (mousedb.animal.admin.AnimalInline attribute)}

\begin{fulllineitems}
\phantomsection\label{api:mousedb.animal.admin.AnimalInline.radio_fields}\pysigline{\bfcode{radio\_fields}\strong{ = \{`Gender': 1, `Genotype': 1, `Background': 1, `Strain': 1\}}}
\end{fulllineitems}


\end{fulllineitems}

\index{BreedingAdmin (class in mousedb.animal.admin)}

\begin{fulllineitems}
\phantomsection\label{api:mousedb.animal.admin.BreedingAdmin}\pysiglinewithargsret{\strong{class }\code{mousedb.animal.admin.}\bfcode{BreedingAdmin}}{\emph{model}, \emph{admin\_site}}{}
Bases: \code{django.contrib.admin.options.ModelAdmin}

Settings in the admin interface for dealing with Breeding objects.

This interface also includes an form for adding objects associated with this breeding cage.
\index{actions (mousedb.animal.admin.BreedingAdmin attribute)}

\begin{fulllineitems}
\phantomsection\label{api:mousedb.animal.admin.BreedingAdmin.actions}\pysigline{\bfcode{actions}\strong{ = {[}'mark\_deactivated'{]}}}
\end{fulllineitems}

\index{fields (mousedb.animal.admin.BreedingAdmin attribute)}

\begin{fulllineitems}
\phantomsection\label{api:mousedb.animal.admin.BreedingAdmin.fields}\pysigline{\bfcode{fields}\strong{ = (`Male', `Females', `Timed\_Mating', `Cage', `Rack', `Rack\_Position', `BreedingName', `Strain', `Start', `End', `Crosstype', `Notes', `Active')}}
\end{fulllineitems}

\index{inlines (mousedb.animal.admin.BreedingAdmin attribute)}

\begin{fulllineitems}
\phantomsection\label{api:mousedb.animal.admin.BreedingAdmin.inlines}\pysigline{\bfcode{inlines}\strong{ = {[}\textless{}class `mousedb.animal.admin.AnimalInline'\textgreater{}{]}}}
\end{fulllineitems}

\index{list\_display (mousedb.animal.admin.BreedingAdmin attribute)}

\begin{fulllineitems}
\phantomsection\label{api:mousedb.animal.admin.BreedingAdmin.list_display}\pysigline{\bfcode{list\_display}\strong{ = (`Cage', `Start', `Rack', `Rack\_Position', `Strain', `Crosstype', `BreedingName', `Notes', `Active')}}
\end{fulllineitems}

\index{list\_filter (mousedb.animal.admin.BreedingAdmin attribute)}

\begin{fulllineitems}
\phantomsection\label{api:mousedb.animal.admin.BreedingAdmin.list_filter}\pysigline{\bfcode{list\_filter}\strong{ = (`Timed\_Mating', `Strain', `Active', `Crosstype')}}
\end{fulllineitems}

\index{mark\_deactivated() (mousedb.animal.admin.BreedingAdmin method)}

\begin{fulllineitems}
\phantomsection\label{api:mousedb.animal.admin.BreedingAdmin.mark_deactivated}\pysiglinewithargsret{\bfcode{mark\_deactivated}}{\emph{request}, \emph{queryset}}{}
An admin action for marking several cages as inactive.

This action sets the selected cages as Active=False and Death=today.
This admin action also shows as the output the number of mice sacrificed.

\end{fulllineitems}

\index{media (mousedb.animal.admin.BreedingAdmin attribute)}

\begin{fulllineitems}
\phantomsection\label{api:mousedb.animal.admin.BreedingAdmin.media}\pysigline{\bfcode{media}}
\end{fulllineitems}

\index{ordering (mousedb.animal.admin.BreedingAdmin attribute)}

\begin{fulllineitems}
\phantomsection\label{api:mousedb.animal.admin.BreedingAdmin.ordering}\pysigline{\bfcode{ordering}\strong{ = (`Active', `Start')}}
\end{fulllineitems}

\index{radio\_fields (mousedb.animal.admin.BreedingAdmin attribute)}

\begin{fulllineitems}
\phantomsection\label{api:mousedb.animal.admin.BreedingAdmin.radio_fields}\pysigline{\bfcode{radio\_fields}\strong{ = \{`Crosstype': 2, `Strain': 1\}}}
\end{fulllineitems}

\index{raw\_id\_fields (mousedb.animal.admin.BreedingAdmin attribute)}

\begin{fulllineitems}
\phantomsection\label{api:mousedb.animal.admin.BreedingAdmin.raw_id_fields}\pysigline{\bfcode{raw\_id\_fields}\strong{ = (`Male', `Females')}}
\end{fulllineitems}

\index{search\_fields (mousedb.animal.admin.BreedingAdmin attribute)}

\begin{fulllineitems}
\phantomsection\label{api:mousedb.animal.admin.BreedingAdmin.search_fields}\pysigline{\bfcode{search\_fields}\strong{ = {[}'Cage'{]}}}
\end{fulllineitems}


\end{fulllineitems}

\index{StrainAdmin (class in mousedb.animal.admin)}

\begin{fulllineitems}
\phantomsection\label{api:mousedb.animal.admin.StrainAdmin}\pysiglinewithargsret{\strong{class }\code{mousedb.animal.admin.}\bfcode{StrainAdmin}}{\emph{model}, \emph{admin\_site}}{}
Bases: \code{django.contrib.admin.options.ModelAdmin}

Settings in the admin interface for dealing with Strain objects.
\index{fields (mousedb.animal.admin.StrainAdmin attribute)}

\begin{fulllineitems}
\phantomsection\label{api:mousedb.animal.admin.StrainAdmin.fields}\pysigline{\bfcode{fields}\strong{ = (`Strain', `Strain\_slug', `Comments', `Source')}}
\end{fulllineitems}

\index{media (mousedb.animal.admin.StrainAdmin attribute)}

\begin{fulllineitems}
\phantomsection\label{api:mousedb.animal.admin.StrainAdmin.media}\pysigline{\bfcode{media}}
\end{fulllineitems}

\index{prepopulated\_fields (mousedb.animal.admin.StrainAdmin attribute)}

\begin{fulllineitems}
\phantomsection\label{api:mousedb.animal.admin.StrainAdmin.prepopulated_fields}\pysigline{\bfcode{prepopulated\_fields}\strong{ = \{`Strain\_slug': (`Strain',)\}}}
\end{fulllineitems}


\end{fulllineitems}



\subsection{Test Files}
\label{api:id7}\phantomsection\label{api:module-mousedb.animal.tests}\index{mousedb.animal.tests (module)}
This file contains tests for the animal application.

These tests will verify generation and function of a new breeding, strain and animal object.
\index{AnimalModelTests (class in mousedb.animal.tests)}

\begin{fulllineitems}
\phantomsection\label{api:mousedb.animal.tests.AnimalModelTests}\pysiglinewithargsret{\strong{class }\code{mousedb.animal.tests.}\bfcode{AnimalModelTests}}{\emph{methodName='runTest'}}{}
Bases: \code{django.test.testcases.TestCase}

Tests the model attributes of Animal objects contained in the animal app.
\index{fixtures (mousedb.animal.tests.AnimalModelTests attribute)}

\begin{fulllineitems}
\phantomsection\label{api:mousedb.animal.tests.AnimalModelTests.fixtures}\pysigline{\bfcode{fixtures}\strong{ = {[}'test\_animals', `test\_strain'{]}}}
\end{fulllineitems}

\index{setUp() (mousedb.animal.tests.AnimalModelTests method)}

\begin{fulllineitems}
\phantomsection\label{api:mousedb.animal.tests.AnimalModelTests.setUp}\pysiglinewithargsret{\bfcode{setUp}}{}{}
Instantiate the test client.

\end{fulllineitems}

\index{tearDown() (mousedb.animal.tests.AnimalModelTests method)}

\begin{fulllineitems}
\phantomsection\label{api:mousedb.animal.tests.AnimalModelTests.tearDown}\pysiglinewithargsret{\bfcode{tearDown}}{}{}
Depopulate created model instances from test database.

\end{fulllineitems}

\index{test\_animal\_unicode() (mousedb.animal.tests.AnimalModelTests method)}

\begin{fulllineitems}
\phantomsection\label{api:mousedb.animal.tests.AnimalModelTests.test_animal_unicode}\pysiglinewithargsret{\bfcode{test\_animal\_unicode}}{}{}
This is a test for creating a new animal object, with only the minimum fields being entered.  It then tests that the correct unicode representation is being generated.

\end{fulllineitems}

\index{test\_create\_animal\_minimal() (mousedb.animal.tests.AnimalModelTests method)}

\begin{fulllineitems}
\phantomsection\label{api:mousedb.animal.tests.AnimalModelTests.test_create_animal_minimal}\pysiglinewithargsret{\bfcode{test\_create\_animal\_minimal}}{}{}
This is a test for creating a new animal object, with only the minimum fields being entered

\end{fulllineitems}


\end{fulllineitems}

\index{AnimalViewTests (class in mousedb.animal.tests)}

\begin{fulllineitems}
\phantomsection\label{api:mousedb.animal.tests.AnimalViewTests}\pysiglinewithargsret{\strong{class }\code{mousedb.animal.tests.}\bfcode{AnimalViewTests}}{\emph{methodName='runTest'}}{}
Bases: \code{django.test.testcases.TestCase}

Tests the views associated with animal objects.
\index{fixtures (mousedb.animal.tests.AnimalViewTests attribute)}

\begin{fulllineitems}
\phantomsection\label{api:mousedb.animal.tests.AnimalViewTests.fixtures}\pysigline{\bfcode{fixtures}\strong{ = {[}'test\_animals', `test\_strain'{]}}}
\end{fulllineitems}

\index{setUp() (mousedb.animal.tests.AnimalViewTests method)}

\begin{fulllineitems}
\phantomsection\label{api:mousedb.animal.tests.AnimalViewTests.setUp}\pysiglinewithargsret{\bfcode{setUp}}{}{}
Instantiate the test client.  Creates a test user.

\end{fulllineitems}

\index{tearDown() (mousedb.animal.tests.AnimalViewTests method)}

\begin{fulllineitems}
\phantomsection\label{api:mousedb.animal.tests.AnimalViewTests.tearDown}\pysiglinewithargsret{\bfcode{tearDown}}{}{}
Depopulate created model instances from test database.

\end{fulllineitems}

\index{test\_animal\_delete() (mousedb.animal.tests.AnimalViewTests method)}

\begin{fulllineitems}
\phantomsection\label{api:mousedb.animal.tests.AnimalViewTests.test_animal_delete}\pysiglinewithargsret{\bfcode{test\_animal\_delete}}{}{}
This test checks the view which displays an animal deletion page.

It checks for the correct templates and status code and that the animal is passed correctly to the context.

\end{fulllineitems}

\index{test\_animal\_detail() (mousedb.animal.tests.AnimalViewTests method)}

\begin{fulllineitems}
\phantomsection\label{api:mousedb.animal.tests.AnimalViewTests.test_animal_detail}\pysiglinewithargsret{\bfcode{test\_animal\_detail}}{}{}
This tests the animal-detail view, ensuring that templates are loaded correctly.

This view uses a user with superuser permissions so does not test the permission levels for this view.

\end{fulllineitems}

\index{test\_animal\_edit() (mousedb.animal.tests.AnimalViewTests method)}

\begin{fulllineitems}
\phantomsection\label{api:mousedb.animal.tests.AnimalViewTests.test_animal_edit}\pysiglinewithargsret{\bfcode{test\_animal\_edit}}{}{}
This test checks the view which displays a animal edit page.

It checks for the correct templates and status code and that the animal is passed correctly to the context.

\end{fulllineitems}

\index{test\_animal\_list() (mousedb.animal.tests.AnimalViewTests method)}

\begin{fulllineitems}
\phantomsection\label{api:mousedb.animal.tests.AnimalViewTests.test_animal_list}\pysiglinewithargsret{\bfcode{test\_animal\_list}}{}{}
This test checks the view which displays a breeding list page showing active animals.  It checks for the correct templates and status code.

\end{fulllineitems}

\index{test\_animal\_list\_all() (mousedb.animal.tests.AnimalViewTests method)}

\begin{fulllineitems}
\phantomsection\label{api:mousedb.animal.tests.AnimalViewTests.test_animal_list_all}\pysiglinewithargsret{\bfcode{test\_animal\_list\_all}}{}{}
This test checks the view which displays a breeding list page showing all animals.  It checks for the correct templates and status code.

\end{fulllineitems}

\index{test\_animal\_new() (mousedb.animal.tests.AnimalViewTests method)}

\begin{fulllineitems}
\phantomsection\label{api:mousedb.animal.tests.AnimalViewTests.test_animal_new}\pysiglinewithargsret{\bfcode{test\_animal\_new}}{}{}
This test checks the view which displays a new animal.

It checks for the correct templates and status code.

\end{fulllineitems}


\end{fulllineitems}

\index{BreedingModelTests (class in mousedb.animal.tests)}

\begin{fulllineitems}
\phantomsection\label{api:mousedb.animal.tests.BreedingModelTests}\pysiglinewithargsret{\strong{class }\code{mousedb.animal.tests.}\bfcode{BreedingModelTests}}{\emph{methodName='runTest'}}{}
Bases: \code{django.test.testcases.TestCase}

Tests the model attributes of Breeding objects contained in the animal app.
\index{fixtures (mousedb.animal.tests.BreedingModelTests attribute)}

\begin{fulllineitems}
\phantomsection\label{api:mousedb.animal.tests.BreedingModelTests.fixtures}\pysigline{\bfcode{fixtures}\strong{ = {[}'test\_group', `test\_animals', `test\_breeding', `test\_strain'{]}}}
\end{fulllineitems}

\index{setUp() (mousedb.animal.tests.BreedingModelTests method)}

\begin{fulllineitems}
\phantomsection\label{api:mousedb.animal.tests.BreedingModelTests.setUp}\pysiglinewithargsret{\bfcode{setUp}}{}{}
Instantiate the test client.

\end{fulllineitems}

\index{tearDown() (mousedb.animal.tests.BreedingModelTests method)}

\begin{fulllineitems}
\phantomsection\label{api:mousedb.animal.tests.BreedingModelTests.tearDown}\pysiglinewithargsret{\bfcode{tearDown}}{}{}
Depopulate created model instances from test database.

\end{fulllineitems}

\index{test\_autoset\_active\_state() (mousedb.animal.tests.BreedingModelTests method)}

\begin{fulllineitems}
\phantomsection\label{api:mousedb.animal.tests.BreedingModelTests.test_autoset_active_state}\pysiglinewithargsret{\bfcode{test\_autoset\_active\_state}}{}{}
This is a test for creating a new breeding object, with only the minimum being entered.  That object is then tested for the active state being automatically set when a End date is specified.

\end{fulllineitems}

\index{test\_create\_breeding\_minimal() (mousedb.animal.tests.BreedingModelTests method)}

\begin{fulllineitems}
\phantomsection\label{api:mousedb.animal.tests.BreedingModelTests.test_create_breeding_minimal}\pysiglinewithargsret{\bfcode{test\_create\_breeding\_minimal}}{}{}
This is a test for creating a new breeding object, with only the minimum being entered.

\end{fulllineitems}

\index{test\_duration() (mousedb.animal.tests.BreedingModelTests method)}

\begin{fulllineitems}
\phantomsection\label{api:mousedb.animal.tests.BreedingModelTests.test_duration}\pysiglinewithargsret{\bfcode{test\_duration}}{}{}
This test verifies that the duration is set correctly.

\end{fulllineitems}

\index{test\_study\_absolute\_url() (mousedb.animal.tests.BreedingModelTests method)}

\begin{fulllineitems}
\phantomsection\label{api:mousedb.animal.tests.BreedingModelTests.test_study_absolute_url}\pysiglinewithargsret{\bfcode{test\_study\_absolute\_url}}{}{}
This test verifies that the absolute url of a breeding object is set correctly.

\end{fulllineitems}

\index{test\_unweaned() (mousedb.animal.tests.BreedingModelTests method)}

\begin{fulllineitems}
\phantomsection\label{api:mousedb.animal.tests.BreedingModelTests.test_unweaned}\pysiglinewithargsret{\bfcode{test\_unweaned}}{}{}
This is a test for the unweaned animal list.  It creates several animals for a breeding object and tests that they are tagged as unweaned.  They are then weaned and retested to be tagged as not unweaned.  This test is incomplete.

\end{fulllineitems}


\end{fulllineitems}

\index{BreedingViewTests (class in mousedb.animal.tests)}

\begin{fulllineitems}
\phantomsection\label{api:mousedb.animal.tests.BreedingViewTests}\pysiglinewithargsret{\strong{class }\code{mousedb.animal.tests.}\bfcode{BreedingViewTests}}{\emph{methodName='runTest'}}{}
Bases: \code{django.test.testcases.TestCase}

These are tests for views based on Breeding objects.  Included are tests for breeding list (active and all), details, create, update and delete pages as well as for the timed mating lists.
\index{fixtures (mousedb.animal.tests.BreedingViewTests attribute)}

\begin{fulllineitems}
\phantomsection\label{api:mousedb.animal.tests.BreedingViewTests.fixtures}\pysigline{\bfcode{fixtures}\strong{ = {[}'test\_breeding', `test\_animals', `test\_strain'{]}}}
\end{fulllineitems}

\index{setUp() (mousedb.animal.tests.BreedingViewTests method)}

\begin{fulllineitems}
\phantomsection\label{api:mousedb.animal.tests.BreedingViewTests.setUp}\pysiglinewithargsret{\bfcode{setUp}}{}{}
\end{fulllineitems}

\index{tearDown() (mousedb.animal.tests.BreedingViewTests method)}

\begin{fulllineitems}
\phantomsection\label{api:mousedb.animal.tests.BreedingViewTests.tearDown}\pysiglinewithargsret{\bfcode{tearDown}}{}{}
\end{fulllineitems}

\index{test\_breeding\_delete() (mousedb.animal.tests.BreedingViewTests method)}

\begin{fulllineitems}
\phantomsection\label{api:mousedb.animal.tests.BreedingViewTests.test_breeding_delete}\pysiglinewithargsret{\bfcode{test\_breeding\_delete}}{}{}
This test checks the view which displays a breeding detail page.  It checks for the correct templates and status code.

\end{fulllineitems}

\index{test\_breeding\_detail() (mousedb.animal.tests.BreedingViewTests method)}

\begin{fulllineitems}
\phantomsection\label{api:mousedb.animal.tests.BreedingViewTests.test_breeding_detail}\pysiglinewithargsret{\bfcode{test\_breeding\_detail}}{}{}
This test checks the view which displays a breeding detail page.  It checks for the correct templates and status code.

\end{fulllineitems}

\index{test\_breeding\_edit() (mousedb.animal.tests.BreedingViewTests method)}

\begin{fulllineitems}
\phantomsection\label{api:mousedb.animal.tests.BreedingViewTests.test_breeding_edit}\pysiglinewithargsret{\bfcode{test\_breeding\_edit}}{}{}
This test checks the view which displays a breeding edit page.  It checks for the correct templates and status code.

\end{fulllineitems}

\index{test\_breeding\_list() (mousedb.animal.tests.BreedingViewTests method)}

\begin{fulllineitems}
\phantomsection\label{api:mousedb.animal.tests.BreedingViewTests.test_breeding_list}\pysiglinewithargsret{\bfcode{test\_breeding\_list}}{}{}
This test checks the view which displays a breeding list page showing active breeding cages.  It checks for the correct templates and status code.

\end{fulllineitems}

\index{test\_breeding\_list\_all() (mousedb.animal.tests.BreedingViewTests method)}

\begin{fulllineitems}
\phantomsection\label{api:mousedb.animal.tests.BreedingViewTests.test_breeding_list_all}\pysiglinewithargsret{\bfcode{test\_breeding\_list\_all}}{}{}
This test checks the view which displays a breeding list page, for all the cages.  It checks for the correct templates and status code.

\end{fulllineitems}

\index{test\_breeding\_new() (mousedb.animal.tests.BreedingViewTests method)}

\begin{fulllineitems}
\phantomsection\label{api:mousedb.animal.tests.BreedingViewTests.test_breeding_new}\pysiglinewithargsret{\bfcode{test\_breeding\_new}}{}{}
This test checks the view which displays a new breeding page.  It checks for the correct templates and status code.

\end{fulllineitems}

\index{test\_timed\_mating\_list() (mousedb.animal.tests.BreedingViewTests method)}

\begin{fulllineitems}
\phantomsection\label{api:mousedb.animal.tests.BreedingViewTests.test_timed_mating_list}\pysiglinewithargsret{\bfcode{test\_timed\_mating\_list}}{}{}
This test checks the view which displays a breeding list page, for all the cages.  It checks for the correct templates and status code.

\end{fulllineitems}


\end{fulllineitems}

\index{CageViewTests (class in mousedb.animal.tests)}

\begin{fulllineitems}
\phantomsection\label{api:mousedb.animal.tests.CageViewTests}\pysiglinewithargsret{\strong{class }\code{mousedb.animal.tests.}\bfcode{CageViewTests}}{\emph{methodName='runTest'}}{}
Bases: \code{django.test.testcases.TestCase}

These are tests for views based on animal objects as directed by cage urls.  Included are tests for cage-list, cage-list-all and cage-detail
\index{fixtures (mousedb.animal.tests.CageViewTests attribute)}

\begin{fulllineitems}
\phantomsection\label{api:mousedb.animal.tests.CageViewTests.fixtures}\pysigline{\bfcode{fixtures}\strong{ = {[}'test\_animals', `test\_strain'{]}}}
\end{fulllineitems}

\index{setUp() (mousedb.animal.tests.CageViewTests method)}

\begin{fulllineitems}
\phantomsection\label{api:mousedb.animal.tests.CageViewTests.setUp}\pysiglinewithargsret{\bfcode{setUp}}{}{}
\end{fulllineitems}

\index{tearDown() (mousedb.animal.tests.CageViewTests method)}

\begin{fulllineitems}
\phantomsection\label{api:mousedb.animal.tests.CageViewTests.tearDown}\pysiglinewithargsret{\bfcode{tearDown}}{}{}
\end{fulllineitems}

\index{test\_cage\_detail() (mousedb.animal.tests.CageViewTests method)}

\begin{fulllineitems}
\phantomsection\label{api:mousedb.animal.tests.CageViewTests.test_cage_detail}\pysiglinewithargsret{\bfcode{test\_cage\_detail}}{}{}
This test checks the view which displays a animal list page showing all animals with a specified cage number.  It checks for the correct templates and status code.

\end{fulllineitems}

\index{test\_cage\_list() (mousedb.animal.tests.CageViewTests method)}

\begin{fulllineitems}
\phantomsection\label{api:mousedb.animal.tests.CageViewTests.test_cage_list}\pysiglinewithargsret{\bfcode{test\_cage\_list}}{}{}
This test checks the view which displays a cage list page showing all animals.  It checks for the correct templates and status code.

\end{fulllineitems}


\end{fulllineitems}

\index{DateViewTests (class in mousedb.animal.tests)}

\begin{fulllineitems}
\phantomsection\label{api:mousedb.animal.tests.DateViewTests}\pysiglinewithargsret{\strong{class }\code{mousedb.animal.tests.}\bfcode{DateViewTests}}{\emph{methodName='runTest'}}{}
Bases: \code{django.test.testcases.TestCase}

These are tests for views based on animal objects as directed by date based urls.  Included are tests for archive-home, archive-month and archive-year
\index{fixtures (mousedb.animal.tests.DateViewTests attribute)}

\begin{fulllineitems}
\phantomsection\label{api:mousedb.animal.tests.DateViewTests.fixtures}\pysigline{\bfcode{fixtures}\strong{ = {[}'test\_animals', `test\_strain'{]}}}
\end{fulllineitems}

\index{setUp() (mousedb.animal.tests.DateViewTests method)}

\begin{fulllineitems}
\phantomsection\label{api:mousedb.animal.tests.DateViewTests.setUp}\pysiglinewithargsret{\bfcode{setUp}}{}{}
\end{fulllineitems}

\index{tearDown() (mousedb.animal.tests.DateViewTests method)}

\begin{fulllineitems}
\phantomsection\label{api:mousedb.animal.tests.DateViewTests.tearDown}\pysiglinewithargsret{\bfcode{tearDown}}{}{}
\end{fulllineitems}

\index{test\_archive\_home() (mousedb.animal.tests.DateViewTests method)}

\begin{fulllineitems}
\phantomsection\label{api:mousedb.animal.tests.DateViewTests.test_archive_home}\pysiglinewithargsret{\bfcode{test\_archive\_home}}{}{}
This test checks the view which displays a summary of the birthdates of animals.  It checks for the correct templates and status code.

\end{fulllineitems}

\index{test\_archive\_month() (mousedb.animal.tests.DateViewTests method)}

\begin{fulllineitems}
\phantomsection\label{api:mousedb.animal.tests.DateViewTests.test_archive_month}\pysiglinewithargsret{\bfcode{test\_archive\_month}}{}{}
This test checks the view which displays a list of the animals, filtered by month.  It checks for the correct templates and status code.

\end{fulllineitems}

\index{test\_archive\_year() (mousedb.animal.tests.DateViewTests method)}

\begin{fulllineitems}
\phantomsection\label{api:mousedb.animal.tests.DateViewTests.test_archive_year}\pysiglinewithargsret{\bfcode{test\_archive\_year}}{}{}
This test checks the view which displays a list of the animals, filtered by year.  It checks for the correct templates and status code.

\end{fulllineitems}


\end{fulllineitems}

\index{StrainModelTests (class in mousedb.animal.tests)}

\begin{fulllineitems}
\phantomsection\label{api:mousedb.animal.tests.StrainModelTests}\pysiglinewithargsret{\strong{class }\code{mousedb.animal.tests.}\bfcode{StrainModelTests}}{\emph{methodName='runTest'}}{}
Bases: \code{django.test.testcases.TestCase}

Tests the model attributes of Strain objects contained in the animal app.
\index{setUp() (mousedb.animal.tests.StrainModelTests method)}

\begin{fulllineitems}
\phantomsection\label{api:mousedb.animal.tests.StrainModelTests.setUp}\pysiglinewithargsret{\bfcode{setUp}}{}{}
Instantiate the test client.  Creates a test user.

\end{fulllineitems}

\index{tearDown() (mousedb.animal.tests.StrainModelTests method)}

\begin{fulllineitems}
\phantomsection\label{api:mousedb.animal.tests.StrainModelTests.tearDown}\pysiglinewithargsret{\bfcode{tearDown}}{}{}
Depopulate created model instances from test database.

\end{fulllineitems}

\index{test\_create\_strain\_all() (mousedb.animal.tests.StrainModelTests method)}

\begin{fulllineitems}
\phantomsection\label{api:mousedb.animal.tests.StrainModelTests.test_create_strain_all}\pysiglinewithargsret{\bfcode{test\_create\_strain\_all}}{}{}
This is a test for creating a new strain object, with only all fields being entered

\end{fulllineitems}

\index{test\_create\_strain\_minimal() (mousedb.animal.tests.StrainModelTests method)}

\begin{fulllineitems}
\phantomsection\label{api:mousedb.animal.tests.StrainModelTests.test_create_strain_minimal}\pysiglinewithargsret{\bfcode{test\_create\_strain\_minimal}}{}{}
This is a test for creating a new strain object, with only the minimum fields being entered

\end{fulllineitems}

\index{test\_strain\_absolute\_url() (mousedb.animal.tests.StrainModelTests method)}

\begin{fulllineitems}
\phantomsection\label{api:mousedb.animal.tests.StrainModelTests.test_strain_absolute_url}\pysiglinewithargsret{\bfcode{test\_strain\_absolute\_url}}{}{}
This is a test for creating a new strain object, then testing absolute url.

\end{fulllineitems}

\index{test\_strain\_unicode() (mousedb.animal.tests.StrainModelTests method)}

\begin{fulllineitems}
\phantomsection\label{api:mousedb.animal.tests.StrainModelTests.test_strain_unicode}\pysiglinewithargsret{\bfcode{test\_strain\_unicode}}{}{}
This is a test for creating a new strain object, then testing the unicode representation of the strain.

\end{fulllineitems}


\end{fulllineitems}

\index{StrainViewTests (class in mousedb.animal.tests)}

\begin{fulllineitems}
\phantomsection\label{api:mousedb.animal.tests.StrainViewTests}\pysiglinewithargsret{\strong{class }\code{mousedb.animal.tests.}\bfcode{StrainViewTests}}{\emph{methodName='runTest'}}{}
Bases: \code{django.test.testcases.TestCase}

Test the views contained in the animal app relating to Strain objects.
\index{fixtures (mousedb.animal.tests.StrainViewTests attribute)}

\begin{fulllineitems}
\phantomsection\label{api:mousedb.animal.tests.StrainViewTests.fixtures}\pysigline{\bfcode{fixtures}\strong{ = {[}'test\_strain'{]}}}
\end{fulllineitems}

\index{setUp() (mousedb.animal.tests.StrainViewTests method)}

\begin{fulllineitems}
\phantomsection\label{api:mousedb.animal.tests.StrainViewTests.setUp}\pysiglinewithargsret{\bfcode{setUp}}{}{}
Instantiate the test client.  Creates a test user.

\end{fulllineitems}

\index{tearDown() (mousedb.animal.tests.StrainViewTests method)}

\begin{fulllineitems}
\phantomsection\label{api:mousedb.animal.tests.StrainViewTests.tearDown}\pysiglinewithargsret{\bfcode{tearDown}}{}{}
Depopulate created model instances from test database.

\end{fulllineitems}

\index{test\_strain\_delete() (mousedb.animal.tests.StrainViewTests method)}

\begin{fulllineitems}
\phantomsection\label{api:mousedb.animal.tests.StrainViewTests.test_strain_delete}\pysiglinewithargsret{\bfcode{test\_strain\_delete}}{}{}
This tests the strain-delete view, ensuring that templates are loaded correctly.

This view uses a user with superuser permissions so does not test the permission levels for this view.

\end{fulllineitems}

\index{test\_strain\_detail() (mousedb.animal.tests.StrainViewTests method)}

\begin{fulllineitems}
\phantomsection\label{api:mousedb.animal.tests.StrainViewTests.test_strain_detail}\pysiglinewithargsret{\bfcode{test\_strain\_detail}}{}{}
This tests the strain-detail view, ensuring that templates are loaded correctly.

This view uses a user with superuser permissions so does not test the permission levels for this view.

\end{fulllineitems}

\index{test\_strain\_detail\_all() (mousedb.animal.tests.StrainViewTests method)}

\begin{fulllineitems}
\phantomsection\label{api:mousedb.animal.tests.StrainViewTests.test_strain_detail_all}\pysiglinewithargsret{\bfcode{test\_strain\_detail\_all}}{}{}
This tests the strain-detail-all view, ensuring that templates are loaded correctly.

This view uses a user with superuser permissions so does not test the permission levels for this view.

\end{fulllineitems}

\index{test\_strain\_edit() (mousedb.animal.tests.StrainViewTests method)}

\begin{fulllineitems}
\phantomsection\label{api:mousedb.animal.tests.StrainViewTests.test_strain_edit}\pysiglinewithargsret{\bfcode{test\_strain\_edit}}{}{}
This tests the strain-edit view, ensuring that templates are loaded correctly.

This view uses a user with superuser permissions so does not test the permission levels for this view.

\end{fulllineitems}

\index{test\_strain\_list() (mousedb.animal.tests.StrainViewTests method)}

\begin{fulllineitems}
\phantomsection\label{api:mousedb.animal.tests.StrainViewTests.test_strain_list}\pysiglinewithargsret{\bfcode{test\_strain\_list}}{}{}
This tests the strain-list view, ensuring that templates are loaded correctly.

This view uses a user with superuser permissions so does not test the permission levels for this view.

\end{fulllineitems}

\index{test\_strain\_new() (mousedb.animal.tests.StrainViewTests method)}

\begin{fulllineitems}
\phantomsection\label{api:mousedb.animal.tests.StrainViewTests.test_strain_new}\pysiglinewithargsret{\bfcode{test\_strain\_new}}{}{}
This tests the strain-new view, ensuring that templates are loaded correctly.

This view uses a user with superuser permissions so does not test the permission levels for this view.

\end{fulllineitems}


\end{fulllineitems}

\index{ToDoViewTests (class in mousedb.animal.tests)}

\begin{fulllineitems}
\phantomsection\label{api:mousedb.animal.tests.ToDoViewTests}\pysiglinewithargsret{\strong{class }\code{mousedb.animal.tests.}\bfcode{ToDoViewTests}}{\emph{methodName='runTest'}}{}
Bases: \code{django.test.testcases.TestCase}

Tests the views associated with animal objects for the three todo lists.
\index{fixtures (mousedb.animal.tests.ToDoViewTests attribute)}

\begin{fulllineitems}
\phantomsection\label{api:mousedb.animal.tests.ToDoViewTests.fixtures}\pysigline{\bfcode{fixtures}\strong{ = {[}'test\_animals', `test\_strain'{]}}}
\end{fulllineitems}

\index{setUp() (mousedb.animal.tests.ToDoViewTests method)}

\begin{fulllineitems}
\phantomsection\label{api:mousedb.animal.tests.ToDoViewTests.setUp}\pysiglinewithargsret{\bfcode{setUp}}{}{}
Instantiate the test client.  Creates a test user.

\end{fulllineitems}

\index{tearDown() (mousedb.animal.tests.ToDoViewTests method)}

\begin{fulllineitems}
\phantomsection\label{api:mousedb.animal.tests.ToDoViewTests.tearDown}\pysiglinewithargsret{\bfcode{tearDown}}{}{}
Depopulate created model instances from test database.

\end{fulllineitems}

\index{test\_eartag\_list() (mousedb.animal.tests.ToDoViewTests method)}

\begin{fulllineitems}
\phantomsection\label{api:mousedb.animal.tests.ToDoViewTests.test_eartag_list}\pysiglinewithargsret{\bfcode{test\_eartag\_list}}{}{}
This test checks the view which displays an animal list page showing animals which need to be eartagged.  It checks for the correct templates and status code.

\end{fulllineitems}

\index{test\_genotype\_list() (mousedb.animal.tests.ToDoViewTests method)}

\begin{fulllineitems}
\phantomsection\label{api:mousedb.animal.tests.ToDoViewTests.test_genotype_list}\pysiglinewithargsret{\bfcode{test\_genotype\_list}}{}{}
This test checks the view which displays an animal list page showing animals which need to be genotyped.  It checks for the correct templates and status code.

\end{fulllineitems}

\index{test\_no\_cage\_list() (mousedb.animal.tests.ToDoViewTests method)}

\begin{fulllineitems}
\phantomsection\label{api:mousedb.animal.tests.ToDoViewTests.test_no_cage_list}\pysiglinewithargsret{\bfcode{test\_no\_cage\_list}}{}{}
This test checks the view which displays an animal list page showing animals which need to have a cage entered.  It checks for the correct templates and status code.

\end{fulllineitems}

\index{test\_no\_rack\_list() (mousedb.animal.tests.ToDoViewTests method)}

\begin{fulllineitems}
\phantomsection\label{api:mousedb.animal.tests.ToDoViewTests.test_no_rack_list}\pysiglinewithargsret{\bfcode{test\_no\_rack\_list}}{}{}
This test checks the view which displays an animal list page showing animals which need to have a rack entered.  It checks for the correct templates and status code.

\end{fulllineitems}

\index{test\_todo\_home() (mousedb.animal.tests.ToDoViewTests method)}

\begin{fulllineitems}
\phantomsection\label{api:mousedb.animal.tests.ToDoViewTests.test_todo_home}\pysiglinewithargsret{\bfcode{test\_todo\_home}}{}{}
This test checks the view which displays a summary of the todo lists.  It checks for the correct templates and status code.

\end{fulllineitems}

\index{test\_wean\_list() (mousedb.animal.tests.ToDoViewTests method)}

\begin{fulllineitems}
\phantomsection\label{api:mousedb.animal.tests.ToDoViewTests.test_wean_list}\pysiglinewithargsret{\bfcode{test\_wean\_list}}{}{}
This test checks the view which displays an animal list page showing animals which need to be weaned.  It checks for the correct templates and status code.

\end{fulllineitems}


\end{fulllineitems}



\section{Timed Mating Package}
\label{api:timed-mating-package}\label{api:module-mousedb.timed_mating}\index{mousedb.timed\_mating (module)}
This package defines the timed\_mating app.

Timed matings are a specific type of breeding set.  
Generally, for these experiments a mating cage is set up and pregnancy is defined by a plug event.  
Based on this information, the age of an embryo can be estimated.  
When a breeding cage is defined, one option is to set this cage as a timed mating cage (ie Timed\_Mating=True).  
If this is the case, then a plug event can be registered and recorded for this mating set.  
If the mother gives birth then this cage is implicitly set as a normal breeding cage.


\subsection{Models}
\label{api:id8}\phantomsection\label{api:module-mousedb.timed_mating.models}\index{mousedb.timed\_mating.models (module)}
This defines the data model for the timed\_mating app.

Currently the only data model is for PlugEvents.
\index{PlugEvents (class in mousedb.timed\_mating.models)}

\begin{fulllineitems}
\phantomsection\label{api:mousedb.timed_mating.models.PlugEvents}\pysiglinewithargsret{\strong{class }\code{mousedb.timed\_mating.models.}\bfcode{PlugEvents}}{\emph{*args}, \emph{**kwargs}}{}
Bases: \code{django.db.models.base.Model}

This defines the model for PlugEvents.

A PlugEvent requires a date.  All other fields are optional.
Upon observation of a plug event, the PlugDate, Breeding Cage, Femalem, Male, Researcher and Notes can be set.
Upon sacrifice of the mother, then genotyped alive and dead embryos can be entered, along with the SacrificeDate, Researcher and Notes.
\index{Breeding (mousedb.timed\_mating.models.PlugEvents attribute)}

\begin{fulllineitems}
\phantomsection\label{api:mousedb.timed_mating.models.PlugEvents.Breeding}\pysigline{\bfcode{Breeding}}
\end{fulllineitems}

\index{PlugEvents.DoesNotExist}

\begin{fulllineitems}
\phantomsection\label{api:mousedb.timed_mating.models.PlugEvents.DoesNotExist}\pysigline{\strong{exception }\bfcode{DoesNotExist}}
Bases: \code{django.core.exceptions.ObjectDoesNotExist}

\end{fulllineitems}

\index{PlugEvents.MultipleObjectsReturned}

\begin{fulllineitems}
\phantomsection\label{api:mousedb.timed_mating.models.PlugEvents.MultipleObjectsReturned}\pysigline{\strong{exception }\code{PlugEvents.}\bfcode{MultipleObjectsReturned}}
Bases: \code{django.core.exceptions.MultipleObjectsReturned}

\end{fulllineitems}

\index{PlugFemale (mousedb.timed\_mating.models.PlugEvents attribute)}

\begin{fulllineitems}
\phantomsection\label{api:mousedb.timed_mating.models.PlugEvents.PlugFemale}\pysigline{\code{PlugEvents.}\bfcode{PlugFemale}}
\end{fulllineitems}

\index{PlugMale (mousedb.timed\_mating.models.PlugEvents attribute)}

\begin{fulllineitems}
\phantomsection\label{api:mousedb.timed_mating.models.PlugEvents.PlugMale}\pysigline{\code{PlugEvents.}\bfcode{PlugMale}}
\end{fulllineitems}

\index{Researcher (mousedb.timed\_mating.models.PlugEvents attribute)}

\begin{fulllineitems}
\phantomsection\label{api:mousedb.timed_mating.models.PlugEvents.Researcher}\pysigline{\code{PlugEvents.}\bfcode{Researcher}}
\end{fulllineitems}

\index{get\_absolute\_url() (mousedb.timed\_mating.models.PlugEvents method)}

\begin{fulllineitems}
\phantomsection\label{api:mousedb.timed_mating.models.PlugEvents.get_absolute_url}\pysiglinewithargsret{\code{PlugEvents.}\bfcode{get\_absolute\_url}}{\emph{*moreargs}, \emph{**morekwargs}}{}
The permalink for a plugevent is /mousedb/timed\_mating/plugs/\textbf{id}.

\end{fulllineitems}

\index{get\_next\_by\_PlugDate() (mousedb.timed\_mating.models.PlugEvents method)}

\begin{fulllineitems}
\phantomsection\label{api:mousedb.timed_mating.models.PlugEvents.get_next_by_PlugDate}\pysiglinewithargsret{\code{PlugEvents.}\bfcode{get\_next\_by\_PlugDate}}{\emph{*moreargs}, \emph{**morekwargs}}{}
\end{fulllineitems}

\index{get\_previous\_by\_PlugDate() (mousedb.timed\_mating.models.PlugEvents method)}

\begin{fulllineitems}
\phantomsection\label{api:mousedb.timed_mating.models.PlugEvents.get_previous_by_PlugDate}\pysiglinewithargsret{\code{PlugEvents.}\bfcode{get\_previous\_by\_PlugDate}}{\emph{*moreargs}, \emph{**morekwargs}}{}
\end{fulllineitems}

\index{objects (mousedb.timed\_mating.models.PlugEvents attribute)}

\begin{fulllineitems}
\phantomsection\label{api:mousedb.timed_mating.models.PlugEvents.objects}\pysigline{\code{PlugEvents.}\bfcode{objects}\strong{ = \textless{}django.db.models.manager.Manager object at 0x023302B0\textgreater{}}}
\end{fulllineitems}

\index{save() (mousedb.timed\_mating.models.PlugEvents method)}

\begin{fulllineitems}
\phantomsection\label{api:mousedb.timed_mating.models.PlugEvents.save}\pysiglinewithargsret{\code{PlugEvents.}\bfcode{save}}{}{}
Over-rides the default save function for PlugEvents.

If a sacrifice date is set for an object in this model, then Active is set to False.

\end{fulllineitems}


\end{fulllineitems}



\subsection{Forms}
\label{api:id9}\phantomsection\label{api:module-mousedb.timed_mating.forms}\index{mousedb.timed\_mating.forms (module)}
This package describes forms used by the Timed Mating app.
\index{BreedingPlugForm (class in mousedb.timed\_mating.forms)}

\begin{fulllineitems}
\phantomsection\label{api:mousedb.timed_mating.forms.BreedingPlugForm}\pysiglinewithargsret{\strong{class }\code{mousedb.timed\_mating.forms.}\bfcode{BreedingPlugForm}}{\emph{data=None}, \emph{files=None}, \emph{auto\_id='id\_\%s'}, \emph{prefix=None}, \emph{initial=None}, \emph{error\_class=\textless{}class `django.forms.util.ErrorList'\textgreater{}}, \emph{label\_suffix=':'}, \emph{empty\_permitted=False}, \emph{instance=None}}{}
Bases: \code{django.forms.models.ModelForm}

This form is used to enter Plug Events from a specific breeding cage.
\index{BreedingPlugForm.Meta (class in mousedb.timed\_mating.forms)}

\begin{fulllineitems}
\phantomsection\label{api:mousedb.timed_mating.forms.BreedingPlugForm.Meta}\pysigline{\strong{class }\bfcode{Meta}}~\index{exclude (mousedb.timed\_mating.forms.BreedingPlugForm.Meta attribute)}

\begin{fulllineitems}
\phantomsection\label{api:mousedb.timed_mating.forms.BreedingPlugForm.Meta.exclude}\pysigline{\bfcode{exclude}\strong{ = {[}'Breeding'{]}}}
\end{fulllineitems}

\index{model (mousedb.timed\_mating.forms.BreedingPlugForm.Meta attribute)}

\begin{fulllineitems}
\phantomsection\label{api:mousedb.timed_mating.forms.BreedingPlugForm.Meta.model}\pysigline{\bfcode{model}}
alias of \code{PlugEvents}

\end{fulllineitems}


\end{fulllineitems}

\index{base\_fields (mousedb.timed\_mating.forms.BreedingPlugForm attribute)}

\begin{fulllineitems}
\phantomsection\label{api:mousedb.timed_mating.forms.BreedingPlugForm.base_fields}\pysigline{\code{BreedingPlugForm.}\bfcode{base\_fields}\strong{ = \{`PlugDate': \textless{}django.forms.fields.DateField object at 0x0360BE50\textgreater{}, `PlugFemale': \textless{}django.forms.models.ModelChoiceField object at 0x02E53550\textgreater{}, `PlugMale': \textless{}django.forms.models.ModelChoiceField object at 0x0360BFF0\textgreater{}, `SacrificeDate': \textless{}django.forms.fields.DateField object at 0x0345CEB0\textgreater{}, `Researcher': \textless{}django.forms.models.ModelChoiceField object at 0x03624190\textgreater{}, `WT\_Alive': \textless{}django.forms.fields.IntegerField object at 0x03624170\textgreater{}, `HET\_Alive': \textless{}django.forms.fields.IntegerField object at 0x036241B0\textgreater{}, `KO\_Alive': \textless{}django.forms.fields.IntegerField object at 0x036241F0\textgreater{}, `WT\_Dead': \textless{}django.forms.fields.IntegerField object at 0x034671F0\textgreater{}, `HET\_Dead': \textless{}django.forms.fields.IntegerField object at 0x03624230\textgreater{}, `KO\_Dead': \textless{}django.forms.fields.IntegerField object at 0x03624270\textgreater{}, `Active': \textless{}django.forms.fields.BooleanField object at 0x0345CF90\textgreater{}, `Notes': \textless{}django.forms.fields.CharField object at 0x036242D0\textgreater{}\}}}
\end{fulllineitems}

\index{declared\_fields (mousedb.timed\_mating.forms.BreedingPlugForm attribute)}

\begin{fulllineitems}
\phantomsection\label{api:mousedb.timed_mating.forms.BreedingPlugForm.declared_fields}\pysigline{\code{BreedingPlugForm.}\bfcode{declared\_fields}\strong{ = \{\}}}
\end{fulllineitems}

\index{media (mousedb.timed\_mating.forms.BreedingPlugForm attribute)}

\begin{fulllineitems}
\phantomsection\label{api:mousedb.timed_mating.forms.BreedingPlugForm.media}\pysigline{\code{BreedingPlugForm.}\bfcode{media}}
\end{fulllineitems}


\end{fulllineitems}



\subsection{Views and URLs}
\label{api:id10}\phantomsection\label{api:module-mousedb.timed_mating.views}\index{mousedb.timed\_mating.views (module)}
This package defines custom views for the timed\_mating application.

Currently all views are generic CRUD views except for the view in which a plug event is defined from a breeding cage.
\index{PlugEventsCreate (class in mousedb.timed\_mating.views)}

\begin{fulllineitems}
\phantomsection\label{api:mousedb.timed_mating.views.PlugEventsCreate}\pysiglinewithargsret{\strong{class }\code{mousedb.timed\_mating.views.}\bfcode{PlugEventsCreate}}{\emph{**kwargs}}{}
Bases: {\hyperref[api:mousedb.views.RestrictedCreateView]{\code{mousedb.views.RestrictedCreateView}}}

This class generates the plugevents-new view.

This permission restricted view takes a url in the form \textbf{/plugs/new} and generates an empty plugevents\_form.html.
\index{model (mousedb.timed\_mating.views.PlugEventsCreate attribute)}

\begin{fulllineitems}
\phantomsection\label{api:mousedb.timed_mating.views.PlugEventsCreate.model}\pysigline{\bfcode{model}}
alias of \code{PlugEvents}

\end{fulllineitems}

\index{template\_name (mousedb.timed\_mating.views.PlugEventsCreate attribute)}

\begin{fulllineitems}
\phantomsection\label{api:mousedb.timed_mating.views.PlugEventsCreate.template_name}\pysigline{\bfcode{template\_name}\strong{ = `plugevents\_form.html'}}
\end{fulllineitems}


\end{fulllineitems}

\index{PlugEventsDelete (class in mousedb.timed\_mating.views)}

\begin{fulllineitems}
\phantomsection\label{api:mousedb.timed_mating.views.PlugEventsDelete}\pysiglinewithargsret{\strong{class }\code{mousedb.timed\_mating.views.}\bfcode{PlugEventsDelete}}{\emph{**kwargs}}{}
Bases: {\hyperref[api:mousedb.views.RestrictedDeleteView]{\code{mousedb.views.RestrictedDeleteView}}}

This class generates the plugevents-delete view.

This permission restricted view takes a url in the form \textbf{/plugs/\#/delete} and passes that object to the confirm\_delete.html page.
\index{context\_object\_name (mousedb.timed\_mating.views.PlugEventsDelete attribute)}

\begin{fulllineitems}
\phantomsection\label{api:mousedb.timed_mating.views.PlugEventsDelete.context_object_name}\pysigline{\bfcode{context\_object\_name}\strong{ = `plugevent'}}
\end{fulllineitems}

\index{model (mousedb.timed\_mating.views.PlugEventsDelete attribute)}

\begin{fulllineitems}
\phantomsection\label{api:mousedb.timed_mating.views.PlugEventsDelete.model}\pysigline{\bfcode{model}}
alias of \code{PlugEvents}

\end{fulllineitems}

\index{success\_url (mousedb.timed\_mating.views.PlugEventsDelete attribute)}

\begin{fulllineitems}
\phantomsection\label{api:mousedb.timed_mating.views.PlugEventsDelete.success_url}\pysigline{\bfcode{success\_url}\strong{ = `/mousedb/plugs/'}}
\end{fulllineitems}

\index{template\_name (mousedb.timed\_mating.views.PlugEventsDelete attribute)}

\begin{fulllineitems}
\phantomsection\label{api:mousedb.timed_mating.views.PlugEventsDelete.template_name}\pysigline{\bfcode{template\_name}\strong{ = `confirm\_delete.html'}}
\end{fulllineitems}


\end{fulllineitems}

\index{PlugEventsDetail (class in mousedb.timed\_mating.views)}

\begin{fulllineitems}
\phantomsection\label{api:mousedb.timed_mating.views.PlugEventsDetail}\pysiglinewithargsret{\strong{class }\code{mousedb.timed\_mating.views.}\bfcode{PlugEventsDetail}}{\emph{**kwargs}}{}
Bases: {\hyperref[api:mousedb.views.ProtectedDetailView]{\code{mousedb.views.ProtectedDetailView}}}

This class generates the plugevents-detail view.

This login protected takes a url in the form \textbf{/plugs/1} for plug event id=1 and passes a \textbf{plug} object to plugevents\_detail.html
\index{context\_object\_name (mousedb.timed\_mating.views.PlugEventsDetail attribute)}

\begin{fulllineitems}
\phantomsection\label{api:mousedb.timed_mating.views.PlugEventsDetail.context_object_name}\pysigline{\bfcode{context\_object\_name}\strong{ = `plugevent'}}
\end{fulllineitems}

\index{model (mousedb.timed\_mating.views.PlugEventsDetail attribute)}

\begin{fulllineitems}
\phantomsection\label{api:mousedb.timed_mating.views.PlugEventsDetail.model}\pysigline{\bfcode{model}}
alias of \code{PlugEvents}

\end{fulllineitems}

\index{template\_name (mousedb.timed\_mating.views.PlugEventsDetail attribute)}

\begin{fulllineitems}
\phantomsection\label{api:mousedb.timed_mating.views.PlugEventsDetail.template_name}\pysigline{\bfcode{template\_name}\strong{ = `plugevents\_detail.html'}}
\end{fulllineitems}


\end{fulllineitems}

\index{PlugEventsList (class in mousedb.timed\_mating.views)}

\begin{fulllineitems}
\phantomsection\label{api:mousedb.timed_mating.views.PlugEventsList}\pysiglinewithargsret{\strong{class }\code{mousedb.timed\_mating.views.}\bfcode{PlugEventsList}}{\emph{**kwargs}}{}
Bases: {\hyperref[api:mousedb.views.ProtectedListView]{\code{mousedb.views.ProtectedListView}}}

This class generates an object list for PlugEvent objects.

This login protected view takes all PlugEvents objects and sends them to plugevents\_list.html as a plug\_list dictionary.
The url for this view is \textbf{/plugs/}
\index{context\_object\_name (mousedb.timed\_mating.views.PlugEventsList attribute)}

\begin{fulllineitems}
\phantomsection\label{api:mousedb.timed_mating.views.PlugEventsList.context_object_name}\pysigline{\bfcode{context\_object\_name}\strong{ = `plugevents\_list'}}
\end{fulllineitems}

\index{model (mousedb.timed\_mating.views.PlugEventsList attribute)}

\begin{fulllineitems}
\phantomsection\label{api:mousedb.timed_mating.views.PlugEventsList.model}\pysigline{\bfcode{model}}
alias of \code{PlugEvents}

\end{fulllineitems}

\index{template\_name (mousedb.timed\_mating.views.PlugEventsList attribute)}

\begin{fulllineitems}
\phantomsection\label{api:mousedb.timed_mating.views.PlugEventsList.template_name}\pysigline{\bfcode{template\_name}\strong{ = `plugevents\_list.html'}}
\end{fulllineitems}


\end{fulllineitems}

\index{PlugEventsListStrain (class in mousedb.timed\_mating.views)}

\begin{fulllineitems}
\phantomsection\label{api:mousedb.timed_mating.views.PlugEventsListStrain}\pysiglinewithargsret{\strong{class }\code{mousedb.timed\_mating.views.}\bfcode{PlugEventsListStrain}}{\emph{**kwargs}}{}
Bases: {\hyperref[api:mousedb.timed_mating.views.PlugEventsList]{\code{mousedb.timed\_mating.views.PlugEventsList}}}

This class generates a strain filtered list for Plug Event objects.

This is a subclass of PlugEventsList and returns as context\_object\_name plug\_events\_list to plugevents\_list.html.
It takes a named argument (strain) which is a Strain\_slug and filters based on that strain.
\index{get\_queryset() (mousedb.timed\_mating.views.PlugEventsListStrain method)}

\begin{fulllineitems}
\phantomsection\label{api:mousedb.timed_mating.views.PlugEventsListStrain.get_queryset}\pysiglinewithargsret{\bfcode{get\_queryset}}{}{}
The queryset is over-ridden to show only plug events in which the strain matches the breeding strain.

\end{fulllineitems}


\end{fulllineitems}

\index{PlugEventsUpdate (class in mousedb.timed\_mating.views)}

\begin{fulllineitems}
\phantomsection\label{api:mousedb.timed_mating.views.PlugEventsUpdate}\pysiglinewithargsret{\strong{class }\code{mousedb.timed\_mating.views.}\bfcode{PlugEventsUpdate}}{\emph{**kwargs}}{}
Bases: {\hyperref[api:mousedb.views.RestrictedUpdateView]{\code{mousedb.views.RestrictedUpdateView}}}

This class generates the plugevents-edit view.

This permission restricted view takes a url in the form \textbf{/plugs/\#/edit} and generates a plugevents\_form.html with that object.
\index{context\_object\_name (mousedb.timed\_mating.views.PlugEventsUpdate attribute)}

\begin{fulllineitems}
\phantomsection\label{api:mousedb.timed_mating.views.PlugEventsUpdate.context_object_name}\pysigline{\bfcode{context\_object\_name}\strong{ = `plugevent'}}
\end{fulllineitems}

\index{model (mousedb.timed\_mating.views.PlugEventsUpdate attribute)}

\begin{fulllineitems}
\phantomsection\label{api:mousedb.timed_mating.views.PlugEventsUpdate.model}\pysigline{\bfcode{model}}
alias of \code{PlugEvents}

\end{fulllineitems}

\index{template\_name (mousedb.timed\_mating.views.PlugEventsUpdate attribute)}

\begin{fulllineitems}
\phantomsection\label{api:mousedb.timed_mating.views.PlugEventsUpdate.template_name}\pysigline{\bfcode{template\_name}\strong{ = `plugevents\_form.html'}}
\end{fulllineitems}


\end{fulllineitems}

\index{breeding\_plugevent() (in module mousedb.timed\_mating.views)}

\begin{fulllineitems}
\phantomsection\label{api:mousedb.timed_mating.views.breeding_plugevent}\pysiglinewithargsret{\code{mousedb.timed\_mating.views.}\bfcode{breeding\_plugevent}}{\emph{request}, \emph{*args}, \emph{**kwargs}}{}
This view defines a form for adding new plug events from a breeding cage.

This form requires a breeding\_id from a breeding set and restricts the PlugFemale and PlugMale to animals that are defined in that breeding cage.

\end{fulllineitems}

\phantomsection\label{api:module-mousedb.timed_mating.urls}\index{mousedb.timed\_mating.urls (module)}
This urlconf sets the directions for the timed\_mating app.

It takes a url in the form of /plug/something and sends it to the appropriate view class or function.


\subsection{Administrative Site Configuration}
\label{api:id11}\phantomsection\label{api:module-mousedb.timed_mating.admin}\index{mousedb.timed\_mating.admin (module)}
Settings to control the admin interface for the timed\_mating app.

This file defines a PlugEventsAdmin object to enter parameters about individual plug events/
\index{PlugEventsAdmin (class in mousedb.timed\_mating.admin)}

\begin{fulllineitems}
\phantomsection\label{api:mousedb.timed_mating.admin.PlugEventsAdmin}\pysiglinewithargsret{\strong{class }\code{mousedb.timed\_mating.admin.}\bfcode{PlugEventsAdmin}}{\emph{model}, \emph{admin\_site}}{}
Bases: \code{django.contrib.admin.options.ModelAdmin}

This class defines the admin interface for the PlugEvents model.
\index{list\_display (mousedb.timed\_mating.admin.PlugEventsAdmin attribute)}

\begin{fulllineitems}
\phantomsection\label{api:mousedb.timed_mating.admin.PlugEventsAdmin.list_display}\pysigline{\bfcode{list\_display}\strong{ = (`PlugDate', `PlugFemale', `PlugMale', `SacrificeDate', `Researcher', `Active')}}
\end{fulllineitems}

\index{media (mousedb.timed\_mating.admin.PlugEventsAdmin attribute)}

\begin{fulllineitems}
\phantomsection\label{api:mousedb.timed_mating.admin.PlugEventsAdmin.media}\pysigline{\bfcode{media}}
\end{fulllineitems}


\end{fulllineitems}



\subsection{Test Files}
\label{api:id12}\phantomsection\label{api:module-mousedb.timed_mating.tests}\index{mousedb.timed\_mating.tests (module)}
This file contains tests for the timed\_mating application.

These tests will verify generation of a new PlugEvent object.
\index{Timed\_MatingModelTests (class in mousedb.timed\_mating.tests)}

\begin{fulllineitems}
\phantomsection\label{api:mousedb.timed_mating.tests.Timed_MatingModelTests}\pysiglinewithargsret{\strong{class }\code{mousedb.timed\_mating.tests.}\bfcode{Timed\_MatingModelTests}}{\emph{methodName='runTest'}}{}
Bases: \code{django.test.testcases.TestCase}

Test the models contained in the `timed\_mating' app.
\index{setUp() (mousedb.timed\_mating.tests.Timed\_MatingModelTests method)}

\begin{fulllineitems}
\phantomsection\label{api:mousedb.timed_mating.tests.Timed_MatingModelTests.setUp}\pysiglinewithargsret{\bfcode{setUp}}{}{}
Instantiate the test client.  Creates a test user.

\end{fulllineitems}

\index{tearDown() (mousedb.timed\_mating.tests.Timed\_MatingModelTests method)}

\begin{fulllineitems}
\phantomsection\label{api:mousedb.timed_mating.tests.Timed_MatingModelTests.tearDown}\pysiglinewithargsret{\bfcode{tearDown}}{}{}
Depopulate created model instances from test database.

\end{fulllineitems}

\index{test\_create\_plugevent\_minimal() (mousedb.timed\_mating.tests.Timed\_MatingModelTests method)}

\begin{fulllineitems}
\phantomsection\label{api:mousedb.timed_mating.tests.Timed_MatingModelTests.test_create_plugevent_minimal}\pysiglinewithargsret{\bfcode{test\_create\_plugevent\_minimal}}{}{}
This is a test for creating a new PlugEvent object, with only the minimum being entered.

\end{fulllineitems}

\index{test\_create\_plugevent\_most\_fields() (mousedb.timed\_mating.tests.Timed\_MatingModelTests method)}

\begin{fulllineitems}
\phantomsection\label{api:mousedb.timed_mating.tests.Timed_MatingModelTests.test_create_plugevent_most_fields}\pysiglinewithargsret{\bfcode{test\_create\_plugevent\_most\_fields}}{}{}
This is a test for creating a new PlugEvent object.

This test uses a Breeding, PlugDate, PlugMale and PlugFemale field.

\end{fulllineitems}

\index{test\_set\_plugevent\_inactive() (mousedb.timed\_mating.tests.Timed\_MatingModelTests method)}

\begin{fulllineitems}
\phantomsection\label{api:mousedb.timed_mating.tests.Timed_MatingModelTests.test_set_plugevent_inactive}\pysiglinewithargsret{\bfcode{test\_set\_plugevent\_inactive}}{}{}
This is a test for the automatic inactivation of a cage when the SacrificeDate is entered.

\end{fulllineitems}


\end{fulllineitems}

\index{Timed\_MatingViewTests (class in mousedb.timed\_mating.tests)}

\begin{fulllineitems}
\phantomsection\label{api:mousedb.timed_mating.tests.Timed_MatingViewTests}\pysiglinewithargsret{\strong{class }\code{mousedb.timed\_mating.tests.}\bfcode{Timed\_MatingViewTests}}{\emph{methodName='runTest'}}{}
Bases: \code{django.test.testcases.TestCase}

Test the views contained in the `timed\_mating' app.
\index{fixtures (mousedb.timed\_mating.tests.Timed\_MatingViewTests attribute)}

\begin{fulllineitems}
\phantomsection\label{api:mousedb.timed_mating.tests.Timed_MatingViewTests.fixtures}\pysigline{\bfcode{fixtures}\strong{ = {[}'test\_breeding', `test\_plugevents', `test\_animals', `test\_strain'{]}}}
\end{fulllineitems}

\index{setUp() (mousedb.timed\_mating.tests.Timed\_MatingViewTests method)}

\begin{fulllineitems}
\phantomsection\label{api:mousedb.timed_mating.tests.Timed_MatingViewTests.setUp}\pysiglinewithargsret{\bfcode{setUp}}{}{}
Instantiate the test client.  Creates a test user.

\end{fulllineitems}

\index{tearDown() (mousedb.timed\_mating.tests.Timed\_MatingViewTests method)}

\begin{fulllineitems}
\phantomsection\label{api:mousedb.timed_mating.tests.Timed_MatingViewTests.tearDown}\pysiglinewithargsret{\bfcode{tearDown}}{}{}
Depopulate created model instances from test database.

\end{fulllineitems}

\index{test\_breeding\_plugevent\_new() (mousedb.timed\_mating.tests.Timed\_MatingViewTests method)}

\begin{fulllineitems}
\phantomsection\label{api:mousedb.timed_mating.tests.Timed_MatingViewTests.test_breeding_plugevent_new}\pysiglinewithargsret{\bfcode{test\_breeding\_plugevent\_new}}{}{}
This tests the breeding-plugevent-new view, ensuring that templates are loaded correctly.

This view uses a user with superuser permissions so does not test the permission levels for this view.

\end{fulllineitems}

\index{test\_plugevent\_delete() (mousedb.timed\_mating.tests.Timed\_MatingViewTests method)}

\begin{fulllineitems}
\phantomsection\label{api:mousedb.timed_mating.tests.Timed_MatingViewTests.test_plugevent_delete}\pysiglinewithargsret{\bfcode{test\_plugevent\_delete}}{}{}
This tests the plugevent-delete view, ensuring that templates are loaded correctly.

This view uses a user with superuser permissions so does not test the permission levels for this view.

\end{fulllineitems}

\index{test\_plugevent\_detail() (mousedb.timed\_mating.tests.Timed\_MatingViewTests method)}

\begin{fulllineitems}
\phantomsection\label{api:mousedb.timed_mating.tests.Timed_MatingViewTests.test_plugevent_detail}\pysiglinewithargsret{\bfcode{test\_plugevent\_detail}}{}{}
This tests the plugevent-detail view, ensuring that templates are loaded correctly.

This view uses a user with superuser permissions so does not test the permission levels for this view.

\end{fulllineitems}

\index{test\_plugevent\_edit() (mousedb.timed\_mating.tests.Timed\_MatingViewTests method)}

\begin{fulllineitems}
\phantomsection\label{api:mousedb.timed_mating.tests.Timed_MatingViewTests.test_plugevent_edit}\pysiglinewithargsret{\bfcode{test\_plugevent\_edit}}{}{}
This tests the plugevent-edit view, ensuring that templates are loaded correctly.

This view uses a user with superuser permissions so does not test the permission levels for this view.

\end{fulllineitems}

\index{test\_plugevent\_list() (mousedb.timed\_mating.tests.Timed\_MatingViewTests method)}

\begin{fulllineitems}
\phantomsection\label{api:mousedb.timed_mating.tests.Timed_MatingViewTests.test_plugevent_list}\pysiglinewithargsret{\bfcode{test\_plugevent\_list}}{}{}
This tests the plugevent-list view, ensuring that templates are loaded correctly.

This view uses a user with superuser permissions so does not test the permission levels for this view.

\end{fulllineitems}

\index{test\_plugevent\_list\_strain() (mousedb.timed\_mating.tests.Timed\_MatingViewTests method)}

\begin{fulllineitems}
\phantomsection\label{api:mousedb.timed_mating.tests.Timed_MatingViewTests.test_plugevent_list_strain}\pysiglinewithargsret{\bfcode{test\_plugevent\_list\_strain}}{}{}
This tests the plugevent-list-strain view, ensuring that templates are loaded correctly.

This view uses a user with superuser permissions so does not test the permission levels for this view.

\end{fulllineitems}

\index{test\_plugevent\_new() (mousedb.timed\_mating.tests.Timed\_MatingViewTests method)}

\begin{fulllineitems}
\phantomsection\label{api:mousedb.timed_mating.tests.Timed_MatingViewTests.test_plugevent_new}\pysiglinewithargsret{\bfcode{test\_plugevent\_new}}{}{}
This tests the plugevent-new view, ensuring that templates are loaded correctly.

This view uses a user with superuser permissions so does not test the permission levels for this view.

\end{fulllineitems}


\end{fulllineitems}



\section{Groups Package}
\label{api:groups-package}\label{api:module-mousedb.groups}\index{mousedb.groups (module)}
This package defines the Group application.
This app defines generic Group and License information for a particular installation of MouseDB.  
Because every page on this site identifies both the Group and data restrictions, at a minimum, group information must be provided upon installation (see installation instructions).


\subsection{Models}
\label{api:id13}\phantomsection\label{api:module-mousedb.groups.models}\index{mousedb.groups.models (module)}\index{Group (class in mousedb.groups.models)}

\begin{fulllineitems}
\phantomsection\label{api:mousedb.groups.models.Group}\pysiglinewithargsret{\strong{class }\code{mousedb.groups.models.}\bfcode{Group}}{\emph{*args}, \emph{**kwargs}}{}
Bases: \code{django.db.models.base.Model}

This defines the data structure for the Group model.

The only required field is group.
All other fields (group\_slug, group\_url, license, contact\_title, contact\_first, contact\_last and contact\_email) are optional.
\index{Group.DoesNotExist}

\begin{fulllineitems}
\phantomsection\label{api:mousedb.groups.models.Group.DoesNotExist}\pysigline{\strong{exception }\bfcode{DoesNotExist}}
Bases: \code{django.core.exceptions.ObjectDoesNotExist}

\end{fulllineitems}

\index{Group.MultipleObjectsReturned}

\begin{fulllineitems}
\phantomsection\label{api:mousedb.groups.models.Group.MultipleObjectsReturned}\pysigline{\strong{exception }\code{Group.}\bfcode{MultipleObjectsReturned}}
Bases: \code{django.core.exceptions.MultipleObjectsReturned}

\end{fulllineitems}

\index{get\_contact\_title\_display() (mousedb.groups.models.Group method)}

\begin{fulllineitems}
\phantomsection\label{api:mousedb.groups.models.Group.get_contact_title_display}\pysiglinewithargsret{\code{Group.}\bfcode{get\_contact\_title\_display}}{\emph{*moreargs}, \emph{**morekwargs}}{}
\end{fulllineitems}

\index{license (mousedb.groups.models.Group attribute)}

\begin{fulllineitems}
\phantomsection\label{api:mousedb.groups.models.Group.license}\pysigline{\code{Group.}\bfcode{license}}
\end{fulllineitems}

\index{objects (mousedb.groups.models.Group attribute)}

\begin{fulllineitems}
\phantomsection\label{api:mousedb.groups.models.Group.objects}\pysigline{\code{Group.}\bfcode{objects}\strong{ = \textless{}django.db.models.manager.Manager object at 0x02330890\textgreater{}}}
\end{fulllineitems}


\end{fulllineitems}

\index{License (class in mousedb.groups.models)}

\begin{fulllineitems}
\phantomsection\label{api:mousedb.groups.models.License}\pysiglinewithargsret{\strong{class }\code{mousedb.groups.models.}\bfcode{License}}{\emph{*args}, \emph{**kwargs}}{}
Bases: \code{django.db.models.base.Model}

This defines the data structure for the License model.

The only required field is license.
If the contents of this installation are being made available using some licencing criteria this can either be defined in the notes field, or in an external website.
\index{License.DoesNotExist}

\begin{fulllineitems}
\phantomsection\label{api:mousedb.groups.models.License.DoesNotExist}\pysigline{\strong{exception }\bfcode{DoesNotExist}}
Bases: \code{django.core.exceptions.ObjectDoesNotExist}

\end{fulllineitems}

\index{License.MultipleObjectsReturned}

\begin{fulllineitems}
\phantomsection\label{api:mousedb.groups.models.License.MultipleObjectsReturned}\pysigline{\strong{exception }\code{License.}\bfcode{MultipleObjectsReturned}}
Bases: \code{django.core.exceptions.MultipleObjectsReturned}

\end{fulllineitems}

\index{group\_set (mousedb.groups.models.License attribute)}

\begin{fulllineitems}
\phantomsection\label{api:mousedb.groups.models.License.group_set}\pysigline{\code{License.}\bfcode{group\_set}}
\end{fulllineitems}

\index{objects (mousedb.groups.models.License attribute)}

\begin{fulllineitems}
\phantomsection\label{api:mousedb.groups.models.License.objects}\pysigline{\code{License.}\bfcode{objects}\strong{ = \textless{}django.db.models.manager.Manager object at 0x02330AD0\textgreater{}}}
\end{fulllineitems}


\end{fulllineitems}



\subsection{Views and URLs}
\label{api:id14}\phantomsection\label{api:module-mousedb.groups.views}\index{mousedb.groups.views (module)}

\subsection{Administrative Site Configuration}
\label{api:id15}\phantomsection\label{api:module-mousedb.groups.admin}\index{mousedb.groups.admin (module)}\index{GroupAdmin (class in mousedb.groups.admin)}

\begin{fulllineitems}
\phantomsection\label{api:mousedb.groups.admin.GroupAdmin}\pysiglinewithargsret{\strong{class }\code{mousedb.groups.admin.}\bfcode{GroupAdmin}}{\emph{model}, \emph{admin\_site}}{}
Bases: \code{django.contrib.admin.options.ModelAdmin}

Defines the admin interface for Groups.

Currently set as default.
\index{media (mousedb.groups.admin.GroupAdmin attribute)}

\begin{fulllineitems}
\phantomsection\label{api:mousedb.groups.admin.GroupAdmin.media}\pysigline{\bfcode{media}}
\end{fulllineitems}


\end{fulllineitems}

\index{LicenseAdmin (class in mousedb.groups.admin)}

\begin{fulllineitems}
\phantomsection\label{api:mousedb.groups.admin.LicenseAdmin}\pysiglinewithargsret{\strong{class }\code{mousedb.groups.admin.}\bfcode{LicenseAdmin}}{\emph{model}, \emph{admin\_site}}{}
Bases: \code{django.contrib.admin.options.ModelAdmin}

Defines the admin interface for Licences.

Currently set as default.
\index{media (mousedb.groups.admin.LicenseAdmin attribute)}

\begin{fulllineitems}
\phantomsection\label{api:mousedb.groups.admin.LicenseAdmin.media}\pysigline{\bfcode{media}}
\end{fulllineitems}


\end{fulllineitems}

\index{LogEntryAdmin (class in mousedb.groups.admin)}

\begin{fulllineitems}
\phantomsection\label{api:mousedb.groups.admin.LogEntryAdmin}\pysiglinewithargsret{\strong{class }\code{mousedb.groups.admin.}\bfcode{LogEntryAdmin}}{\emph{model}, \emph{admin\_site}}{}
Bases: \code{django.contrib.admin.options.ModelAdmin}

Defines the admin interface for the LogEntry objects.
\index{fields (mousedb.groups.admin.LogEntryAdmin attribute)}

\begin{fulllineitems}
\phantomsection\label{api:mousedb.groups.admin.LogEntryAdmin.fields}\pysigline{\bfcode{fields}\strong{ = (`user', `content\_type', `object\_id', `action\_flag', `object\_repr', `change\_message')}}
\end{fulllineitems}

\index{list\_display (mousedb.groups.admin.LogEntryAdmin attribute)}

\begin{fulllineitems}
\phantomsection\label{api:mousedb.groups.admin.LogEntryAdmin.list_display}\pysigline{\bfcode{list\_display}\strong{ = (`user', `content\_type', `object\_id', `action\_time')}}
\end{fulllineitems}

\index{media (mousedb.groups.admin.LogEntryAdmin attribute)}

\begin{fulllineitems}
\phantomsection\label{api:mousedb.groups.admin.LogEntryAdmin.media}\pysigline{\bfcode{media}}
\end{fulllineitems}


\end{fulllineitems}



\subsection{Test Files}
\label{api:id16}\phantomsection\label{api:module-mousedb.groups.tests}\index{mousedb.groups.tests (module)}
This file contains tests for the groups application.

These tests will verify generation of a new group and license object.
\index{GroupsModelTests (class in mousedb.groups.tests)}

\begin{fulllineitems}
\phantomsection\label{api:mousedb.groups.tests.GroupsModelTests}\pysiglinewithargsret{\strong{class }\code{mousedb.groups.tests.}\bfcode{GroupsModelTests}}{\emph{methodName='runTest'}}{}
Bases: \code{django.test.testcases.TestCase}

Test the models contained in the `groups' app.
\index{setUp() (mousedb.groups.tests.GroupsModelTests method)}

\begin{fulllineitems}
\phantomsection\label{api:mousedb.groups.tests.GroupsModelTests.setUp}\pysiglinewithargsret{\bfcode{setUp}}{}{}
Instantiate the test client.

\end{fulllineitems}

\index{tearDown() (mousedb.groups.tests.GroupsModelTests method)}

\begin{fulllineitems}
\phantomsection\label{api:mousedb.groups.tests.GroupsModelTests.tearDown}\pysiglinewithargsret{\bfcode{tearDown}}{}{}
Depopulate created model instances from test database.

\end{fulllineitems}

\index{test\_create\_group\_all\_fields() (mousedb.groups.tests.GroupsModelTests method)}

\begin{fulllineitems}
\phantomsection\label{api:mousedb.groups.tests.GroupsModelTests.test_create_group_all_fields}\pysiglinewithargsret{\bfcode{test\_create\_group\_all\_fields}}{}{}
This is a test for creating a new group object, with all fields being entered, except license.

\end{fulllineitems}

\index{test\_create\_group\_minimal() (mousedb.groups.tests.GroupsModelTests method)}

\begin{fulllineitems}
\phantomsection\label{api:mousedb.groups.tests.GroupsModelTests.test_create_group_minimal}\pysiglinewithargsret{\bfcode{test\_create\_group\_minimal}}{}{}
This is a test for creating a new group object, with only the minimum being entered.

\end{fulllineitems}

\index{test\_create\_license\_all\_fields() (mousedb.groups.tests.GroupsModelTests method)}

\begin{fulllineitems}
\phantomsection\label{api:mousedb.groups.tests.GroupsModelTests.test_create_license_all_fields}\pysiglinewithargsret{\bfcode{test\_create\_license\_all\_fields}}{}{}
This is a test for creating a new license object, with all fields being entered.

\end{fulllineitems}

\index{test\_create\_license\_minimal() (mousedb.groups.tests.GroupsModelTests method)}

\begin{fulllineitems}
\phantomsection\label{api:mousedb.groups.tests.GroupsModelTests.test_create_license_minimal}\pysiglinewithargsret{\bfcode{test\_create\_license\_minimal}}{}{}
This is a test for creating a new license object, with only the minimum being entered.

\end{fulllineitems}


\end{fulllineitems}



\chapter{Indices and tables}
\label{index:indices-and-tables}\begin{itemize}
\item {} 
\emph{genindex}

\item {} 
\emph{modindex}

\item {} 
\emph{search}

\end{itemize}


\renewcommand{\indexname}{Python Module Index}
\begin{theindex}
\def\bigletter#1{{\Large\sffamily#1}\nopagebreak\vspace{1mm}}
\bigletter{m}
\item {\texttt{mousedb}}, \pageref{api:module-mousedb}
\item {\texttt{mousedb.animal}}, \pageref{api:module-mousedb.animal}
\item {\texttt{mousedb.animal.admin}}, \pageref{api:module-mousedb.animal.admin}
\item {\texttt{mousedb.animal.forms}}, \pageref{api:module-mousedb.animal.forms}
\item {\texttt{mousedb.animal.models}}, \pageref{api:module-mousedb.animal.models}
\item {\texttt{mousedb.animal.tests}}, \pageref{api:module-mousedb.animal.tests}
\item {\texttt{mousedb.animal.urls}}, \pageref{api:module-mousedb.animal.urls}
\item {\texttt{mousedb.animal.views}}, \pageref{api:module-mousedb.animal.views}
\item {\texttt{mousedb.data}}, \pageref{api:module-mousedb.data}
\item {\texttt{mousedb.data.admin}}, \pageref{api:module-mousedb.data.admin}
\item {\texttt{mousedb.data.forms}}, \pageref{api:module-mousedb.data.forms}
\item {\texttt{mousedb.data.models}}, \pageref{api:module-mousedb.data.models}
\item {\texttt{mousedb.data.tests}}, \pageref{api:module-mousedb.data.tests}
\item {\texttt{mousedb.data.urls}}, \pageref{api:module-mousedb.data.urls}
\item {\texttt{mousedb.data.views}}, \pageref{api:module-mousedb.data.views}
\item {\texttt{mousedb.groups}}, \pageref{api:module-mousedb.groups}
\item {\texttt{mousedb.groups.admin}}, \pageref{api:module-mousedb.groups.admin}
\item {\texttt{mousedb.groups.models}}, \pageref{api:module-mousedb.groups.models}
\item {\texttt{mousedb.groups.tests}}, \pageref{api:module-mousedb.groups.tests}
\item {\texttt{mousedb.groups.views}}, \pageref{api:module-mousedb.groups.views}
\item {\texttt{mousedb.tests}}, \pageref{api:module-mousedb.tests}
\item {\texttt{mousedb.timed\_mating}}, \pageref{api:module-mousedb.timed_mating}
\item {\texttt{mousedb.timed\_mating.admin}}, \pageref{api:module-mousedb.timed_mating.admin}
\item {\texttt{mousedb.timed\_mating.forms}}, \pageref{api:module-mousedb.timed_mating.forms}
\item {\texttt{mousedb.timed\_mating.models}}, \pageref{api:module-mousedb.timed_mating.models}
\item {\texttt{mousedb.timed\_mating.tests}}, \pageref{api:module-mousedb.timed_mating.tests}
\item {\texttt{mousedb.timed\_mating.urls}}, \pageref{api:module-mousedb.timed_mating.urls}
\item {\texttt{mousedb.timed\_mating.views}}, \pageref{api:module-mousedb.timed_mating.views}
\item {\texttt{mousedb.urls}}, \pageref{api:module-mousedb.urls}
\item {\texttt{mousedb.views}}, \pageref{api:module-mousedb.views}
\end{theindex}

\renewcommand{\indexname}{Index}
\printindex
\end{document}
